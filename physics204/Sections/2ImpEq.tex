\section{Equations You Need to Know!!!}
\subsection{Thermodynamics}
\begin{align}
\frac{\Delta L}{L}=\alpha\Delta T\\
Q=mc\Delta T\\
Q=mL\\
PV=nRT\\
PV=Nk_bT\\
P=\eta k_BT\\
v_{rms}=\sqrt{\frac{3RT}{M}}=\sqrt{\frac{3k_BT}{M}}\\
K_{avg}=\frac{3}{2}k_BT\\
E_{int}=NK_{avg}\\
\lambda=\frac{1}{\pi d^2 \eta}\\
E_{int}=nC_VT\\
\Delta E_{int} = nC_V\Delta T\\
C_V=\frac{d}{2}R=\left(\frac{3}{2}R\right)_{mon}=\left(\frac{5}{2}R\right)\\
C_P=C_V+R\\
\gamma = \frac{C_P}{C_V}\\
PV^{\gamma}=constant\\
W=\int{PdV}\\
\Delta E_{int}=Q-W\\
\Delta E_{int}=nC_P\Delta T - nR\Delta T\\
\end{align}
\newpage

\subsection{Electricity}
\begin{align}
\vec{F_{12}}=\frac{1}{4\pi\epsilon_0}\frac{q_1q_2}{r_{12}^2}\hat{r}_{12}\\
\vec{E_1}=\frac{1}{4\pi\epsilon_0}\frac{q_1}{r^2}\hat{r}\\
\vec{F_{12}}=q_2\vec{E_1}\\
dQ=\lambda dl\\
dQ=\sigma dA\\
dQ=\rho dV\\
d\vec{E}=\frac{1}{4\pi\epsilon_0}\frac{dQ}{r^2}\hat{r}\\
\Phi_E=\int \vec{E}\cdot d\vec{A}\\
\oint\vec{E}\cdot d\vec{A}=\frac{Q_{enc}}{\epsilon_0}\\
EA=\frac{Q_{enc}}{\epsilon_0}\\
U_{12}=\frac{1}{4\pi\epsilon_0}\frac{q_1q_2}{r_{12}}\\
V_1=\frac{U_{12}}{q_2}\\
V_1=\frac{1}{4\pi\epsilon_0}\frac{q_1}{r}\\
dV=\frac{1}{4\pi\epsilon_0}\frac{dQ}{r}\\
V_f-V_i=-\int_{s_i}^{s_f}\vec{E}\cdot d\vec{s}\\
E_s=-\frac{dV}{ds}
\end{align}
\newpage
\subsubsection{A little note about Volume Integrals}
In Cartesian coordinates, $dV=dxdydz$. In Cylindrical coordinates, you have a z axis, a $\Phi$ axis, and an r axis. A differential amount of radius is dr, a differential amount of z is dz, and a differential amount of $\Phi$ is $d\Phi$. $d\Phi$ is really just a certain amount of arclength that is covered. We can call this arclength, $dl_\Phi$.
\begin{align}
    dV&=drdl_\Phi dz\\
    \shortintertext{So we can sub in arclength for $dl_\Phi$}\\
    &=dr(rd\Phi)dz\\
    dV&=rdrd\Phi dz\\
\end{align}
Now for spherical coordinates, we have 2 angles and a radius. $\theta$ is wrapping around the radius of the circle, where $\Phi$ is going from the north pole to the south pole. The radius is just r.
\begin{align*}
    dV&=drdl_\theta dl_\Phi\\
    &=dr(rd\theta)(rsin\theta d\Phi)\\
    dV&=r^2sin\theta drd\theta d\Phi\\
\end{align*}
\newpage

\subsection{Circuits}
\begin{align}
    V=IR\\
    R_{eq in series}=R_1+R_2+R_3+...\\
    \frac{1}{R_{eq in para}}=\frac{1}{R_1}+\frac{1}{R_2}+\frac{1}{R_3}+...\\
    \shortintertext{Power}\\
    P=V*I\\
    P=R*I^2\\
    P=\frac{V^2}{R}\\
    \shortintertext{Voltage}\\
    V=R*I\\
    V=\frac{P}{I}\\
    V=\sqrt{P*R}\\
    \shortintertext{Resisitance}\\
    R=\frac{V}{I}\\
    R=\frac{V^2}{P}\\
    R=\frac{P}{I^2}\\
    \shortintertext{Current}\\
    I=\frac{V}{R^2}\\
    I=\frac{P}{V}\\
    I=\sqrt{\frac{P}{R}}
\end{align}
\newpage

\subsection{Magnetism}
\begin{align}
    F=q\vec{v}\times\vec{B}\\
    F=I\vec{l}\times\vec{B}\\
    \vec{B}=\frac{\mu_0}{4\pi}\frac{q\vec{v}\times\hat{r}}{r^2}\\
    \oint\vec{B}\cdot d\vec{l}=\mu_0I_{th}\\
    E_{ind}=-\frac{d}{dt}\Phi_B\\
    \oint\vec{E}\cdot d\vec{l}=-\frac{d}{dt}\int\vec{B}\cdot d\vec{a}\\
\end{align}

\subsection{Optics}
\begin{align*}
    E=E_0sin(kx-\omega t +\phi_0)\\
    B=B_0sin(kx-\omega t +\phi_0)\\
    \vec{S}=\frac{1}{\mu_0}\vec{E}\times\vec{B}\\
    c=\frac{1}{\sqrt{\epsilon_0\mu_0}}\\
    c=\lambda f\\
    E=cB\\
    \mathscr{I}=\frac{P}{A}=\frac{\mathscr{E}}{At}\\
    \mathscr{I}=S_{avg}=\frac{1}{\mu_0}E_{rms}B_{rms}\\
    E_{rms}^2=\frac{E_0^2}{2}\\
    B_{rms}=\frac{B_0^2}{2}\\
    \mathscr{P}=\frac{\mathscr{I}}{c}\text{ or }2\frac{\mathscr{I}}{c}\\
    \theta_1=\theta_1'\\
    n_1sin\theta_1=n_2sin\theta_2\\
    \frac{1}{f}=\frac{1}{d_0}+\frac{1}{d_i}\\
    m=\frac{h_i}{h_0}=\frac{-d_i}{d_0}\\
\end{align*}
\newpage