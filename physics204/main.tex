\documentclass[letterpaper]{article}
\usepackage{fancyhdr}
\usepackage[margin=1.25in]{geometry}
\usepackage{comment}
\usepackage{color}
\usepackage{hyperref}
\hypersetup{
	colorlinks=true,
	linktoc=all,
	linkcolor=black
}

%Math
\usepackage{amsmath}
\usepackage{amsfonts}
\usepackage{amssymb}
\usepackage{amsthm}
\usepackage{mathtools}
\usepackage{mathrsfs}

%Figures
\usepackage{siunitx}
\usepackage{tikz}
\usepackage{pgfplots}
\usepackage{circuitikz}

%Defining \alignedbox command
\newlength\dlf
\newcommand\alignedbox[2]{
	% #1 = before alignment
	% #2 = after alignment
	&
	\begingroup
	\settowidth\dlf{$\displaystyle #1$}
	\addtolength\dlf{\fboxsep+\fboxrule}
	\hspace{-\dlf}
	\boxed{#1 #2}
	\endgroup
}

\title{Physics 204 Notes}
\date{Fall 2020}
\author{Cameron Williamson}

\begin{document}
\maketitle
\tableofcontents
\newpage

\section{SI Units}
  \subsection{Base Units}
  \begin{itemize}
    \item Length - meter - m
    \item Mass - kilogram - kg
    \item Time - second - s
    \item Electric Current - ampere - A
    \item Thermodynamic Temperature - kelvin - K
    \item Amount of substance - mole - mol
    \item Luminous intensity - candela - cd
  \end{itemize}
  \subsection{Derived Units}
  \begin{itemize}
    \item Frequency - hertz - Hz - $s^{-1}$
    \item Force - newton - N - $m*kg*s^{-2}$
    \item Pressure - pascal - Pa - $\frac{N}{m^2}$
    \item Energy - joule - J - $N*m$
    \item Power - watt - W - $\frac{J}{s}$
    \item Electric charge - coulomb - C - $s*A$
    \item Electric potential - volt - V - $\frac{W}{A}$
    \item Electric resistance - ohm - $\Omega$ - $\frac{V}{A}$
    \item Celsius temperature - degree Celsius - $\si{\degree}C$ - $K-272.15 $
  \end{itemize}
\newpage
\section{Equations You Need to Know!!!}
\subsection{Thermodynamics}
\begin{align}
\frac{\Delta L}{L}=\alpha\Delta T\\
Q=mc\Delta T\\
Q=mL\\
PV=nRT\\
PV=Nk_bT\\
P=\eta k_BT\\
v_{rms}=\sqrt{\frac{3RT}{M}}=\sqrt{\frac{3k_BT}{M}}\\
K_{avg}=\frac{3}{2}k_BT\\
E_{int}=NK_{avg}\\
\lambda=\frac{1}{\pi d^2 \eta}\\
E_{int}=nC_VT\\
\Delta E_{int} = nC_V\Delta T\\
C_V=\frac{d}{2}R=\left(\frac{3}{2}R\right)_{mon}=\left(\frac{5}{2}R\right)\\
C_P=C_V+R\\
\gamma = \frac{C_P}{C_V}\\
PV^{\gamma}=constant\\
W=\int{PdV}\\
\Delta E_{int}=Q-W\\
\Delta E_{int}=nC_P\Delta T - nR\Delta T\\
\end{align}
\newpage

\subsection{Electricity}
\begin{align}
\vec{F_{12}}=\frac{1}{4\pi\epsilon_0}\frac{q_1q_2}{r_{12}^2}\hat{r}_{12}\\
\vec{E_1}=\frac{1}{4\pi\epsilon_0}\frac{q_1}{r^2}\hat{r}\\
\vec{F_{12}}=q_2\vec{E_1}\\
dQ=\lambda dl\\
dQ=\sigma dA\\
dQ=\rho dV\\
d\vec{E}=\frac{1}{4\pi\epsilon_0}\frac{dQ}{r^2}\hat{r}\\
\Phi_E=\int \vec{E}\cdot d\vec{A}\\
\oint\vec{E}\cdot d\vec{A}=\frac{Q_{enc}}{\epsilon_0}\\
EA=\frac{Q_{enc}}{\epsilon_0}\\
U_{12}=\frac{1}{4\pi\epsilon_0}\frac{q_1q_2}{r_{12}}\\
V_1=\frac{U_{12}}{q_2}\\
V_1=\frac{1}{4\pi\epsilon_0}\frac{q_1}{r}\\
dV=\frac{1}{4\pi\epsilon_0}\frac{dQ}{r}\\
V_f-V_i=-\int_{s_i}^{s_f}\vec{E}\cdot d\vec{s}\\
E_s=-\frac{dV}{ds}
\end{align}
\newpage
\subsubsection{A little note about Volume Integrals}
In Cartesian coordinates, $dV=dxdydz$. In Cylindrical coordinates, you have a z axis, a $\Phi$ axis, and an r axis. A differential amount of radius is dr, a differential amount of z is dz, and a differential amount of $\Phi$ is $d\Phi$. $d\Phi$ is really just a certain amount of arclength that is covered. We can call this arclength, $dl_\Phi$.
\begin{align}
    dV&=drdl_\Phi dz\\
    \shortintertext{So we can sub in arclength for $dl_\Phi$}\\
    &=dr(rd\Phi)dz\\
    dV&=rdrd\Phi dz\\
\end{align}
Now for spherical coordinates, we have 2 angles and a radius. $\theta$ is wrapping around the radius of the circle, where $\Phi$ is going from the north pole to the south pole. The radius is just r.
\begin{align*}
    dV&=drdl_\theta dl_\Phi\\
    &=dr(rd\theta)(rsin\theta d\Phi)\\
    dV&=r^2sin\theta drd\theta d\Phi\\
\end{align*}
\newpage

\subsection{Circuits}
\begin{align}
    V=IR\\
    R_{eq in series}=R_1+R_2+R_3+...\\
    \frac{1}{R_{eq in para}}=\frac{1}{R_1}+\frac{1}{R_2}+\frac{1}{R_3}+...\\
    \shortintertext{Power}\\
    P=V*I\\
    P=R*I^2\\
    P=\frac{V^2}{R}\\
    \shortintertext{Voltage}\\
    V=R*I\\
    V=\frac{P}{I}\\
    V=\sqrt{P*R}\\
    \shortintertext{Resisitance}\\
    R=\frac{V}{I}\\
    R=\frac{V^2}{P}\\
    R=\frac{P}{I^2}\\
    \shortintertext{Current}\\
    I=\frac{V}{R^2}\\
    I=\frac{P}{V}\\
    I=\sqrt{\frac{P}{R}}
\end{align}
\newpage

\subsection{Magnetism}
\begin{align}
    F=q\vec{v}\times\vec{B}\\
    F=I\vec{l}\times\vec{B}\\
    \vec{B}=\frac{\mu_0}{4\pi}\frac{q\vec{v}\times\hat{r}}{r^2}\\
    \oint\vec{B}\cdot d\vec{l}=\mu_0I_{th}\\
    E_{ind}=-\frac{d}{dt}\Phi_B\\
    \oint\vec{E}\cdot d\vec{l}=-\frac{d}{dt}\int\vec{B}\cdot d\vec{a}\\
\end{align}

\subsection{Optics}
\begin{align*}
    E=E_0sin(kx-\omega t +\phi_0)\\
    B=B_0sin(kx-\omega t +\phi_0)\\
    \vec{S}=\frac{1}{\mu_0}\vec{E}\times\vec{B}\\
    c=\frac{1}{\sqrt{\epsilon_0\mu_0}}\\
    c=\lambda f\\
    E=cB\\
    \mathscr{I}=\frac{P}{A}=\frac{\mathscr{E}}{At}\\
    \mathscr{I}=S_{avg}=\frac{1}{\mu_0}E_{rms}B_{rms}\\
    E_{rms}^2=\frac{E_0^2}{2}\\
    B_{rms}=\frac{B_0^2}{2}\\
    \mathscr{P}=\frac{\mathscr{I}}{c}\text{ or }2\frac{\mathscr{I}}{c}\\
    \theta_1=\theta_1'\\
    n_1sin\theta_1=n_2sin\theta_2\\
    \frac{1}{f}=\frac{1}{d_0}+\frac{1}{d_i}\\
    m=\frac{h_i}{h_0}=\frac{-d_i}{d_0}\\
\end{align*}
\newpage
\section{Gasses And Thermodynamics}
	\subsection{Temperature}
	The measure of hot and cold. More scientifically this is the measure of microscopic kinetic energy
	\newline
	The temperature scales are Kelvin (K) which has a zero point of absolute 0, Celsius (C), which has a zero point of water freezing and has 100 at the boiling point of water, and fahrenheit (F).
	\newline
	\begin{align*}
	T_C &= T_k - 273.15\\
	T_F &= \frac{9}{5}T_C+32
	\end{align*}
	\subsection{Thermal Equilibrium}
	Two objects are in thermal equilibrium when, while in direct contact, they have the same temperature. When 2 objects have different temperatures, $T_A$ and $T_B$ are brought into thermal contact and they will exchange energy via heat transfer until they reach thermal equilibrium.
	\newline
	\newline
	Heat transfer is the spontaneous exchange of microscopic energy due to molecular or atomic collisions.
	\subsection{Thermal Expansion}
	Thermal expansion is when a solid object expands when it receives a raise in temperature
	\begin{align*}
	\frac{dL}{dT} = \alpha L\\
	\frac{dA}{dT} = 2\alpha A\\
	\frac{dV}{dT} = 3\alpha V
	\end{align*}
	Where $\alpha$ is the coefficient of linear expansion
	\subsection{Calorimetry}
	Calorimetry is the science of measuring heat transfer. Internal/Thermal energy is the amoung of energy stored in an object as described by its temperature.
	Heat Transfer = Q
	\subsection{First Law of Thermodynamics}
	\begin{align*}
	\Delta E_{int} &= Q-W\\
	dE_{int} &= dQ-dQ\\
	Q &=c \Delta T\\
	c &= \frac{C}{M}\\
	Q &= mc\Delta T
	\end{align*}
	Where c is the specific heat capacity, m is the mass of the substance, and $\Delta T$ is the change in temperature.
	\subsection{Phases of Matter}
	The phases of matter (from hottest to coldest) are Plasma, Gas (condenses to, or evaporates from), Liquid (freezes to, or melts from), Solid, and Condensates. During a phase change $Q=mL_v$ or $Q=mL_v$
	\subsubsection{Example 1}
	The U-district bridge has a span of 450ft. How much will it expand in spokane. The bridge is made out of concrete and steel which both have $\alpha = 12*10^{-6}\frac{1}{C\si{\degree}}$. $T_{high}=110\si{\degree}F \to 120\si{\degree}F$, $T_{low}=-30\si{\degree}F \to -40\si{\degree}F$
	\begin{align*}
	t_{high}&=489\si{\degree}C\\
	T_{low}&=-40\si{\degree}C
	L&=137m
	\end{align*}
	\newline
	\newline
	\begin{align*}
	\frac{\Delta L}{\Delta T}&=\alpha L\\
	\Delta L&=\alpha L\Delta T\\
	&=\left(12*10^{-6}\frac{1}{C\si{\degree}}\right)(137m)(48.9\si{\degree}C--40\si{\degree}C)\\
	&=0.146m
	\end{align*}
	\subsubsection{Example 2}
	Your freezer is set to $23\si{\degree}F$, you remove a 26g ice cube and place it in an empty glass. The next day the ice has melted and come to $72\si{\degree}F$. Find the change in energy of the ice.
	\begin{align*}
	\Delta E_{int}&=M_{ice}C_{ice} \Delta T_{01}+M_{ice}L_{f_{ice}}+M_{water}C_{water} \Delta T_{12}\\
	&=M_{ice}(C_{ice}\Delta T_{01}+L_{f_{ice}}+C_{water} \Delta T_{12})
	\end{align*}


	\subsection{Molecular Model of Gasses}
	We will describe gasses using state variables. State variables are macroscopic quantities used to describe the state of matter. Relative state variables can be used together to make quantitative predictions. This can be done using equations of state. Pressure $P=\frac{F}{A}$, Volume V, temperature T, the number of gas particles N, or the number of moles $n=\frac{N}{N_A}$ where $N_A$ is Avogadros number ($6.022*10^{23}$)
	\subsection{Simplifying Assumptions of Gaseous Behavior}
	These are assumptions for an ideal gas that will make math easier:
	\begin{itemize}
	\item Distance between the gas molecules is much larger than the diameter of the molecules themselves, thus forces between molecules can be ignored.
	\item When gas molecules colide we will assume they are perfectly elastic collisions, thus $P$ can be easily derived.
	\item Due to the large distance between molecules, they are very compressible.
	\item The temperature of the gas is well above boiling point for given $P$ and V.
	\end{itemize}
	When a gas can be accurately described by these rules, the gas is considered an ideal gas. Most gasses experienced everyday are ideal gasses. The equation of state for an ideal gas is the ideal gas law:
	\begin{equation*}
	PV=NK_BT \to PV=n(N_AK_B)T \to PV=nRT
	\end{equation*}
	Boltons Constant $\to K_B=1.38*10^{-23}\frac{J}{K}$
	\newline Gas Constant $\to R=8.314\frac{J}{mol*K}$
	\subsection{Finding the Pressure of an Ideal Gas}
	Assuming a gas is within a rigid walled cube of length $l$, and when a gas molecule hits a wall it will change direction but conserve kinetic energy:
	\begin{align*}
	\Delta P \to \Delta P_x = -2mv_x
	\vec{F}=m\vec{a}
	\vec{F}=\frac{d}{dt}\vec{P} = \frac{\Delta \vec{P}}{\Delta t}
	\end{align*}
	Average time between colisions is $2L=v_x \Delta t$
	\begin{align*}
	\left|\frac{\Delta P_x}{\Delta t}\right| = \frac{2mv_x}{\frac{2L}{V_x}}=\frac{mv^2}{L}=F_x\\
	P_i=\frac{F}{A}=\frac{F_x}{L^2}=\frac{m_iv_{xi}^2}{L^3}
	\end{align*}
	Total Pressure: $P=\sum_{i}{P_i}=\frac{m}{L^3}(\sum_{i}V_x^2)$
	\begin{align*}
	\shortintertext{We know: } \vec{v}&=v_x\vec{i}+v_y\vec{j}+v_z\vec{k}\\
	v^2 &= v_x^2+v_y^2+v_z^2\\
	v^3 &= v_x^3+v_y^3+v_z^3
	\end{align*}
	Therefore we can determine that:
	\begin{align*}
	P &= \frac{m}{L^3}N(v_x^2)_{avg}\\
	P &= \frac{m}{3L^3}N(v^2)_{avg}\\
	PV &= \frac{m}{3}N(v^2)_{avg}
	\end{align*}
	Remember that for ideal gasses, $PV=NK_BT$, therefore  $K_BT=\frac{1}{3}m(v^2)_{avg}$
	\subsection{Root Mean Squared}
	Root mean squared is the average velocity for any given particle. It is defined by the equation $v_{rms}=\sqrt{v_{avg}^2}=\sqrt{\frac{3K_BT}{m}}$, where M is the mass of the object (particle) and T is the temperature of the particle.
	\subsection{Kinetic Average}
	\begin{equation*}
	K_{avg}=\frac{1}{2}m(v^2)_{avg}=\frac{3}{2}K_BT
	\end{equation*}
	\newline
	\begin{align*}
	E_{int}&=K_{tot}=\sum{i}K_i=NK_{avg_i}\\
	E_{int} &= \frac{3}{2}NK_BT\\
	\shortintertext{Or }E_{int}&=\frac{3}{2}nRT
	\end{align*}
	\subsection{Mean Free Path}
	The mean free path of a gas is the average distance traveled by a gas molecule between collisions. Considering a room of volume V, that is filled with gas with molecules with diameter d.
	\begin{align*}
	\lambda &= \frac{\text{Length of path in time}\Delta T}{\text{Number of collisions within}\Delta T}\\
	\lambda &= \frac{v \Delta T}{N\frac{V_{Cylinder}}{V}}\\
	\lambda &= \frac{V}{\sqrt{2}N\pi d^2}\\
	\text{Number Density } \eta &= \frac{N}{V}\\
	\lambda &=\frac{1}{\sqrt{2}\eta \pi d^2}\\
	PV &= nRT = NK_BT \text{, } P=\eta K_BT
	\end{align*}
	\subsubsection{Example 1}
	Find the RMS speed($v_{rms}$) of a $N_2$ molecule in a room at 20$\si{\degree}C$. DiNitrogen has a molecular mass of $14\frac{g}{mol}$, the R constant = $8.314\frac{J}{mol*K}$, M = molar mass and m = mass and T=293K.
	Recall that $M=mN_A$.
	\begin{align*}
	v_{rms}&=\sqrt{(v^2)_{avg}}\\
	&=\sqrt{\frac{3K_BT}{m}}\\
	&=\sqrt{\frac{3K_BT}{\frac{M}{N_A}}}\\
	&=\sqrt{\frac{3N_AK_BT}{M}}=\sqrt{\frac{3RT}{M}}\\
	v_{rms}&=\sqrt{\frac{3\left(8.314\frac{J}{mol*K}\right)(295K)}{0.028\frac{Kg}{mol}}}
	\end{align*}


	\subsection{Work}
	How much work will a gas do on its environment? Recall, $P=\frac{F}{A}$.
	\begin{align*}
	dW&=\vec{F}d\vec{x}\\
	&=Fdx-PAdx\\
	&=PdV
	\end{align*}
	\subsection{How The First law of Thermodynamics Relates to Work}
	\begin{align*}
	\Delta E_{int} = Q-W \to dE_{int} &=Q-PdV\\
	d\left(\frac{3}{2}NK_BT\right) &= Q-PdV\\
	\left(\frac{3}{2}NK_BT\right) &=Q-QdV
	\end{align*}
	\begin{itemize}
	\item Presure thermal equilibrium (quasi-static)
	\begin{align*}
		\int{dw}&=\int{PdV}\\
		W_{1\to2}&=\int_{v_1}^{v^2}{PdV}
	\end{align*}
	\item Isoconic case ($\Delta V = 0$)
	\begin{align*}
		w=0
	\end{align*}
	\item isoboric ($\Delta P=0$)
	\begin{align*}
		w_{1\to2}&=P\int_{v_1}^{v_2}{dV}\\
		&=P(V_2-V_1)
	\end{align*}
	\item isothermal ($\Delta T=0$)
	\begin{align*}
		w_{1\to2} &= \int_{V_1}^{V_2}{PdV}\\
		P=\frac{NK_BT}{V}\\
		w_{1\to2} &=\int_{V_1}^{V_2}{\frac{NK_BT}{V}dV}\\
		&=NK_BT\int_{V_1}^{V_2}\frac{dV}{V}\\
		w_{1\to2} &= NK_BTln \left( \frac{V_2}{V_1} \right)
	\end{align*}
	\end{itemize}
	\subsection{Pressure / Volume Diagrams}
	This is an easy way to interperate the changes due to temperature, pressure, and volume changes.
	\begin{itemize}
	\item Adiabatic $(Q=0)\to E_{int}=W$
	\item isothermal $(\Delta T = 0)$
	\begin{align*}
		\Delta E_{int} &= Q-W\\
		\frac{3}{2}NK_BdT&=Q-PdV \text{, } PdV=Q
	\end{align*}
	\item Isocloric $(\Delta V = 0) \to W=0 \to \Delta E_{int}=0$
	\item isoboric $(\Delta P = 0) \to W=P\Delta V \to \Delta E_{int}=Q-P\Delta V$
	\end{itemize}
	\subsubsection{Example 1}
	A sealed ideal gas is in a rigid container of $0.6m^3$ initially at room temperature $(T_1=20\si{\degree}C\to 293K)$ and pressure $(P_1=1atm\cong 10^5pa)$. If the temp ``doubles'' to $T_2=40\si{\degree}C\to313K$,
	\newline A.) what is the new pressure$(P_2)$?
	\begin{align*}
	\newline
	PV&=nRT\\
	\text{Or, }P_1V_1&=n_1RT_1 \text{, }P_2V_2=n_2RT_1\\
	\frac{P_1}{T_1}&=\frac{n_1R}{V_1} = const\\
	\frac{P_2}{T_2}&=\frac{n_1R}{V_2} = \frac{n_1R}{V_1}\\
	\text{therefore:}\\
	\frac{P_1}{T_1}&=\frac{P_2}{T_2}\\
	P_2&=P_1\frac{T_2}{T_1}\\
	&=(10^5pa)\frac{315K}{295K}=1.07*10^5pa
	\end{align*}
	B.) How much work did the gas do?
	\begin{align*}
	dW&=PdW\\
	w_{1\to2}&=\int_{V_1}^{V^2}{PdV}\\
	w_{1\to2}&=0
	\text{There was no change in volume}
	\end{align*}
	C.) How much heat was transfered (Not using $mc\Delta T$)?
	\begin{align*}
	\Delta E_{int}&=Q-W (W=\int{PdV}=0)\\
	\Delta E_{int}&=Q\\
	Q&=\frac{3}{2}nR(T_2-T_1)=\frac{3}{2}NK_B(T_2-T_1)\\
	&=\frac{3}{2}(P_2V_2-P_1V_1)\\
	Q&= \frac{3}{2}V_1(P_2,P_1)\\
	&=6,140J
	\end{align*}
	\subsubsection{Example 2}
	A sealed ideal gas undergoes an isobaric $(\Delta P=0)$ expansion during which it triples in volume. Then it is isothermically $(\Delta T=0)$ compressed to original volume, after which it cools to its original temperature. Find Q and W in terms of initial pressure and volume for each stage of cucle and the full cycle. Draw a PV diagram as well.
	\begin{align*}
	PV&=nRT\\
	P&=\frac{nRT}{V}\to T=\frac{PV}{nR}\\
	\Delta E_{int_{cyc}}&=0=Q_{cyc}-W{cyc}\\
	Q_{cyc}&=W_{cyc}\\
	\text{A.) Find the work from 1 to 2:}\\
	W_{1\to2}&=\int_{V_1}^{V_2}{PdV}\\
	&=\int_{V_1}^{3V_1}{dV}\\
	&=P_1(3V_1-V_1)\\
	w_{1\to2}&=2P_1V_1\\
	\text{B.) Find the work from 2 to 3:}\\
	W_{2\to3}&=\int_{v_2}^{v_3}PdV \to PV=nRT \to P=\frac{nRT}{V}\\
	&=\int_{V_2}^{V_1}{\frac{nRT}{V}dV}\\
	&=n_2RT_2\int_{V_2=3V}^{V_1}{\frac{dv}{V}}\\
	&=n_2RT_2ln\left(\frac{V_1}{3V_1}\right)=n_2RT_2ln\left(\frac{1}{3}\right)\\
	w_{2\to3}&=-n_2RT_2ln(3) \to PV=nRT\\
	&=-P_2V_2ln(3)\\
	w_{2\to3}&=-3ln(3)P_1V_1=-3.29P_1V_1\\
	\text{C.) Find the work from 3 to 1:}\\
	w_{3\to1} &= \int_{v_3}^{v_1}{PdV}=0\\
	w_{3\to1} &= 0\\
	\text{D.) Find the heat of the cycle}\\
	w_{cyc}&=w_{1\to2}+w_{2\to3}+w{2\to3}=P_1V_1(2-3ln3)\\
	&= -1.30P_1V_1\\
	\text{Since, } \Delta E_{int_cyc}=0, \text{ then } 0&=Q_{cyc}-W_{cyc}\\
	Q_{cyc}&=W_{cyc}=-1.30P_1V_1\\
	\text{And since }2\to3\text{ is an isotherm, we know } Q&=W\\
	Q_{2\to3}&=W_{2\to3}=-3ln(3)P_1V_1\\
	\text{For }1\to2 \text{: }\\
	\Delta E_int_{1\to2} &= Q_{1\to2}-W_{1\to2}\\
	Q_{1\to2}&=\frac{3}{2}nR(T_2-T_1)+2P_1V_1\\
	&=\frac{3}{2}n_2RT_2-\frac{3}{2}n_1RT_1+2P_1V_1\\
	&=\frac{3}{2}P_2V_2-\frac{3}{2}P_1V_1+2P_1V_1\\
	&=\frac{3}{2}P_13v_2-\frac{3}{2}P_1V_1+2P_1V_1\\
	&= \left(\frac{9}{2}-\frac{3}{2}+2\right)P_1V_1\\
	Q_{1\to2}&=5P_1V_1\\
	Q_{cyc}&=Q_{1\to2}+Q_{2\to3}+Q{3\to1}\\
	Q{3\to1}&=Q_{cyc}-Q_{1\to2}-Q_{2\to3}\\
	&=((2-3ln3)-5+3ln3)P_1V_1\\
	Q_{3\to1}&=-3P_1V_1
	\end{align*}


	\subsection{Using $C_V$ and $C_P$}
	While considering the first law, $dE_{int}=Q-PdV$, consider a constant volume: $dE_{int}=Q$, also: $E_{int}=\frac{3}{2}nRT$.
	\begin{align*}
	\frac{3}{2}nRdT&=Q\\
	nC_vdT &=Q\\
	\frac{3}{2}R=C_v \text{ if volume is constant}\\
	dE_{int}&=nC_vdT\\
	\end{align*}
	when $\Delta V=0$, then $Q=nC_V\Delta T$, but in all cases, $\Delta E_{int}= nC_V\Delta T$.
	\newline
	\begin{align*}
	\text{We have introduced } C_V&=\frac{3}{2}R\\
	\text{But recall: } (v)_{avg}^2&=V_{x_{avg}}^2+V_{y_{avg}}^2+V_{z_{avg}}^2\\
	\text{which gives us three degrees of translation, which is also why:}\\
	k_{avg}&=\frac{1}{2}m(v^2)_{avg}\\
	\text{this leads us to } E_{int}&=\frac{3}{2}nRT\\
	\end{align*}
	We can generalize $C_V=\frac{3}{2}R$ to be $C_V=\frac{d}{2}R$ for degrees if freedom. Other degrees of freedom come from both roation and vigration. Most monatomic atoms have 3 degrees of translation, and zero degrees of rotation and vibration. Diatomic atoms on the other hand have 3 degrees of translation, 2 degrees of rotation, and 1 degree of vibration.
	\newline
	\newline
	Consider the case when $(\Delta P=0)$. Let's preserve the form $Q=nC_PdT$
	\begin{align*}
	dE_{int}&=Q-Pdv\\
	\Delta E_{int}&=Q-P\Delta V\\
	nC_V\Delta T&=Q-P\Delta V\\
	nC_V\Delta T &= nC_P\Delta T-P\Delta V\\
	\text{Note that } PV=nRT\\
	dPV=d(nRT) \text{ therefore }P\Delta V=nR\Delta T\\
	nC_v\Delta T&=nC_P \Delta T-nR\Delta T\\
	C_V&=C_P-R\\
	C_P&=C_V+R\\
	\end{align*}
	Now for $(Q=0)$,
	\begin{align*}
	dE_{int}&=Q-PdV\\
	nC_VdT&=-PdV\\
	ndT&=-\frac{P}{C_V}dT\\
	\text{because }PV=nRT \text{, we can conclude } ndT&=\frac{V}{R}+\frac{P}{R}dV\\
	-\frac{P}{C_V}dT&=\frac{V}{R}dP+\frac{P}{R}dV\\
	\text{divide everything by PV: } -\frac{1}{C_V}\frac{D_V}{V} &=\frac{1}{R}\frac{dP}{P}+\frac{1}{R}\frac{dV}{V}\\
	\text{Multiply by }C_VR \text{ } -R\frac{dV}{V}&=C_V\frac{dP}{P}+C_V{dV}{V}\\
	0&=C_V\frac{dP}{P}+(C_V+R)\frac{dV}{V}\\
	&=C_V\frac{dP}{P}+C_P\frac{dV}{V}\\
	&=\int{\frac{dP}{P}}+\frac{C_P}{C_V}\int{\frac{dV}{V}}\\
	\text{const}&=ln\frac{P}{P_o}+\frac{C_P}{C_V}ln\frac{V}{V_o}\\
	PV^{\left(\frac{C_P}{C_V}\right)}&=\text{const}\\
	PV^{\gamma}&=\text{const for } \gamma=\frac{C_P}{C_V}\\
	\end{align*}
	\subsubsection{Example 1}
	2.2 moles of Ar gas are in a sealed metal container at room temp $(20\si{\degree}C=68\si{\degree}F)$. We then put the container on the sidewalk on a hot $(35\si{\degree}C=95\si{\degree}C)$ day. Let the gas come to temp $(95\si{\degree}F)$.
	\newline
	A.) Was any work done by or on the gas? There was little displacement, so almost 0 work done on the container.
	\newline
	B.) was any heat transferred to or from the gas? Heat was transfered to the gas.
	\newline
	C.) Find the gass' internal energy. We can always write:
	\newline
	\begin{align*}
	E_{int}&=\frac{3}{2}NK_BT=\frac{3}{2}nRT\\
	dE_{int}&=\frac{3}{2}K_B(TdV+NdT)\\
	dE_{int}&=\frac{3}{2}K_BdT\\
	\Delta E_{int}&=\frac{3}{2}NK_B\Delta T\\
	\Delta E_{int}&=nR\Delta T\\
	&=\frac{3}{2}(2.2mol)(8.314\frac{J}{molK})(35\si{\degree}C-20\si{\degree}C)\\
	&=411J\\
	\text{More generally } E_{int}=\frac{d}{2}nRT \text{Where d=degrees of translation}\\
	&=nC_VT \text{for } C_V=\frac{d}{2}R\\
	\end{align*}
	\newline
	Around room temperature for monotimic atoms, $C_V=\frac{3}{2}R$. For diatomic molecules it is $C_V=\frac{5}{2}R$.
	\subsubsection{Example 2}
	Solve the same problem where the only difference is the gas is a 2.2 moles of $H_2$. Given the temperatures, we know that $C_V=\frac{5}{2}R$.
	\begin{align*}
	\Delta E_{int}&=nC_V\Delta T\\
	&=n\frac{5}{2}R\Delta T\\
	&=\frac{5}{2}nR\Delta T\\
	&=\frac{5}{2}(2.2mol)(8.214\frac{J}{molK})(15K)\\
	&=687J
	\end{align*}
	\subsubsection{Example 3}
	Calculate work done by a gas during adiabatic expansion from $V_1$ to $V_2=4V_1$ in terms of $V_1$ and $V_2$. Remeber that adiabats result in no heat transfer $(Q=0)$ and $PV^\gamma{}=constant$, which means $\gamma{}=\frac{C_P}{C_V}=const$. Let $PV^\gamma{}=const$.
	\begin{align*}
	\text{Always: } w&=\int_{V_1}^{V^2}{PdV}\\
	&=\int_{V_1}{V_2}\frac{b}{V^\gamma}dv\\
	&=b\int_{V_1}^{V_2}{V^{-\gamma}}dv\\
	&=\frac{b}{-\gamma+1}V^{\gamma+1}|_{V_1}{V_2=4V_1}\\
	&=\frac{b}{1-\gamma}\left[(4V_1)^{1-\gamma}-V_1^{1-\gamma}\right]\\
	&=\frac{b}{1-\gamma}\left[4^{1-\gamma}V_1^{1-\gamma}-V_1^{1-\gamma}\right]\\
	&=\frac{b}{1-\gamma}V_1^{1-\gamma}\left[4^{1-\gamma}-1\right]\\
	&=\frac{4^{1-\gamma}-1}{1-\gamma}bV_1^{1-\gamma}\\
	&=\frac{4^{1-\gamma}-1}{1-\gamma}P_1V_1^{\gamma}V_1^{1-\gamma}\\
	W&=\frac{4^{1-\gamma}-1}{1-\gamma}P_1V_1
	\end{align*}
	Work done by a gas during an adiabatic expansion from $V_1to4V_1$. Let's find a more simplified answer for monatomic and diatomic gasses. The adiabatic ratio is $\gamma=\frac{C_P}{C_V}$, but it also means $C_P=C_V+R$. At room temperature, $C_V=\frac{3}{2}R$ for monotomic and $C_V=\frac{5}{2}R$.
	\begin{align*}
	\gamma_{mon}&=\frac{\frac{3}{2}R+R}{\frac{3}{2}R}=\frac{\frac{5}{2}R}{\frac{3}{2}R}=\frac{5}{3}\\
	\gamma_{dia}&=\frac{\frac{5}{2}R+R}{\frac{5}{2}R}=\frac{\frac{9}{2}R}{\frac{5}{2}R}=\frac{7}{5}\\
	\text{We found that } W&=\frac{4^{1-\gamma}-1}{1-\gamma}P_1V_1\\
	W_{dia}&=\frac{4^{1-\gamma}-1}{1-\gamma}P_1V_1 \text{ for } \gamma_{dia}=\frac{7}{5}\\
	w_{dia}&=1.06P_1V_1\\
	W_{mon}&=\frac{4^{\frac{3}{3}-\frac{5}{3}}-1}{\frac{3}{3}\frac{5}{3}}P_1V_1\\
	&=\frac{-4^{\frac{-2}{3}}-1}{\frac{-2}{3}}P_1V_1\\
	&=\frac{3}{2}\left(1-4^{\frac{-2}{3}}\right)P_1V_1\\
	&=.905P_1V_1\\
	\Delta{}E_{int}&=-1\\
	\end{align*}
	\subsection{Entropy}
	Reversible processes can have their processes reversed in time. When played inr evers, the behavior still looks physical (like a video). Any real system will have disipative forces and thus will not be perfectly reversible. Considering heat transfer, recording an ice cube melt and playing that video in reverse would look nonsensical. According to the second law of thermodynamics, heat always flows spontaneously from a hotter object to a colder object and vice versa. Irreversible process can happen within a closed system.
	\newline
	\newline
	For a reversible system, entropy is given by $dS=\frac{dQ}{T}$. For isothermal processes:
	\begin{align*}
	\int{dS}&=\int{\frac{dQ}{T}}\\
	\int{dS}&=\frac{1}{T}\int{dQ}\\
	\Delta S &= \frac{Q}{T}
	\end{align*}
	Imagine an ice cube melting at $0\si{\degree}C$. Then $Q=mL_f$,
	\begin{equation*}
	\Delta S =\frac{mLf}{T}
	\end{equation*}
	In general for a reversible process, we know:
	\begin{align*}
	dE_{int}&=dQ-dW\\
	dQ&=dE_{int}+dW\\
	\text{then entropy is: } ds&=\frac{dQ}{T}=\frac{dE_{int}+dW}{T}\\
	&=\frac{nC_VdT+PdV}{T}=nC_V\frac{dT}{T}+P\frac{dV}{T}\\
	\text{from the ideal gas law, } PV&=nRT \text{ or } \frac{P}{T}=\frac{nR}{V}\\
	\text{then, } \int{dS}&=\int{nC_V\frac{dT}{T}}+\int{\frac{nRdV}{V}}\\
	\Delta S_{0\to1}&=nC_Vln\frac{T_1}{T_2}+nRln\frac{V_1}{V_2}\\
	\text{For a reversible complete cycle:}\\
	\oint dS&=0
	\end{align*}	
\newpage
\section{Electricity}
  \subsection{Electrostatics}
  Electric charge is a property of some objects that allow such objects to feel an electric force. The units for charge are coulombs (C). The smallest amount of charge an object can have is the elementary charge: $e=1.602*10^{-19}C$. For an electron, this is -e, while for a proton it is +e. As everyone knows, like charges repel, and opposite charges attract each other.\par 
  Any material we use is made of atoms, which themselves are made of charge. There are two different types of materials, insulators and conductors. For insulators, atomic/molecular electrons are stuck in place with their parent atom/molecule and their electrons are immobile. For conductors, the atomic/molecular electrons are shared among the material (these are also known as mobile electrons). Our typical insulators are wood, plastic, and printer paper, while typical conductors are metal, water, and humans.\par
  \begin{align*}
    \text{The force that charge } q_1 \text{ exerts on charge} q_2 \text{ is: (Coulombs law)}\\
    \vec{F_{1\to2}}(r_{1\to2})=\frac{1}{4\pi \epsilon_0}\frac{q_1q_2}{r_{1\to2}^2}\hat{r_{1\to2}}\\
    \text{Where } \epsilon_0=8.85*10^{-12}\frac{C^2}{Nm^2}
  \end{align*}
  \subsubsection{Example}
  Find the net force on $q_c$ due to $q_a$ and $q_b$ where $q_a+q_b$ are fixed in space. Remeber that forces are vectors.
  \begin{align*}
    \vec{F_{net_c}}&=\vec{F_{bc}}+\vec{F_{ac}}\\
    &=\frac{1}{4\pi\epsilon_0}\frac{q_bq_c}{r_{bc}^2}\hat{r_{bc}}+\frac{1}{4\pi\epsilon_0}\frac{q_bq_c}{r_{ac}^2}\hat{r_{ac}}\\
    &=\frac{1}{4\pi\epsilon_0}\frac{q_bq_c}{X_0^2}(\hat{+i})+\frac{1}{4\pi\epsilon_0}\frac{q_aq_c}{Y_0^2}(\hat{+j})\\
    &=\frac{1}{4\pi\epsilon_0}\left(\frac{q_bq_c}{x_0^2}\hat{i}+\frac{q_bq_c}{y_0^2}\hat{j}\right)\\
    \text{if } q_b=q_a \text{ and } x_0=y_0 \text{, then }\\
    \vec{F_{net_c}}&=\frac{q_aq_c}{4\pi\epsilon_0x_0^2}(\hat{i}+\hat{j})\\
  \end{align*}\
  Coulombs law works with super position:
  \begin{equation*}
    \vec{F_{1\to2}}=\frac{1}{4\pi\epsilon_0}\frac{q_1q_2}{r_{1\to2}^2}\hat{r_{1\to2}}
  \end{equation*}
  Find the force on charge $Q_A$ due to other stationary charges:
  \begin{align*}
    \vec{F_{ba}}&=\frac{1}{4\pi\epsilon_0}\frac{q_aq_b}{r_{ba}}\hat{r_{ba}}\\
    &=\frac{1}{4\pi\epsilon_0}\frac{q_aq_b}{y_0^2}\hat{j}\\
    \vec{F_{ca}}&=\frac{1}{4\pi\epsilon_0}\frac{q_aq_c}{r_{ca}^2}\hat{r_{ca}}\\
    &=\frac{q_cq_a}{4\pi\epsilon_0}\left[\frac{1}{r_{ca}^2}sin(\theta)(\hat{i})+\frac{1}{r_{ca}^2}cos(\theta)(-\hat{j})\right]\\
    &=\frac{q_cq_a}{r\pi\epsilon_0}\left[\frac{1}{x_0^2+y_0^2}\frac{x_0}{\sqrt{x_0^2+y_0^2}}\hat{i}-\frac{1}{\sqrt{x_0^2+y_0^2}}\frac{y_0}{\sqrt{x_0^2+y_0^2}}\hat{j}\right]\\
    &=\frac{q_cq_a}{4\pi\epsilon_0}\frac{1}{(x_0^2+y_0^2)^{\frac{3}{2}}}(x_0\hat{i}-y_0\hat{j})\\
    \text{Now find }f_{net_a}&=F_{ca}+F_{ba}\\
    &=(F_{ba_x}+F_{ca_x})\hat{i}+(F_{ba_y}+F_{ca_y})\hat{j}\\
    F_{net_a}&=\frac{1}{4\pi\epsilon_0}\left[\frac{q_cq_a}{(x_0^2+y_0^2)^{\frac{3}{2}}}x_0\hat{i}+\left(\frac{q_bq_a}{y_0}-\frac{q_cq_a}{(x_0^2+y_0^2)^{\frac{3}{2}}}y_o\right)\hat{j}\right]\\
  \end{align*} 
  \subsection{Electric Fields}
  A field is a function that has different values throughout space that can be changed in time. Temperature is a scalar field when looking at the temperature around a room. Wind patterns are a vector field on a weather map. We will study the electric field $\vec{E}$. This field helps us find the force exerted on any charge Q.
  \begin{equation*}
    \vec{F}=Q\vec{E}
  \end{equation*}
  The field $\vec{E}$ created by a charge q is:
  \begin{equation*}
    \vec{E}=\frac{1}{4\pi\epsilon_0}\frac{q}{r^2}\hat{r}\\
  \end{equation*}
  This field acts on another charge Q such that:
  \begin{align*}
    \vec{F}&=Q\vec{E}\\
    &=Q\frac{1}{4\pi\epsilon_0}\frac{q}{r^2}\hat{r}\\
    \vec{F}&=\frac{1}{4\pi\epsilon_0}\frac{qQ}{r^2}\hat{r} \text{ (Coulombs Law)}\\
  \end{align*}
  $\vec{E}$ Fields point away from positive cherges and towards negative ones. Since $\vec{E}$ is a vector, if two or more static charges each create a field, the net field is their vector sum (superposition).
  \begin{equation*}
    \vec{E_{net}}=\sum_{i=1}^n\vec{E_i}
  \end{equation*}
  \subsubsection{Example 1}
  Two identical charges $q_1=q_2=q$ are equidistant from the origin on the x-axis. Find $\vec{E}$ anywhere on the x-axis.
  \begin{align*}
    \vec{E}&=2\left(\frac{1}{4\pi\epsilon_0}\frac{q_2}{r_2^2}cos\theta\right)\hat{k}\\
    &=2\frac{1}{4\pi\epsilon_0}\frac{2qz}{\left(z^2+\left(\frac{L}{2}\right)^2\right)^{\frac{3}{2}}}\hat{k}\\
    \vec{E}(z)&=\frac{1}{4\pi\epsilon_0}\frac{2qz}{\left(z^2+\left(\frac{L}{2}\right)^{\frac{3}{2}}\right)}\hat{K}
  \end{align*}
  What if the charges that make a field are grouped together closely? Then we will describe those charges as a continuous distribution. Linear charge density is $\lambda=\frac{dq}{dl}\to dq=\lambda dl$. Recall,
  \begin{align*}
    \vec{E}&=\frac{1}{4\pi\epsilon_0}\frac{q}{r^2}\hat{r}\\
    &=\sum_{i=1}^N\frac{1}{4\pi\epsilon_0}\frac{q_i}{r_i^2}\hat{r_i}\to \vec{E}=\frac{1}{4\pi\epsilon_0}\int\frac{dq}{r^2}\hat{r}\\
    \text{For this problem: }\\
    E&=\frac{1}{r\pi\epsilon_0}\int_{\frac{-L}{2}}^{\frac{L}{2}}\frac{\lambda dx}{z^2+x^2}\frac{z}{\sqrt{z^2+x^2}}\hat{k}\\
    &=\frac{1}{r\pi\epsilon_0}\lambda z \int_{\frac{-L}{2}}^{\frac{L}{2}}\frac{dx}{(z^2+x^2)^{\frac{3}{2}}}\hat{k}\\
    &=\frac{1}{4\pi\epsilon_0}\lambda z \left(\frac{x}{z^2\sqrt{x^2+z^2}}\right)\Big|_{\frac{-L}{2}}^{\frac{L}{2}}\hat{k}\\
    &=\frac{1}{4\pi\epsilon_0}\lambda z \left[\frac{\frac{L}{2}-\frac{-L}{2}}{z^2\sqrt{\frac{L}{2}^2+z^2}}\right]\hat{k}\\
    &=\frac{1}{4\pi\epsilon_0}2\lambda z\frac{\frac{L}{2}}{z^2\sqrt{\frac{L}{2}^2+z^2}}\hat{k}
  \end{align*}
  If the line of charge is $\inf$ in length, then the only change would be the bounds of the integral.
  \subsubsection{Example 2}
  Find $\vec{E}$ where the midpoint of a uniform line of charge as shown. The X components cancel.
  \begin{align*}
    d\vec{E}&=\frac{1}{4\pi\epsilon_0}\frac{dq}{r^2}\hat{r}\\
    \lambda&=\frac{dq}{dl}=\frac{dq}{dx}\\
    dq&=\lambda dx\\
    \text{We see that } E_x=0 \text{ while } E_z\neq 0 \text{ then, }\\
    d\vec{E}&=\frac{1}{4\pi\epsilon_0}\frac{\lambda dx}{x^2+z^2}\frac{z}{\sqrt{x^2+z^2}}\hat{k}\\
    &=\frac{1}{4\pi\epsilon_0}\frac{\lambda dxz}{(x^2+z^2)^{\frac{3}{2}}}\hat{k}\\
    E&=\int_{\frac{-L}{2}}^{\frac{L}{2}}\frac{1}{4\pi\epsilon_0}\frac{\lambda dxz}{(x^2+z^2)^{\frac{3}{2}}}\\
    &=\frac{\lambda z}{4\pi\epsilon_0}\int_{\frac{-L}{2}}^{\frac{L}{2}}(x^2+z^2)^{\frac{-3}{2}}dx\\
    \vec{E}(z)&=\frac{2\lambda z}{4\pi\epsilon_0}\frac{\frac{L}{2}}{z^2\sqrt{\frac{L}{2}^2+z^2}}\hat{k}
  \end{align*}
  \subsubsection{Example 3}
  Find $\vec{E}$ for the sitution shown:
  \begin{align*}
    d\vec{E}&=\frac{1}{4\pi\epsilon_0}\frac{dq}{r^2}\hat{r}\\
    &=\frac{1}{4\pi\epsilon_0}\frac{\lambda dx}{r^2}\left(cos\theta\hat{i}+sin\theta\hat{k}\right)\\
    &=\frac{\lambda}{4\pi\epsilon_0}\left[\int_{0}^{L}\frac{dx(x-L)}{(z^2+(x-L))^{\frac{3}{2}}}\hat{i}+\int_0^L\frac{dxz}{(z^2+(x-L))^{\frac{3}{2}}}  \right]\\
    \shortintertext{We will let x'=x-l } x=0\to x'=0-L=-L\\
    x=L\to x'=0\\
    dx'=dx-dL \text{ because } L=const\\
    &=\frac{\lambda}{4\pi\epsilon_0}\left[\int_{-L}^0\frac{dx'x'}{(z^2+x'^2)^{\frac{3}{2}}}\hat{i}+\int_{-L}^0\frac{dx'z}{(z^2+x'^2)^{\frac{3}{2}}}\hat{k}\right]\\
    \vec{E}&=\frac{\lambda}{4\pi\epsilon_0}\left[\frac{x'}{x'^2\sqrt{x^2+z^2}}\Big|_{x'=L}^{x'=0}\hat{i}+\frac{-1}{\sqrt{x^2+z^2}}\Big|_{x'=-L}^{x'=0}\hat{k}\right]
  \end{align*}
  \subsection{Electric Field Lines}
  We can draw field lines to represent how an electric field looks in space. Recall that for a point charge, $\vec{E}=\frac{1}{4\pi\epsilon_0}\frac{q}{r^2}\hat{r}$
  \newline
  Let's find $\vec{E}$ above the midpoint of two opposite charges +q and -q which are a distance d apart.
  \begin{align*}
    r^2&=z^2+\left(\frac{d}{2}\right)^2\\
    \vec{E}&=\vec{E}_-\vec{E}_+\\
    &=\frac{1}{4\pi\epsilon_0}\frac{q}{z^2+\frac{d}{2}}\left(-\frac{\frac{d}{2}}{\sqrt{z^2+\frac{d}{2}^2}}\hat{i}-\frac{z}{\sqrt{z^2+\frac{d}{2}^2}}\hat{k}\right)+\frac{1}{4\pi\epsilon_0}\frac{q}{z^2+\frac{d}{2}}\left(-\frac{\frac{d}{2}}{z^2}\hat{i}+\frac{z}{\sqrt{z^2+\frac{d}{2}^2}}\hat{k}\right)\\
    \vec{E}&=-\frac{1}{4\pi\epsilon_0}\frac{qd}{\left(z^2+\frac{d}{2}^2\right)^{\frac{3}{2}}}\hat{i}\\
    \shortintertext{Consider $\vec{p}=q\vec{d}$. This is an electric dipole moment.}\\
    \vec{E}&=-\frac{1}{r\pi\epsilon_0}\frac{\vec{p}}{\left(z^2+\frac{d}{2}^2\right)^{\frac{3}{2}}}\\ 
    \shortintertext{As you get further from the dipole, z$>>$d:}\\ 
    \vec{E}&\cong-\frac{1}{4\pi\epsilon_0}\frac{\vec{p}}{(z^2)^{\frac{3}{2}}}\\
    \shortintertext{If you get very far away: $z\to\infty$ or $\frac{d}{2}\to\infty$}\\
    \vec{E}&=\frac{-1}{4\pi\epsilon_0}\frac{qd}{\left(z^2+\frac{d}{2}\right)^{\frac{3}{2}}}\hat{i} =\frac{-1}{4\pi\epsilon_0}\frac{q\left(\frac{d}{z}\right)}{\left(1+\frac{d}{2}\right)^{\frac{3}{2}}}\hat{i}\\
  \end{align*}
  For dipoles in an external field, $\vec{p}=q\vec{d}$. The net displacement force on $\vec{p}$ is 0, but it will have torque.
  \begin{align*}
    \tau=\vec{r}\times\vec{F}=\left(\vec{\frac{d}{2}}\times\vec{F_+}\right)+\left(\vec{\frac{d}{2}}\times\vec{F_-}\right)&=\left(\vec{\frac{d}{2}}\times q\vec{E_{ext}}\right)+\left(-\vec{\frac{d}{2}}\times q\vec{E_{ext}}\right)\\
    &=2\left(\vec{\frac{d}{2}}\times q\vec{E_{ext}}\right)\\
    \vec{\tau}&=(\vec{q}d\times \vec{E_{ext}})\\
    \alignedbox{\vec{\tau}}{=\vec{P}\times\vec{E_{ext}}}\\
  \end{align*}
  \subsubsection{Example 1.}
  The field $\vec{E}$ is above the center of a uniformly charged ring is $\vec{E}=\frac{1}{r\pi\epsilon_0}\frac{Qz}{(z^2+k^2)^{\frac{3}{2}}}\hat{k}$. Now let the ring isntead, be a disk with toal charge Q that is uniformly distributed.
  \begin{align*}
    \shortintertext{Total charge Q over a total area of $\pi R^2$}\\
    \frac{Q}{\pi R^2}&=\sigma\\
    dQ&=\sigma2\pi rdr\\
    dE&=\frac{1}{4\pi\epsilon_0}\frac{dQ}{r^2}\hat{r}\\
    dQ&=\sigma 2\pi r'dr'\\
    \shortintertext{All of the x and y components vanish, which leaves us with:}\\
    \int dE&=\int\frac{1}{4\pi\epsilon_0}\frac{\sigma 2\pi r'dr'}{r'^2+z^2}\frac{z}{\sqrt{r'^2+z^2}}\hat{k}\\
    \vec{E}&=\frac{1}{4\pi\epsilon_0}2\pi\sigma z \int_0^R\frac{r'dr'}{(r'^2+z^2)^{\frac{3}{2}}}\hat{k}\\
    &=\frac{1}{4\pi\epsilon_0}2\pi\sigma z \left(-\frac{1}{\sqrt{r'^2+z^2}}\right)\Big|_{r'=0}^{r'=R}\\
    &=\frac{1}{4\pi\epsilon_0}2\pi\sigma z \left[-\frac{1}{R^2+z^2}-\frac{1}{z}\right]\hat{k}\\
    \alignedbox{\vec{E}}{=\frac{1}{4\pi\epsilon_0}\frac{2Q}{R^2}z\left(\frac{1}{z}-\frac{1}{\sqrt{R^2+z^2}}\right)\hat{k}}
  \end{align*}
  \subsection{Flux and Gauss's Law}
  Flux is the amount of flow of a material or substance through some region. Area vectors point out of the plane of the area itself. Flux is denoted by $\Phi$, and the equation for flux is:
  \begin{equation*}
    \Phi=\int\vec{f}\cdot d\vec{a}
  \end{equation*}
  \subsubsection{Example 1}
  Find the flux $\Phi$ of $\vec{E}=\epsilon_0\frac{z}{z_0}\vec{i}$ through an are $y_0+z_0$ in the y-z plane.
  \begin{align*}
    \Phi&=\int\vec{E}\cdot d\vec{a}\\
    &=\int Eda\\ 
    &=\int_0^{z_0}\int_0^{y_0}Eda\\
    &=\int_0^{z_0}\int_0^{y_0}\epsilon_0\frac{z}{z_0}dydz\\
    &=\epsilon_0\frac{y_0}{z_0}\int_0^{z_0}dz\\
    &=\epsilon_0\frac{y_0}{z_0}\left(\frac{1}{2}z_0^2-\frac{1}{2}0^2\right)\\
    \alignedbox{\Phi}{=\frac{1}{2}\epsilon_0y_0z_0}\\
    \shortintertext{First, a single charge Q. We must apply a Gaussian surface to the point charge:}\\
    \Phi&=\int\vec{E}\cdot d\vec{a}=\int(E\hat{r})\cdot(da\hat{r})\\
    &=\int\vec{E}\cdot d\vec{a}\\
    &=\int\frac{1}{4\pi\epsilon_0}\frac{1}{R^2}da\\
    &=\frac{1}{4\pi\epsilon_0}\int da\\ 
    &=\frac{q}{\epsilon_0}\\
    \intertext{More generally, we write:}\\
    \alignedbox{\Phi_E}{=\oint\vec{E}\cdot d\vec{a}=\frac{q}{\epsilon_0}}
  \end{align*}
  This brings us to Gauss's law. This law is always true but it is only useful in certain situations. 1) When $\vec{E}\cdot d\vec{a}$ is an easy dot product (they are parallel vectors). 2) When E is constant in magnatude across the surface. Then we can pull E through the integral. 3) When the total surface area is known. For example, we want the equations to be able to do the following:
  \begin{align*}
    &\oint\vec{E}\cdot d\vec{a}\\
    &\oint Eda\\
    &E\oint da /to E_a=\frac{q_{inside}}{E_0}
  \end{align*}
  \subsubsection{Example 2}
  A sphere has a uniform charge Q throughout its volume and radius, R. Find E everywhere.
  \begin{align*}
    \shortintertext{Outside:}\\
    \oint\vec{E}\cdot d\vec{a}&=\frac{q_{enc}}{\epsilon_0}\\
    \oint Eda &= \frac{Q}{\epsilon_0}\\
    E\oint da &= \frac{Q}{\epsilon_0}\\
    E4\pi r^2 &=\frac{Q}{\epsilon_0}\\
    \alignedbox{\vec{E}}{=\frac{1}{4\pi\epsilon_0}\frac{Q}{r^2}\hat{r}\text{ for }r>R}
    \shortintertext{Now for the inside:}\\
    \int dq_{enc}&=\int\rho dv\\
    q_{enc}&=\rho\int dv_{gauss}\\
    &=\rho\frac{4}{3}\pi r^3\\
    \text{Also } \rho&=\frac{Q}{\frac{4}{3}\pi R^3}\\
    E4\pi r^2&=\frac{q_{enc}}{\epsilon_0}\\
    E&=\frac{1}{4\pi r^2}\frac{1}{\epsilon_0}Q\frac{r^3}{R^3}\\
    \alignedbox{E}{=\frac{1}{4\pi\epsilon_0}\frac{Qr}{R^3}\hat{r} \text{ for }r<R}
  \end{align*}
  \subsubsection{Example 3}
  Find $\vec{E}$ everywhere for a very long line of charge with a charge density $\lambda$ (constant and uniform charge).
  \begin{align*}
    \oint\vec{E}\cdot d\vec{a}&=\frac{q_{enc}}{\epsilon_0}\\
    \int\vec{E}\cdot d\vec{a}_{curve}+\int\vec{E}\cdot d\vec{a}_{left} +\int\vec{E}\cdot d\vec{a}_{right}&=\int\vec{E}\cdot d\vec{a}_{curve}=\frac{\lambda l}{\epsilon_0}\\
    \int\vec{E}\cdot d\vec{a}_{curve}&=\frac{\lambda l}{\epsilon_0}\\
    E\int d\vec{a}_{curve}&=\frac{\lambda L}{\epsilon_0}\\
    E(2\pi rl)&=\frac{\lambda l}{\epsilon_0}\\
    \alignedbox{\vec{E}}{=\frac{\lambda}{2\pi\epsilon_0r}\hat{r}}\\
  \end{align*}
  \subsubsection{Example 4}
  A sphere has a non-uniform charge density of $\rho=\rho_0\frac{r}{R}$ for a sphere with a radius, R. Find $\vec{E}$ everywhere.
  \begin{align*}
    \int\int\int dxdydz \to \int\int\int r^2sin\theta d\theta dr d\phi\\
    \shortintertext{First, the outside: } r>R\\
    \oint\vec{E}\cdot d\vec{a}&=\frac{q_{enc}}{\epsilon_0}\\
    \oint Eda&=\frac{q_{enc}}{\epsilon_0}\\
    E\oint da&=\frac{q_{enc}}{\epsilon_0}\\
    E(4\pi r^2)=\frac{q_{enc}}{\epsilon_0}\\
    \int dq_{enc} &= \int\rho dV_{enc}\\
    q_{enc}&=\int \rho_0\frac{r}{R}dV\\
    &=\frac{\rho_0}{R}\int_0^{2\pi}\int_0^\pi\int_0^Rrr^2sin\theta dr d\theta d\phi\\
    &=\frac{\rho_0}{R}4\pi\int_0^Rr^3dr\\
    &=\rho_0\frac{1}{R}4\pi\frac{1}{4}R^4\\
    q_{enc}&=\pi\rho_0R^3\\
    E(4\pi r^2)=\frac{1}{\epsilon_0}\pi\rho_0R^3\\
    \alignedbox{\vec{E}}{=\frac{1}{4\epsilon_0}\frac{\rho_0R^3}{r^2}\hat{r}\text{ for }r>R}\\
    \shortintertext{What about for $r<R$? Becuase of the boundary, we will have to seperately find $\vec{E}$ for $r<R$:}\\
    \oint\vec{E}\cdot d\vec{a}&=\frac{q_{enc}}{\epsilon_0}\\
    \oint Eda &=\frac{q_enc}{\epsilon_0}\\
    E\oint da&=\frac{q_{enc}}{\epsilon_0}\\
    E(4\pi r^2)&=\frac{q_{enc}}{\epsilon_0}\\
    \int dq_{enc}&=\int \rho dV_{gauss}\\
    q_{enc}&=\int\rho_0\frac{r}{R}dV_{gauss}=\frac{\rho_0}{R}\int_0^{2\pi}\int_0^\pi\int_0^rrr^2sin\theta dr d\theta d\phi\\
    &=4\pi\frac{\rho_0}{R}\int_0^rr^3dr\\
    q_{enc}&=\pi\frac{\rho}{R}r^4\\
    \to \text{Gauss's law } E(4\pi r^2)&=\frac{1}{\epsilon_0}\pi\rho_0\frac{r^4}{R}\\
    E&=\frac{1}{4\epsilon_0R}\rho_0r^2\\
    \alignedbox{E}{=\frac{1}{4\epsilon_0}\rho_0\frac{r^2}{R}}\\
  \end{align*}
  The equations agree at $r=R$, therefore when greather than or equal to and less than and equal to, the answer also works as $r\to0$ or $r\to\infty$.
  \subsubsection{Example 4 (Gauss's Law with Conductors)}:
  Let's consider 2 large conductivity plates that are side by side. Lets put $\pm Q$ on both of the conductivity plates.
  \begin{align*}
    \oint\vec{E}\cdot d\vec{a}&=\frac{q_{enc}}{\epsilon_0}\\
    \int\vec{E}d\vec{a}_{right}+\int\vec{E}d\vec{a}_{left}+\int\vec{E}d\vec{a}_{top}&+\int\vec{E}d\vec{a}_{bottom}+\int\vec{E}d\vec{a}_{top}+\int\vec{E}d\vec{a}_front\\
    \int\vec{E}d\vec{a}_{right}+\int\vec{E}d\vec{a}_{left}&=\frac{q_{enc}}{\epsilon_0}\\
    E_a+E_a&=\frac{q_{enc}}{\epsilon_0} \text{ Surface charge density}\\
    2E_a&=\frac{\sigma a}{\epsilon_0}\hat{i} \text{ Between plates}\\
    \vec{E}&=\frac{\sigma}{2\epsilon_0}\hat{i} \text{ left of the plates}\\
    \shortintertext{Now for the right plate:}\\
  \end{align*}



  % Notes from 10.05.2020
  \newpage
  \subsection{Electric Potential}
  \subsubsection{Example 1}
  Find $\vec{E}$ above a disk of charge distribution that is uniform. the disk has radius (R) and total charge (Q). Let's find V then $\vec{E}$
  \begin{align*}
    dV&=\frac{1}{4\pi\epsilon_0}\frac{dq}{r}\\
    \shortintertext{The charge distribution is $\sigma=\frac{Q}{\pi R^2}$}\\
    dq&=\sigma dA\\
    &=\sigma 2\pi rdr\\
    \shortintertext{This r is not the same r as before, so we will call them r'}\\
    &=\sigma 2\pi r'dr'\\
    dV&=\frac{1}{4\pi\epsilon_0}\frac{\sigma 2\pi r'dr'}{\sqrt{z^2+r'^2}}\\
    V&=\frac{1}{4\pi\epsilon_0}2\pi\sigma\int_0^R\frac{r'dr'}{\sqrt{z^2+r'^2}}\\
    V&=\frac{1}{4\pi\epsilon_0}2\pi\sigma\sqrt{z^2+r'^2}\Big|_{r'=0}^{r'=R}\\
    \alignedbox{V}{=\frac{1}{4\pi\epsilon_0}2\pi\sigma\left[\sqrt{z^2+R^2}-z\right]}
    \shortintertext{Now let's find $\vec{E}$}\\
    E_x&=-\frac{\partial}{\partial x}V=0\\
    E_y&=\frac{-\partial}{\partial y}V=0\\
    E_z&=\frac{-\partial}{\partial z}V = -\frac{1}{4\pi\epsilon_0}2\pi\sigma\left[\frac{1}{2}(z^2+R^2)^{\frac{-1}{2}}2z-1\right]\\
    \alignedbox{\vec{E}}{=\frac{1}{4\pi\epsilon_0}2\pi\sigma\left[1-\frac{z}{\sqrt{z^2+R^2}}\hat{K}\right]}\\
  \end{align*}
  What does it mean to have potential charge? Potential doesn't have a proper value. Defining a reference point for V. $E_s = -\frac{\partial}{\partial s}V$. Consider V and $V+V_0$ where $V_0$ is a constant. We get the same $\vec{E}$ value. $E_s=-\frac{\partial}{\partial s}V=\frac{-\partial}{\partial s}(V+v_0)$. This is analogous to choosing an origin. For example, if you are going to calculate the velocity of a marker hitting the ground, you have to keep track of position so you must choose an origin. This is the same way that V is chosen when doing these equations. We need to choose where $V=0$. We choose $V=0$ at infinity. Recall that $V=\frac{1}{4\pi\epsilon_0}\frac{q}{r}$.
  \subsection{Electrostatic Potential Energy}
   Electric forces are conservative. This doesn't mean not progressive, it means that they conserve energy. We can do these calculations using energy alone, similar to gravity or $MgH=\frac{1}{2}mv^2$. The law of energy conservation is $\Delta K+\Delta U=0$. Energy is not always conserved though. One physics 103 example is friction (drag). Disipative forces are also a good example of ways energy is not fully conserved. Energy is not lost, it is just converted from translational energy to heat energy. Macroscopically, the energy is not conserved, but at a microscopic level, the energy is fully conserved. We will not consider disipative forces in this course. This means that we can use the law of energy conservation ($\Delta K+\Delta U=0$). We've said previously that the $U=QV$. Very similar to $\vec{F}=Q\vec{E}$.
  \subsubsection{Example 1}
  Charge $Q_1=6\mu C$ and $Q_2=4\mu C$ are released from rest at a distance apart of $l=10cm$. Find their final speeds $v_1$ and $v_2$. The forces of point $Q_1$ and $Q_2$ are going to exert forces away from each other. It is possible to take a force approach to this problem, but as they move apart, their acceleration changes. This makes the problem much more difficult, but you can do an energy approach because the stages in between do not matter within this approach. We will solve this using energy.
  \begin{align*}
    \Delta K+\Delta U &=0\\
    K_1-K_0+U_1-U_2&=0\\
    \shortintertext{We have to change either $Q_1$ and $Q_2$ or change $K_0$ and $K_1$ because they don't mean the same thing. We are going to change Q to be $Q_A$ and $Q_B$. Initially, their kinetic energy is zero:}
    K_1+U_1-U_0&=0\\
    \left(\frac{1}{2}m_Av_a^2+\frac{1}{2}m_bv_b^2\right)+\frac{1}{4\pi\epsilon_0}\frac{Q_AQ_B}{r_{AB}}-\frac{1}{4\pi\epsilon_0}\frac{Q_A}{l}&=0\\
    \shortintertext{Interaction energy is a better term than potential energy. Potential energy goes in pairs. It is how two charges interact, not how a single charge exists. No matter how far apart they are, they will exert a force on each other. $\frac{Q_A}{Q_B}\to0$ because $r_{AB}\to 0$}\\
    \frac{1}{2}m_Av_A^2+\frac{1}{2}m_Bv_B^2&=\frac{1}{4\pi\epsilon_0}\frac{Q_AQ_B}{l}\\
    \shortintertext{Let $M_A=M_B=15g$}\\
    \shortintertext{In this system, $\vec{F_{AB}}=-\vec{F_{AB}}$ Also there are no external forces acting so,}\\
    \vec{F}_{net}&=\vec{F}_{AB}+\vec{F}_{BA}=0=\frac{d}{dt}\vec{P}\\
    \Delta P_{01}&=0=P_1-P_0\\
    0&=m_Av_A+m_Bv_B\\
    \to v_A^2&=V_B^2\frac{m_B^2}{m_A^2}\\
    \to \frac{1}{2}m_A\left[v_B^2\frac{m_B^2}{m_A^2}\right]+\frac{1}{2}m_Bv_B^2&=\frac{1}{4\pi\epsilon_0}\frac{Q_AQ_B}{l}\\
    \frac{1}{2}v_B^2\left[\frac{m_B^2}{m_A}+m_B\right]&=\frac{1}{4\pi\epsilon_0}\frac{Q_AQ_B}{l}\\
    m_bv_B^2&=\frac{1}{4\pi\epsilon_0}\frac{Q_AQ_B}{l}\\
    \alignedbox{v_b}{=\sqrt{\frac{1}{m_b}\frac{1}{4\pi\epsilon_0}\frac{Q_AQ_B}{l}}}\\
    v_a&=-v_B\frac{m_B}{m_A}\\
    \alignedbox{v_A}{=-v_B}\\
  \end{align*}
  \newpage
\section{Circuits}
  \subsection{Capacitants}
  Previously we found that $E=\frac{\sigma}{2\epsilon_0}+\frac{\sigma}{2\epsilon_0}=\frac{\sigma}{\epsilon_0}$ where $\sigma = \frac{Q}{A}$. Let's find potential $V$ between the plates. Recall that $v=-\int\vec{E}\cdot d\vec{s}$.
  \begin{align*}
    V&=-\int_0^d\vec{E}\cdot d\vec{s}\\
    \text{for } \vec{E}&=\frac{\sigma}{\epsilon_0}\\
    V&=-\int_0^d Eds=-\int_0^d Edx\\
    &=-E\int_0^d dx\\
    V&=-Ed \text{ This is from the pos to negative plate}\\
    \shortintertext{Equivilently from - to + plate:}\\
    V&=-int_0^d \vec{E}\cdot d\vec{s}\\
    &=-int_0^d(-Eds)=\int_0^d Edx\\
    V&=Ed \text{ This is from neg to pos plate}
    \shortintertext{Now we want to relate our two equations to a charge to find the charge within the plate.}
  \end{align*}
  Consider that if there are more charges on the plates, then the potential (difference) between them will be larger. Difference is in paranthesis because we could have had a $\Delta V$ instead of just a V if we wanted. The potential goes with the field, and the field is larger if we have more charges in it.
  \begin{align*}
    Q~V\\
    \shortintertext{Let us introduce capacitance $C$ such that $Q=CV$. In magnitude, we found $V=Ed$, which gives the potential difference accross the plates. And $E=\frac{\sigma}{\epsilon_0}$}\\
    V&=\frac{\sigma}{\epsilon_0}d=\frac{1}{\epsilon}\frac{Q}{A}d\\
    \shortintertext{Using our new definition of $V=\frac{Q}{C}$, we find:}\\
    \frac{1}{\epsilon_0}\frac{Q}{A}d&=\frac{Q}{C}\\
    C&=\epsilon_0 \frac{A}{d}\\
    \shortintertext{This is for parallel plate capacitance. This only depends on geometry. Due to the equation $Q=CV$, as soon as we know the charge on the plates we can easily determine the potential energy of the plates. $Q=CV$ is true for all capacitors.}
  \end{align*}
  Let's discuss combining capacitors. For a parallel plate capacitor, we found that it's capacitance is $C=\epsilon_0 \frac{A}{d}$. What if we stacked plates together? What would the capacitancebe for this combined system?
  \begin{align*}
    A_{new}&=A_1+A_2+A_3\\
    &=\frac{d}{\epsilon_0}C_1+\frac{d}{\epsilon_0}C_2+\frac{d}{\epsilon_0}C_3\\
    \frac{d}{\epsilon_0}C_{new}&=\frac{d}{\epsilon_0}C_1+\frac{d}{\epsilon_0}C_2+\frac{d}{\epsilon_0}C_3\\
    \shortintertext{d is the same everywhere, and $\epsilon_0$ is a constant so they both cancel}\\
    C_{new}&=C_1+C_2+C-3\\
    C_{new}&= C_{parallel}\\
    \shortintertext{Capacitants increase when area is increased.}
  \end{align*}
  What if instead we put capacitors one after another (in sequential order)? The area is the same, and the two middle sets of plates are going to have zero charge.
  \begin{align*}
    d_{new}&=d_1+d_2+d_3\\
    \epsilon_0\frac{A}{C_{new}}&=\epsilon_0\frac{A}{C_1}+\epsilon_0\frac{A}{C_2}+\epsilon_0\frac{A}{C_3}\\
    \frac{1}{C_{series}}&=\frac{1}{C_1}+\frac{1}{C_2}+\frac{1}{C_3}+...\\
    C_{series}&=\left[\frac{1}{C_1}+\frac{1}{C_2}+\frac{1}{C_3}+...\right]\\
  \end{align*}
  Capacitance let's us calculate the energy stored inside of the field between its plates. This is similar to throwing a ball from a lower surface to a hgiher one. The kinetic energy is changed into potential energy. 
  \begin{align*}
    \shortintertext{Recall that $U=VQ$.}\\
    \shortintertext{Consider:}\\
    dU&=VdQ\\
    \shortintertext{and $Q=CV$}\\
    \to dU&=\frac{Q}{C}dQ\\
    U&=\frac{1}{C}\int QdQ\\
    &=\frac{1}{C}\frac{1}{2}Q^2\\
    &=\frac{1}{2}\frac{1}{C}C^2V^2\\
    \alignedbox{U}{=\frac{1}{2}CV^2}
    \shortintertext{Or}\\
    \alignedbox{U}{=\frac{1}{2}\frac{Q^2}{C}}\\
    U&=\frac{1}{2}Q^2\frac{V}{Q}\\
    \alignedbox{U}{\frac{1}{2}QV}\\
  \end{align*}
  Let's discuss energy density.
  \begin{align*}
    u&=\frac{U}{volume}\\
    \shortintertext{This is always true, but for a parallel-plate capacitor, the volume is:}
    volume&=Ad\\
    \shortintertext{Then,}\\
    U&=\frac{1}{2}CV^2\\
    \to u&=\frac{1}{2}CV^2\frac{1}{Ad}\\
    &=\frac{1}{2}\left(\epsilon_0\frac{A}{d}\right)V^2\frac{1}{Ad}\\
    u&=\frac{\epsilon_0}{2}\frac{V^2}{d^2}\\
    u&=\frac{\epsilon_0}{2}\left(\frac{V}{d}\right)^2\\
    \shortintertext{Recall,}\\
    V&=Ed\\
    \to E&=\frac{V}{d}\\
    \shortintertext{We find}\\
    \alignedbox{u}{=\frac{\epsilon_0}{2}E^2}\\
    \shortintertext{Energy density relies solely on the energy field between the plates.}\\
  \end{align*}

  \subsubsection{Example 1}
  Let's find the capacitance C of the demo capacitor. The area of the capacitor is A ($A=\pi r^2$) for $r=6cm$. $A=\pi 35cm^2$, also d is 2cm. We also need $\epsilon_0$. This number is a constant. i$\epsilon_0 = 8.85\times10^{-12} \frac{F}{m}$, where F stands for ferads. For parallel plates, $C=\epsilon_0\frac{A}{d}$. The units are ferads $(F)$.
  \begin{align*}
    C&=\left(8.85\times10^{-12}\frac{F}{m}\right)\left(\frac{\pi 36cm^2}{2cm}\right)\left(\frac{1m}{100cm}\right)\\
    \alignedbox{C}{=5.00\times10^{-12}F=5pF}\\
    \shortintertext{C of the other two objects:}\\
    C_{blue}&=10\mu F=10\times10^{-6}F=10^{-5}F\\
    C_{brown}&=0.33\mu F=0.33\times10^{-6}F=3.3\times10^{-7}\\
  \end{align*}
  The capacitance of a capacitor is typically given within tolerances, not as raw numbers.
  Let's find the electric field around the first capacitor. If there is no charge on the plates, then there is no electric field between the plates, thus the capacitor is not storing any energy. An analogy to gravity: if there is no mass for two plates you're trying to calculate gravity for, then there is no gravity between the two plates. By adding electric charge to the plates, a field is created between them in which energy is stored.
  \begin{align*}
    U&=\frac{1}{2}CV^2\\
    Q&=CV\\
    U&=\frac{1}{2}QV\\
    &=\frac{1}{2}\frac{Q^2}{C}\\
    \shortintertext{Let's put a 9V battery across our capacitor.}\\
    U&=\frac{1}{2}CV^2\\
    &=\frac{1}{2}(5\times10^{-12}F)(9V)^2\\
    U&=2.03\times10^{-10}J\\
    \alignedbox{U}{=.203nJ}\\
  \end{align*}
  Consider a microwave oven. Many are powered at around 1200 Watts. We run it for 2 minutes. How much energy does this use? How much time is how much energy used?
  \begin{align*}
    P&=\frac{\Delta E}{\Delta T}\\
    \Delta E &= P\Delta T\\
    &=\left(1200\frac{J}{s}\right)(120s)\\
    \alignedbox{\Delta E}{=144000J}\\
    &=144kJ
  \end{align*}
  There are some benefits for something using a low amount of power.  Let's recall that $C=5\times10^{-5}$. Q of d would like instead $10\times10^{-17}F$. d can combine capacitors. Consider $C=\epsilon\frac{A}{d}$. If you want more capacitance, you can increase the area. If you double the area you double the capacitance.
  \begin{align*}
    C&=C_1+C_2\\
    &=2C\\
    &=10\times10^{-12}\\
  \end{align*}
  \subsubsection{Example 2}
  There is a potential across the plates such that if a charge is placed you can easily calculate the potential energy and then the kinetic energy. Let's find the capacitants of something that is not necessarily parallel plates. We still need 2 plates, but they are not going to be flat plates. We are going to calculate the capacitance of a spherical capacitor.
  \begin{align*}
    Q&=CV\\
    C_{parallelplates}=\epsilon_0\frac{A}{d}\\
    \shortintertext{The capacitance of a capacitor is always fully defined.}\\
    \shortintertext{For the spherical case, let's find C. Q is the charge on either plate, NOT BOTH. This is to prevent having zero as your Q value. The easiest way to find the voltage is by using Gauss' law to find $\vec{E}$ to find the electric field then using that to solve for the volutage.}\\
    \vec{E}&=0 \text{ }r<a\\
    \vec{E}&=0 \text{ }r>b\\
    \shortintertext{For $a<r<b$:}\\
    \oint E\cdot d\vec{a}&=\frac{q_{enc}}{\epsilon_0}\\
    E(4\pi r^2)&=\frac{Q}{\epsilon_0}\\
    \alignedbox{\vec{E}}{=\frac{1}{4\pi\epsilon_0}\frac{Q}{r^2}\hat{r}}\\
    V&=-\int\vec{E}\cdot d\vec{l}
    \shortintertext{From $r=a$ to $r=b$}\\
    V&=-\int_a^b\frac{1}{4\pi\epsilon_0}\frac{Q}{r^2}\hat{r}\cdot d\vec{r}\\
    \shortintertext{Potential is always path independent. The dot product enforces it in this situation. It always ends up being $E\cdot dl$ which is in the $r$ direction. Therefore the direction does not matter but the start and endpoints do matter. Analagous to gravity where the potential energy remains the same regardless of the horizontal position of an object.}\\
    &=-\frac{1}{r\pi\epsilon_0}Q\int_a^b\frac{1}{r^2}\hat{r}\cdot dr\hat{r}\\
    \shortintertext{This can be done because it is being written in it's direction vector times the magnitude. This will make the computations easier. Now we are able to find the dot product because the unit vectors are being dotted now as well.}\\
    &=-\frac{1}{4\pi\epsilon_0}Q\int_a^b r^{-2}dr\\
    V&=-\frac{1}{4\pi\epsilon_0}Q(-r^{-1})\Big|_{r=a}^{r=b}\\
    V&=\frac{1}{4\pi\epsilon_0}Q\left(\frac{1}{b}-\frac{1}{a}\right)\\
    \shortintertext{We need to ask ourselves if the sign or the units make sense for the sign. From $a\to b$, the voltage should decrease because it's being taken away from the electric charge. V is always a scalar so our "direction" is correct. We can now use $Q=CV$. $Q$ is not the total charge, it is the charge on either plate.}\\
    C&=\frac{Q}{V}=\frac{|Q|}{|V|}\\
    &=\frac{Q}{\frac{1}{4\pi\epsilon_0}Q\left(\frac{1}{a}-\frac{1}{b}\right)}\\
    &=4\pi\epsilon_0\left(\frac{1}{a}-\frac{1}{b}\right)^{-1}\\
    \alignedbox{C}{=4\pi\epsilon_0\frac{ab}{b-a}}
  \end{align*}
  \subsection{Dialectics}
  We can nicely build capacitors by placing insulators between the plates. If you can decrease the space between the capacitors, then there is a higher capacitance level. For parallel plates,
  \begin{align*}
    \vec{E_{new}}&=\vec{E}+\vec{E_i}\\
    \vec{E_{new}}&<\vec{E}\\
    \vec{E_{new}}&=\frac{1}{\kappa}\vec{E}\\
    C&=\kappa C_0\\
  \end{align*}
  \subsection{Current and Resistance}
  Electric current is the movement of charges in time. $I=\frac{dq}{dt}$ This means that in order to have a current, the mobile charges must be in the presence of an electric field. If you apply a force, then the charges start to move and the movement of charges is an electric current. Imagine a region with some charges and due to external charges, an external electric field affects the charges in the original region. All of the charges are going to move in the direction of the field. Because of this, the motion of the charges is messy. Also, electrons on a conductor. The electrons are all initially going to be spread out along the conductor, but if an electric field is implemented along the surface of the field, all of the charges are going to move the opposite direction of the electric field. The charges end up zigzagging along the path of the conductor.
  \newline\newline
  Charge is given in Coulombs (C). The elementary charge is $e=1.602\times 10^{-19}C$. The unit for current is $\frac{C}{s}$. This is also called an ampere or amp (A). \newline\newline Consider a copper wire being influenced by an external electric field (figure 4.9.1). $v_d$ is the drift velocity. This is the average speed of charged particles moving in a current. What about for electrons in a copper wire?
  \begin{align*}
    \shortintertext{We've said:}
    I&=\frac{dq}{dt}\\
    \shortintertext{Let's break it down into individual charges:}\\
    I&=\frac{d(eN)}{dt}\\
    &=e\frac{dN}{dT}\\
    &=e\frac{ndV}{dt}\\
    \shortintertext{Where $n$ is the number of electrons in a volume}\\
    &=e\frac{nAdl}{dt}\\
    &=enAv_d\\
    \alignedbox{v_d}{=\frac{I}{enA}}\\
  \end{align*}
  For a current of 10A, consider a copper wire of radius 1mm.We know that $e=1.602\times 10^{-19}$, and $A=\pi r^2$ for $r=1mm$. The number density is around $10^{23}\frac{1}{cm}\left(\frac{100cm}{m}\right)^3=10^{29}\frac{1}{m^3}$. Plugging these numbers in we will find a typical drift velocity of about $v_d \approx 10^{-4}\frac{m}{s}$ or $0.1\frac{mm}{s}$. This is very slow.\newline\newline
  Current dencity is: 
  \begin{align*}
    dI&=\vec{J}\cdot d\vec{A}\\
    I&=\int\vec{J}\cdot d\vec{A}\\
  \end{align*}
  Why do I care about $\vec{J}$ in the first place? When the electrons move in the wire, the external electric field forces them down the wire making those electrons move through a volume of the form $V=Al$. A is the crosssectional area of the wire, so current density let's us relate the movement of the electrons to the shape/geometry of the wire. $\vec{J}$ flows through $d\vec{A}$. The stronger the external field, the larger the current density. 
  \begin{align*}
    \vec{J}&\approx\vec{E}\\
    \vec{J}&=\sigma\vec{E}\\
    \vec{J}&=\frac{1}{\rho}\vec{E}\\
  \end{align*}
  $\sigma$ is electric conductance, and $\rho$ is electric resistivity. Further, $\rho=\frac{1}{\sigma}$. Both $\rho$ and $\sigma$ depend of the properties of the materials of the wire. You can think of $\sigma$ as telling us how good of a conductor a given material is. Whereas, $\rho$ tells us how poor of a conductor the material is.
  \begin{align*}
    \shortintertext{Figure 4.9.3}
    \vec{E}=\rho\vec{J}\\
    \shortintertext{Let's consider uniform $\vec{E}$ considering magnitudes:}\\
    E&=\rho J\\
    &=\rho\frac{I}{A}
    \shortintertext{You can do this substitution because of the following:}\\
    \int dI&=\int\vec{J}\cdot d\vec{A}\\
    &=\int JdA\\
    &=J\int dA\\
    I&=JA\\
    \shortintertext{When electric charges are uniform we can calculate the voltage:}\\
    V&=-\int\vec{E}\cdot d\vec{l}\\
    &=-\int Edl\\
    &=-E\int dl\\
    V&=-El\\
    \shortintertext{Now substituting for E:}
    \frac{V}{l}&=\rho\frac{I}{A}\\
    V&=I\rho\frac{l}{A}\\
    \text{Where }R&=\rho\frac{l}{A}\text{ called resistance.}\\
    \shortintertext{Notice that R depends on material $\rho$ and on geometry $\frac{l}{A}$}\\
    I=\frac{V}{R}=\frac{V}{\rho}\frac{A}{l}\\
  \end{align*}
  Resistance depends on geometry and resistivity. Resistivity is a property of a material (intrinsic property). $R=\rho\frac{l}{A}$.
  \subsubsection{Example 1}
  Let's find the resistivity $\rho$ for the power resistor. We know that $R=10\Omega$. The length of the resistor is $L=18cm$, and the area is $A=2cm\times1cm$. Using the equation above, we can solve for $\rho$ and find the resistance of the object.
  \begin{align*}
    R&=\rho\frac{l}{A}\\
    \rho&=R\frac{A}{L}\\
    \rho&=10\Omega\left(\frac{2cm^2}{18cm}\right)\left(\frac{1m}{100cm}\right)\\
    \alignedbox{\rho}{=0.0111\Omega m}\\
  \end{align*}
  For comparison's sake, the resistivity of silver is $\rho_{silver}=1.59\times10^{-8}\Omega m$.
  \subsubsection{Example 2}
  Resistors in series. There is a long resistor with another resistor right behind it ($r_1,r_2, r_3$). What is the net (equivalent) resistance? An electron goes from one resistor to the next and so on. 
  \begin{align*}
    R&=\rho\frac{L}{A}\to L=R\frac{A}{\rho}\\
    L_{eq}&=L_1+L_2+L_3\\
    R_{eq}\frac{A}{\rho}&=R_{1}\frac{A}{\rho}+R_{2}\frac{A}{\rho}+R_{3}\frac{A}{\rho}\\
    \alignedbox{R_{eq}}{=R_1+R_2+R_3}\\
  \end{align*}
  \subsubsection{Example 2}
  Resistors in parallel. If resistors are in parallel, then the charges see the resistors as one large resistor. Analogous to a river. If a river widens, then the water slows down. Adding up the areas:
  \begin{align*}
    A_{eq}&=A_1+A_2+A_3\\
    A&=\rho\frac{l}{R}\\
    \rho\frac{L}{R_{eq}}&=\rho\frac{L}{R_{1}}+\rho\frac{L}{R_{2}}+\rho\frac{L}{R_{3}}\\
    \alignedbox{\frac{1}{R_{eq}}}{=\frac{1}{R_{1}}+\frac{1}{R_{2}}+\frac{1}{R_{3}}}\\
  \end{align*}
  In series:\newline
  $R_{eq}=R_1+R_2+R_3$\newline
  $C_{eq}=\left(\frac{1}{C_1}+\frac{1}{C_2}+\frac{1}{C_3}+\right)$\newline
  In parallel:\newline
  $R_{eq}=\left(\frac{1}{R_1}+\frac{1}{R_2}+\frac{1}{R_3}+\right)$\newline
  $C_{eq}=C_1+C_2+C_3$\newline
  For some capacitors, if there are only two of them
  \begin{align*}
    C_{eq}&=\left(\frac{1}{C_1}+\frac{1}{C_2}\right)^{-1}\\
    &=\frac{1}{\frac{1}{C_1}+\frac{1}{c_2}}\\
    C_{eq}&=\frac{1}{\frac{1}{C_1}+\frac{1}{c_2}}\frac{C_1C_2}{C_1C_2}\\
    &=\frac{1}{C_2+C_1}\frac{C_1C_2}{1}\\
    \alignedbox{C_{eq}}{=\frac{C_1C_2}{C_1+C_2}}\\
    \alignedbox{R_{eq}}{=\frac{R_1R_2}{R_1+R_2}}\\
  \end{align*}
  Electrical Power
  \begin{align*}
    P&=\frac{dW}{dt}=\frac{dU}{dt}\\
    P&=\frac{d(qV)}{dt}\\
    \shortintertext{For a given voltage, the power output is:}
    P&=V\frac{dq}{dt}\\
    P&=VI\\
    \alignedbox{P}{=IV}\\
    \shortintertext{For a circuit with a resistance R,}\\
    V&=IR\\
    \to P=I^2R \text{ and } P=\frac{V^2}{R}\\
  \end{align*}
  Household outlets in America supply 120 Volts. In order to change the power output of an appliance, we must adjust its resistance to change the current. We want $P_{max}=I_{max}V$. We know that $V=IR$.
  \begin{align*}
    I_{max}&=\frac{V}{R_{max}}
  \end{align*}
  For some resistor of resistance R, the power it outputs is $P=I^2R$. The resistor will disipate energy as heat.
  \subsubsection{Example 3}
  Say a wire of diameter $d=4mm$ has a current through it, $I=6mA$. Assume a uniform current density.
  \begin{align*}
    dI&=\vec{J}\cdot d\vec{a}\\
    \int dI &=\int Jda\\
    I&=J\int da\\
    I&=JA\\
    \alignedbox{J}{=\frac{I}{A}}\\
    \shortintertext{Recall,}\\
    \vec{E}&=\sigma\vec{J}\\
    \to V&= IR\\
  \end{align*}
  \subsection{Circuits and Circuit Analysis}
  Let's start with an example and then lets discuss the physical implications of the example after
  \subsubsection{Example 1}
  \begin{figure}[!h]
    \centering
    \begin{circuitikz} \draw
    (0,0) to[battery=$V_0$] (0,4)
          to[resistor=$R_0$, i>_=$I_0$] (4,4)
          to[resistor=$R_1$, i>_=$I_0$] (4,0)
          to[resistor=$R_2$, i>_=$I_0$] (0,0)
    ;
    \end{circuitikz}
  \end{figure}
  There is one current that describes this entire circuit. Let's let $V_0 = 11V$, $R_0=10k\Omega$, $R_1=12k\Omega$, and $R_2=35k\Omega$. To get $I_0$ we need to know $R_eq$. These resistors are all in series with each other. Because they are all in series,we know that:
  \begin{align*}
    R_{eq}&=R_0+R_1+R_2\\
    &=57k\Omega\\
    \shortintertext{Then with $V=IR$ we set:}\\
    V_0&=I_0R_{eq}\\
    \alignedbox{I_0}{=\frac{V_0}{R_{eq}}}\\
    I_0&=\frac{11V}{57k\Omega}=0.193mA
  \end{align*}
  If you start somewhere in a gravatational field, and you move something down, it has changed in its potential energy. Because electric force is conservative, a charge that travels any path and then comes back to its original location has no net change in potential energy
  \begin{align*}
    \shortintertext{For a given loop,}\\
    \sum qV_{loop}&=0\\
    \sum V_{loop}&=0\\
  \end{align*}
  To use this, pick any starting location. Take a complete path that's called a loop and consider the change in voltage across various circuit elements. For a battery or a power supply, if you travel from the negative to the positive, you have an increase in voltage across the battery. Converse also works. For a resistor, if you are traveling \underline{with} the current, voltage will \underline{drop} across the resistor. 
  \newline\newline
  Let's try a clockwise loop from the top left corner.
  \begin{align*}
    \sum V_{loop} &=\\
    -I_0R_0-I_0R_1-I_0R_2+V_0&=0\\
    V_0&=I_0(R_0+R_1+R_2)\\
    V_{R_0}&=-I_0R_0=(-0.193mA)(10k\Omega)\\
    V_{R_0}&=-1.93V\\
    V_{R_1}&=-2.32V\\
    V_{R_2}&=-6.76V\\
    \alignedbox{\sum V_{loop}}{=-11.0V}\\
    P_{R_0}&=IV\\
    IV&=0.37mW\\
  \end{align*}
  \subsubsection{Example 2}
  \begin{figure}[!h]
    \centering
    \begin{circuitikz}
      \draw
      (4,4) to[battery=$I_0$] (0,4) 
            to[short, -*, i=$I_0$] (0,2)
            node[label={[font=\footnotesize]left:X}] {}
            to[R=$R_1$, i>_=$I_1$] (4,2) 
            node[label={[font=\footnotesize]right:Y}] {} -- (4,4)
      (0,2) -- (0,0)
            to[R=$R_2$, i>_=$I_2$] (4,0) 
            to[short, -*] (4,2)
      ;
    \end{circuitikz}
  \end{figure}
  \begin{align*}
    \sum V_{loop}&=0\\
    a&:-V_0+I_1R_1=0\\
    b&:-I_1R_1+I_2R_2=0\\
    \alignedbox{\sum I_{in}}{=\sum I_{out}}\\
    x&:I_0=I_1+I_2\\
    y&:I_1+I_2=I_0\\
  \end{align*}
  \newline
  There are basically two rules when it comes to circuit analysis. $\sum V_{loop} = 0$ and $\sum I_{in}=I_{out}$. This is essentially conservation of energy and conservation of charge. Let's consider the following circuit.
  \begin{figure}[!h]
    \centering
    \begin{circuitikz}
      \draw
      (4,4) to[isource=$I_0$] (0,4) 
            to[short, -*, i=$I_0$] (0,2)
            node[label={[font=\footnotesize]left:X}] {}
            to[R=$R_1$, i>_=$I_1$] (4,2) 
            node[label={[font=\footnotesize]right:Y}] {} -- (4,4)
      (0,2) -- (0,0)
            to[R=$R_2$, i>_=$I_2$] (4,0) 
            to[short, -*] (4,2)
      ;
    \end{circuitikz}
  \end{figure}
  Let's let $V_0=11V$, $R_1=12k\Omega$, and $R_2=35k\Omega$. First let's find the current directly out of the battery.
  \begin{align*}
    R_{eq}&=\frac{R_1R_2}{R_1+R_2}=8.94k\Omega\\
    V_0&=I_0R_{eq}\\
    \alignedbox{I_0}{=\frac{V_0}{R_{eq}}}\\
    I_0&=\frac{11V}{8.94k\Omega}=1.23mA\\
    P_0=I_0V_0&=(1.23mA)(11V)=13.5mW\\
    \shortintertext{Now let's find the current through each resistor. Let's start with going from the batter to $R_1$ and back (a). and then going from $R_2$ through $R_1$. The third loop could go from the battery to $R_2$ and back but it is not important to do so because all of the variables show up in these two equations already.}\\
    \sum V_{loop}&=0\\
    a&) V_0-I_1R_1 = 0\\
    b&) I_1R_1-I_1R_2 = 0\\
    \sum I_{in}&=\sum I_{out}\\
    x&)I_0=I_1+I_2\\
    y&)I_1+I_2=I_0\\
    \shortintertext{We have already done the physics, and now we must solve for the things that we do not know. This is just basic algebra.}\\
    \to V_0&=I_1R_1\\
    I_1&=\frac{V_0}{R_0}\\
    &=\frac{11V}{12k\Omega}\\
    \alignedbox{I_1}{=0.917mA}\\
    \to I_2&=I_0-I_1\\
    &=1.23mA-0.917mA\\
    \alignedbox{I_2}{=0.313mA}
    \shortintertext{Now let's find the power of these two resistors. Remember $V=IV=I^2R=\frac{V^2}{R}$}\\
    P_1&=I_1^2R_1=(0.917mA)^2(12k\Omega)\\
    P_1&=10.1mW\\
    P_2&=I_2^2R_2=(0.313mA)^2(35k\Omega)\\
    P_2&=3.43mW\\
    \shortintertext{Notice that $P_1+P_2=13.5mW$, which is the same amount of power that the batter puts out! Now let's find the voltage drops across $R_1+R_2$ from $IV$}\\
    V_1&=\frac{P_1}{I_1}=\frac{10.1mW}{0.916mA}\\
    V_1&=11.0V\\
    V_2&=\frac{P_2}{I_2}=\frac{3.43mW}{0.313mA}\\
    V_2&=11.0V\\
    \shortintertext{Notice that $V_1=V_2=V_0$. We get the same voltage across all three.}
  \end{align*}
  The voltage drops across parallel segments of circuit are \underline{equal}. $V_0$, $R_1$, and $R_2$ are all in parallel, which means they all have the same voltage. We essentially did it with the first loop but we did not. Energy is conserved, but voltage is \underline{NOT} conservative. To determine how to combine the resistors, you \underline{must} first determine where the charge will flow.
  \subsubsection{Example (Draw Later)}
  \begin{figure}[h!]
    \centering
    \begin{circuitikz}
      \draw (10,0) to[isource=$V_0$] (10,8) 
                   to[R=$R_0$](5,8) -- (0,8)
                   to[R=$R_4$](0,4) 
                   to[R=$R_5$](0,0) -- (10,0)
            (5,8)  to[R=$R_1$] (5,4) -- (7.5, 4) 
                   to[R=$R_3$](7.5,0)
            (5,4) -- (2.5, 4) 
                   to[R=$R_2$](2.5, 0);
    \end{circuitikz}    
  \end{figure}
  There is a short within this circuit, which means it flows without resistance. Even with a short, you still calculate everything the same way. Let's find the current out of the battery and the power output of $R_0$
  \begin{align*}
    \sum V_{bat} &=0\\
    \sum I_{in} &= \sum I_{out}\\
    \shortintertext{A quick way to get $I_0$ is using $V_0=I_0R_{eq}$. Let's find $R_{eq}$}\\
    R_{eq}&=R_0\frac{R_{45}R_{123}}{R_{45}+R{123}}\\
    \text{Also, } R_{45}&=R_4+R_5\\
    R_{123}&=R_1+\frac{R_2R_3}{R_2+R_3}\\
    \alignedbox{R_{eq}}{=R_0+\frac{\left(R_4+R_5\right)\left(R_1+\frac{R_2R_3}{R_2+R_3}\right)}{R_4+R_5+R_1+\frac{R_2R_3}{R_2+R_3}}}\\
  \end{align*}
  \subsubsection{Using measuring tools}
  How to measure current using a digital multimeter.\newline\newline
  To measure current, we must carefullt add the ammeter to the circuit. To use an ammeter you must break the circuit. So in order to add an ammeter, first be sure the circuit is \underline{not} currently powered! Plug the hole with the ammeter and \underline{then} turn on the power supply.\newline\newline
  How to measure voltage using a DMM.\newline\newline
  Set the DMM to the voltmeter. Set the DMM to be a voltage. We then add the loads across the element whose voltage drop we want to measure. We must not break the circuit or turn off the power supply to measure the voltage.\newline\newline
  Ammeters go into circuits and have low internal resistance, while voltmeters go across circuit elements and have high internal resistance.
  \subsection{Resistors and Capacitors in Circuits}
  \begin{align*}
    V_R&=IR\\
    V_C=\frac{Q}{C}\\
    \shortintertext{When the swtich is closed:}\\
    V_R&=V_C=0\\
    \shortintertext{Let's throw the switch to the position (a):}.\\
    \shortintertext{There are no junctions and there is only one loop:}\\
    \sum V_{loop} = 0\\
    V_0-IR-\frac{Q}{C}&=0\\
  \end{align*}
  How much charge acumulates on the plates? What is Q in terms of capacitance, resistance, and the battery. The amount of charge that accumulates within the circuit is a property of the circuit only, wheras battery voltage, resistance, and capacitance are functions of themselves. Using $I=\frac{dQ}{dt}$, we get an equation that's a differential. We must get $dQ$ with $Q$ and we must get $dt$ with $t$.
  \begin{align*}
    V_0-R\frac{dQ}{dt}-\frac{Q}{C}&=0\\
    V_0-\frac{Q}{C}&=R\frac{dQ}{dt}\\
    CV_0-Q&=RC\frac{dQ}{dt}\\
    \frac{1}{RC}&=\frac{1}{CV_0-Q}\frac{dQ}{dt}\\
    \frac{dt}{RC}&=\frac{dQ}{CV_0-Q}\\
    \int_0^t\frac{dt}{RC}&=\int_0^Q\frac{dQ}{CV_0-Q}\\
    \shortintertext{Let $U=CV_0-Q$, and $du=-dQ$:}\\
    \frac{1}{RC}(t-0)&=\int_{CV_0}^{CV_0-Q}\frac{-du}{u}\\
    \frac{t}{RC}&=-ln\left(\frac{CV_0-Q}{CV_0}\right)\\
    e^{-\frac{t}{RC}}&=\frac{CV_0-Q}{CV_0}\\
    CV_0e^{-\frac{t}{RC}}&=CV_0-Q\\
    Q&=CV_0-CV_0e^{-\frac{t}{RC}}\\
    \alignedbox{Q(t)}{=CV_0\left(1-e^{-\frac{t}{RC}}\right)}\\
    \text{Charging} &\text{ a capacitor}\\
    \shortintertext{At $t=0$}
    Q(0)&=CV_0\left(1-e^{-\frac{0}{RC}}\right)=0\\
    \shortintertext{After a long time, $Q=CV_0$}\\
    V_0&=IR-\frac{CV_0}{C}=0\\
    V_0-IR&=V_0\\
    \shortintertext{This is only true if $I=0$. If we take the dervivative of our $Q(t)$ then we can see how current depends on time:}\\
    I(t)&=\frac{d}{dt}Q(t)\\
    &=CV_0\left(0-\frac{-1}{RC}e^{-\frac{t}{RC}}\right)\\
    \alignedbox{I(t)}{=\frac{V_0}{R}e^{-\frac{t}{RC}}}\\
    \shortintertext{Charging a capacitor. Notice that RC has dimension of time.}
  \end{align*}
  \newpage
\section{Magnetism}
    This is very similar to electricity. Magnetic fields are created by moving electric charges. Magnetic fields are represented as $\vec{B}$. The force that a $\vec{B}$ field exerts on a charge is $\vec{F}=q\vec{v}\times\vec{B}$, where $q$ is the charge, $\vec{v}$ is the velocity, and $\vec{B}$ is the field the charge is in.
    \subsubsection{Example 1}
    An electron is moving with speed v to the right. An external field, $\vec{B}$, points into the page. Find the force on the charge. We are going to be using an x-y axis with z not shown, but it points into the picture. (Figure 6.1)
    \begin{align*}
        \vec{v}&=v(-\hat{i})=-v\hat{i}\\
        \vec{B}&=B\hat{k}\\
        \vec{F}&=-e\vec{v}\times\vec{B}\\
        \vec{F}&=F(-\hat{i})\\
        \shortintertext{Also, let's compute $\vec{F}$}\\
        \vec{F}&=q\vec{v}\times\vec{B}\\
        &=(-e)(-\vec{v}\hat{i})\times(B\hat{k})\\
        &= ev\hat{i}\times B\hat{k}\\
        &=evB(\hat{i}\times\hat{k})\\
        &=evB(-\hat{j})\\
    \end{align*}
    \subsubsection{Example 2}
    Find the period of motion of a positive charge Q in an external magnetic field $\vec{B}$. Let $\vec{v} = v\hat{i}$ and $\vec{B}=B\hat{k}$. (Figure 6.2)
    \begin{align*}
        \vec{F}=Q\vec{v}\times\vec{B}\\
        \shortintertext{Cyclotron motion is the circular motion of a charge particle in a $\vec{B}$ field. This is \underline{uniform} circular motion. This means you cannot use a magnetic field to change speed. Magnetic fields do \underline{no work}.}
        F&=QvBsin\theta\\
        F&=QvB\\
        \shortintertext{Let's equate this force with $\vec{F}=ma$}\\
        QvB&=ma\\
        QvB&=m\frac{v^2}{R}\\
        \frac{Q}{m}&=\frac{v}{BR}
        \shortintertext{The period, $T$, of the field is given by $T=\frac{2\pi R}{v}$}\\
        T&=\frac{2\pi R}{BR\frac{Q}{m}}\\
        \alignedbox{T}{=2\pi\frac{m}{BQ}}\\
        f&=\frac{1}{2\pi}B\frac{Q}{m}\\
        \shortintertext{This is the cyclotron period and the cyclotron frequency.}\\
    \end{align*}


    \subsection{Creating Magnetic Fields}
    In the Biot Savart we know that electric chages that are moving create electric fields. In order to determine the direction we still must do the right hand rule. This is easier to visualize with a current (describes electromagnets.) Instead we are going to talk about a piece of charge:
    \begin{align*}
        \alignedbox{\vec{B}}{=\frac{\mu_0}{4\pi}\frac{q\vec{v}\times\hat{r}}{r^2}}\\
        \alignedbox{d\vec{B}}{=\frac{\mu_0}{4\pi}\frac{dq\vec{v}\times\hat{r}}{r^2}}\\
        \shortintertext{This is the Biot-Savart Law}
    \end{align*}
    \subsubsection{Example 1}
    Find magnetic field $\vec{B}$ a distance d from an infinitely long wire carrying a current I make sure you always place the $d\vec{l}$ so it creates a triangle.
    \begin{align*}
        d\vec{B}&=\frac{\mu_0}{4\pi}\frac{dq\frac{d\vec{l}}{dt}\times\hat{r}}{r^2}\\
        &=\frac{\mu_0}{4\pi}\frac{\frac{dq}{dt}d\vec{l}\times\hat{r}}{r^2}\\
        &=\frac{\mu_0}{4\pi}\frac{Id\vec{l}\times\hat{r}}{r^2}\\
        \shortintertext{Because the current changes over time, $\frac{dq}{dt}=I$. Current is moving charges so current moving on a wire thats dl long in a certain direction creates $d\vec{B}$. This is analagous to electric fields and solving for the electric field.}\\
        \vec{B}&=\frac{\mu_0}{4\pi}I\int_{-\infty}^\infty\frac{d\vec{l}\times\hat{r}}{r^2}\\
        \shortintertext{We took the right hand rule to find the direction of $\vec{B}$ but this does not tell us the magnitude of the vectors because $d\vec{l}$ is changing. We are going to use $sin\theta$ where theta is the angle between $d\vec{l}$ and $\hat{r}$ to make this easier to computate. $\hat{r}$ is a unit vector so its magnitude is zero. It is only there to help us find the direction.}\\
        \vec{B}&=\frac{\mu_0}{4\pi}I\int_{-\infty}^\infty\frac{dlsin(\theta)}{r^2}\\
        \shortintertext{Now we are going to take the $sin\theta$ and change it into the coordinates we are using (xyz). We can figure out that $r^2=x^2+d^2$ and that $sin\theta=\frac{d}{r}$}\\
        &=\frac{\mu_0}{4\pi}I\int_{-\infty}^\infty\frac{dx}{x^2+d^2}\frac{d}{\sqrt{x^2+d^2}}\\
        \shortintertext{d is a constant of integration because the current doesn't change along d at all.}
        &=\frac{\mu_0}{4\pi}Id\int_{-\infty}^\infty\frac{dx}{(x^2+d^2)^\frac{3}{2}}\\
        &=\frac{\mu_0}{4\pi}Id\left[\frac{x}{d^2}\frac{1}{\sqrt{x^2+d^2}}\right]_{x=-\infty}^{x=\infty}\text{ out of page}\\
        \shortintertext{We are going to divide both the top and bottom by x because infinity cannot be at the top of a fraction.}\\
        d\vec{B}&=\frac{\mu_0}{4\pi}Id\left[\frac{1}{d^2}\frac{1}{\sqrt{1+\frac{d^2}{x^2}}}\right]_{x=-\infty}^{x=\infty}\\
        \shortintertext{This is a problem because if we evaluate it now, we will get 0 as our magnetic field. Because of this we must do the following:}
        \vec{B}&=2\frac{\mu_0}{4\pi}\frac{I}{d}\frac{1}{\sqrt{1+\frac{d^2}{x^2}}}\Big|_{x=-\infty}^{x=\infty}\\
        \vec{B}&=\frac{\mu_0}{2\pi}\frac{I}{d}\left[1-0\right]\\
        \alignedbox{\vec{B}}{=\frac{\mu_0I}{2\pi d}\text{ out of page}}\\
    \end{align*}
    There is an additional right hand rule. Put your thumb in the direction of the current wire, then your wrapped fingers will give the direction of $\vec{B}$. 
    \subsubsection{Example 2}
    Remember that magnetic force is written as $\vec{F}=q\vec{v}\times\vec{B}$. In scenario 1 we are going to have a positive charge that is going to move initially to the right with some velocity v. It is about to enter a region with a magnetic charge that points into the page. What electric field can we add to the region to prevent the moving charge from deflecting? By the right hand rule, the particle is going to feel a force upward.
    \begin{align*}
        \vec{F_B}&=q\vec{v}\times\vec{B}\\
        \vec{F_B}&=qvB \text{, up}\\
        \shortintertext{We need $\vec{E}$ to cause an equal force down so that $\vec{F}_{net}=0$}\\
        \vec{F_E}&=q\vec{E}\\
        \shortintertext{$\vec{E}$ must point downward.}\\
        F_E&=F_B\\
        qE&=qvB\\
        E&=vB\\
    \end{align*}
    Now in scenario 2 we have a negative charge that is going to move initially to the right with a magnetic region pointing into the charge. With a negative charge and right hand rule, we must flip our hand over because of the negativity.
    \begin{align*}
        \vec{F_B}&=qvB\text{, Down. we need}\\
        \vec{F}&=q\vec{E} \text{ to point up.}\\
    \end{align*}
    We actually get the same answer as before. This is because they both depend on q, and if the sign on q changes, it changes on both sides of the equation.\newline\newline 
    In scenario 3, our charged particle will be positive and heading right, but the magnetic field is in line with the charge's movement.
    \begin{align*}
        \vec{F_B}&=q\vec{v}\times\vec{B}=0\\
        \to &=qvBsin(0)=0\\
    \end{align*}
    In this case we don't need an $\vec{E}$ field at all to keep the charge from deflecting.\newline\newline
    With scenario 4, we have a positive charge moving to the right into a magnetic field that has a diagonal vector.
    \begin{align*}
        \vec{F_B}&=q\vec{v}\times\vec{B}\\
        F_B&=qvB_{\perp}\\
        \shortintertext{We must cross with the vertical component of B because the horizontal is parallel and gives 0}\\
        \vec{F_B}&=qvBsin(\theta)\text{, out of page}\\
        \text{$\vec{E}$ is into the page}\\
        qE&=qvBsin(\theta)\\
        \to E&=vBsin(\theta)\\
    \end{align*}
    \subsubsection{Example 3}
    Let $\vec{B}=B(-\hat{j})=-B\hat{j}$ and $\vec{v}$ that initially looks like $\vec{v}=v_{0_y}\hat{j}+v_{0_z}\hat{k}$. What will the motion of q look like?
    \begin{align*}
        \vec{F}&=q\vec{v}\times\vec{B}\\
        &=q(v_{0_y)}\hat{k}+v_{0_z)}\hat{k})\times(-B\hat{j})\\
        &=-q(v_{0_y}B(\hat{j}\times\hat{j})+v_{0_z}B(\hat{k}\times\hat{j}))\\
        &=-qv_{0_z}B(\hat{k}\times\hat{j})\\
        &=-qv_{0_z}B(-\hat{i})\\
        \alignedbox{\vec{F_0}}{=qv_{0_z}B\hat{i}}\\
    \end{align*}
    This will cause a helical motion: linear in $\hat{j}$ and cyclotron in the $\hat{i}$ and $\hat{k}$ directions. Basically it creates a spring.
    \subsubsection{Example 4}
    Force on a current.
    \begin{align*}
        \vec{F}&=q\vec{v}\times\vec{B}
        \shortintertext{Imagine a small amount of charge dq. The force exerted on this small amount of charge dq by an external field $\vec{B}$ is:}\\
        d\vec{F}&=dq\vec{v}\times\vec{B}\\
        d\vec{F}&=dq\frac{d\vec{l}}{dt}\times\vec{B}\\
        d\vec{F}&=\frac{dq}{dt}d\vec{l}\times\vec{B}\\
        \alignedbox{d\vec{F}}{=Id\vec{l}\times\vec{B}}\\
        \shortintertext{If everything is uniform, }\\
        \int d\vec{F}&=\int Id\vec{l}\times\vec{B}\\
        \vec{F}&=I\vec{l}\times\vec{B}\\
    \end{align*}
    \subsubsection{Example 5}
    For a uniform external field, $\vec{B}=4mT$ into the page, which way will a free wire move if the current as shown is $I=2A$ CCW. Let $a=10cm$. The force will be \underline{to the right.}
    \begin{align*}
        \vec{F}&=I\vec{l}\times\vec{B}\\
        F&=IlB\\
        &=(2A)(10cm)(4mT)\\
        &=(2A)(0.1m)(4\times10^{-3}T)\\
        \alignedbox{\vec{F}}{=8\times10^{-4}T}\\
    \end{align*}


    \subsection{Holl Effect}
    Consider a metal plate. We will drive a current through this metal plate (figure 6.3). Let's add $\vec{B}$ into the page. The current will feel a force acting upwards due to the right hand rule. This is because our fingers go to the right and curl to inside the page, the thumb gives us up which is the direction in which the force will move. These two figures are almost the same exact setup, but there is a change in the measured voltage. \newline\newline
    If the measured voltage is positive, the current is made up of positively charged particles. If the measured voltage is negative, the current is made up of negative charged particles. It turns out that the charge carriers are negative (electrons).
    \subsubsection{Example 1}
    This is a Biot Savart example.
    \begin{align*}
        d\vec{B}&=\frac{\mu_0}{4\pi}\frac{Id\vec{l}\times\hat{r}}{r^2}\\
        \shortintertext{Let's find $\vec{B}$ at the center of an arc of current (figure 6.5). Remember that the cross product of $d\vec{l}\times\vec{B}$. For magnetic field, we put our thumb in the direction of the current and wrap our hands around to the direction of the field. The current is going to be going along the rod clockwise. $\hat{r}$ is always going to be perpendicular to $d\vec{l}$ because I is always going to be pointing directly out of the circle, while r always points inward.}\\
        d\vec{B}&=\frac{\mu_0}{4\pi}I\frac{dl*1*sin(90)}{r^2}\text{ into page}\\
        \int d\vec{B}&=\int\frac{\mu_0}{4\pi}I\frac{dl}{r^2}\text{ into page}\\
        \vec{B}&=\frac{\mu_0}{4\pi}I\int\frac{dl}{R^2}\text{ into page}\\
        \vec{B}&=\frac{\mu_0}{4\pi}\frac{I}{R^2}\int dl\\
        \vec{B}&=\frac{\mu_0}{4\pi}\frac{I}{R^2}\int Rd\theta\\
        &=\frac{\mu_0}{4\pi}\frac{I}{R}\int d\theta\\
        \alignedbox{\vec{B}}{=\frac{\mu_0I}{4\pi R}\theta\text{ into page}}\\
        \shortintertext{For a full loop of current, $\theta=2\pi$}\\
        \vec{B}&=\frac{\mu_0I}{4\pi R}2\pi\\
        \alignedbox{\vec{B}}{=\frac{\mu_0I}{2R}\text{ into page}}\\
        \shortintertext{If I was variable, then we couldn't pull it out of the integral, but the direction would remain the same.}
    \end{align*}


    \subsection{Ampere's Law}
    \begin{align*}
        \shortintertext{Figure 6.6}\\
        \alignedbox{\oint\vec{B}\cdot d\vec{l}}{=\mu_0I_{th}}\\
        \shortintertext{Very similar to Gauss' law. We must choose a path in which we can nicely handle the expression above.}\\
        \oint Bdl&=\mu_0I_{th}\\
        B\oint dl &=\mu_0I\\
        B2\pi r &=\mu_0 I\\
        \alignedbox{\vec{B}}{\frac{\mu_0I}{2\pi r}\text{ CCW}}\\
        \shortintertext{If you were to pick $d\vec{l}$ to go to the wrong direction, the dot product results in zero so you get: $-\oint Bdl=\mu_0(-I)$. The minuses cancel and you end up getting the same answer: $\vec{B}=\frac{\mu_0I}{2\pi r}$}\\
    \end{align*}
    A solenoid is a coil of wire in the shape of a slinky (it stays in place) this is shown in figure 6.7. Magnetic field lines never start or stop and they never diverge, meaning they are always parallel. An ideal solenoid has a length much larger than it's radius ($L>>R$). Solenoids are used in MRI machines, hence the name (Magnetic resistance imaging). They are the tube that people are pushed into.
    \subsubsection{Example 1}
    Imagine an infinite perfect solenoid (figure 6.8). Let's find $\vec{B}$ for an ideal solenoid. The $d\vec{l}$ loop must encapsulate some form of current in order to be useful with this equation
    \begin{align*}
        \oint \vec{B}\cdot d\vec{l}&=\mu_0 I_{th}\\
        \shortintertext{It is easiest to split this up into multiple loops because there are four sides for the rectangle we chose. We labeled the sides (1, 2, 3, 4) so we could label our integrals.}\\
        \int\vec{B}\cdot d\vec{l_{1}}+\int\vec{B}\cdot d\vec{l_{2}}+\int\vec{B}\cdot d\vec{l_{3}}+\int\vec{B}\cdot d\vec{l_{4}}&=\mu_0I_{th}\\
        \shortintertext{For the first integral, $\vec{B}\cdot d\vec{l_1}$ gives us $bl$. For the second and fourth integrals, the dot product equals zero. For the third segment, $\vec{B}=0$ so we cannot do anything with that. The amount of current all along the loop is the same throughout. I is the current through each of the wires so the current through one wire is going to be I.}\\
        Bl+0+0+0&=\mu_02I\\
        \shortintertext{In a more general case, let's say that $N$ currents pierce throguh the Amperian loop.}\\
        Bl&=\mu_0NI\\
        \shortintertext{If for the whole solenoid length L:}\\
        BL_{tot}&=\mu_0N_{tot}I\\
        B&=\mu_0\frac{N_{tot}}{L_{tot}}I\\
        \alignedbox{B}{=\mu_0nI}\text{ Where $n$ is turns per length}\\
    \end{align*}
    If you want a bigger field you can either increase the current or put more turns in the current.


    \subsection{Magnetic Inductance}
    We will first consider Faraday's Law.
    \begin{align*}
        \Phi_B&=\vec{B}\cdot d\vec{A}\\
        \shortintertext{Where $\Phi_B$ is magnetic flux.}\\
        \alignedbox{E_{ind}}{=-\frac{d}{dt}\Phi_B}\\
        \shortintertext{$E_{ind}$ is also known as E.M.F. which is an induced voltage or potential.}\\
    \end{align*}
    \subsubsection{Example 1}
    A $\vec{B}$ field is into the page and increasing linearly in time: $B=b_0\frac{t}{t_0}$. Find the direction and magnitude of current induced on a circular loop of wire with radius, r, and resistance, R, in plane with the page (figure 6.9). Electricity and magnetism are two different expressions of the same thing (electromagnetism). If a voltage is induced on a wire where we can calculate current we can just use Ohm's Law. We need a magnetic field that fluctuates in time. If the flux doesn't change in time then there is no induced electric field. While magnetic field is into the page everywhere, the resistor only cares about the magnetic field that is on the wire. It is okay that our B value does not have area dependence because we are calculating the flux of the magnetic field. We only need to discuss the area of magnetism when we get the flux.
    \begin{align*}
        \Phi_B&=\int\vec{B}\cdot d\vec{A}\\
        &=\int \left(B_0\frac{t}{t_0}\hat{k}\right)\cdot\left(dA\hat{k}\right)\\
        &=B_0\frac{t}{t_0}\int dA\\
        &=B_0\frac{t}{t_0}\pi r^2\\
        \shortintertext{Always find the flux before taking the time derivative. Now we are moving on to the time derivative:}\\
        \to\frac{t}{dt}\Phi_B&=\frac{d}{dt}B_0\frac{t}{t_0}\pi r^2\\
        &=B_0\frac{\pi r^2}{t_0}\\
        \shortintertext{Recall that $E_{ind}=-\frac{d}{dt}\Phi_B$. The negative sign is because of Lenz's law. This basically just helps determind the direction of the magnitude of $E_{ind}$. \underline{Lenz's Law:} The induced current $I_{ind}=\frac{E_ind}{R}$ has a direction such that $B_{ind}$ opposes the change in flux, that is, $\frac{d}{dt}\Phi_B$. For this case, $I_{ind}$ must flow counter-clockwise because it must oppose the change in the flux.}\\
        |E_{ind}|=|\frac{d}{dt}\Phi_B|=|B_0\frac{\pi r^2}{t_0}|\\
        \alignedbox{I_{ind}}{=\frac{1}{R}B_0\frac{\pi r^2}{t_0}\text{, ccw}}\\
    \end{align*}
    Here are some other Lenz's law cases (figure 6.10). The equation for $\vec{B}$ in this case is $\vec{B}=B\frac{t}{t_0}\hat{k}$. The flux is \underline{decreasing} into the page so the $B_{ind}$ is going to point into the page, which would mean that our $I_{ind}$ is going clockwise. You \underline{oppose the change in the flux}. You do \underline{not} oppose the direction of the magnetic field.\newline\newline
    Let's now look at figure 6.11. This figure has a square loop in the middle and we know that $I_{int}$ goes clockwise because the magnetic force is going into the page and not out of the page. With $B=B_0e^\frac{t}{t_0}$, there is no real change in this problem compare to the last. We must also understand that we deal with a square loop the same way we dealt with a circular 
    one.


    \subsection{Eddy Currents}
    Similar to how if you're rowing a boat. As you pull your oar through the water you get little eddy's around the stick. These are magnetically induced currents that appear when Farade's law results in the slowing down of an object. This is a type of induced current. An example of this is sorting recyclables. You have a platform with a bunch of material heading down a slope. If the box is cardboard, then it will have an induced $E_{mf}$, but won't feel an induced current. If the recyclables are made from aluminum such as cans, then they will continue moving downward.
    \subsubsection{Motional Emf (E)}
    This goes along with figure 6.12. What direction will the force be in?
    \begin{align*}
        \vec{F}&=I\vec{l}\times\vec{B}\\
        &=I_{ind}\vec{l}\times\vec{B_{ext}}\\
        \shortintertext{Say we pull such that the speed of the loop is constant. Because we already know the direction we are going to just use the magnitude to determine what $E_{ind}$ is.}\\
        E_{ind}&=-\frac{d}{dt}\Phi_B\\
        |E_{ind}|&=|-\frac{d}{dt}\Phi_B|\\
        |E_{ind}|&=\frac{d}{dt}(BA)\\
        |E_{ind}|&=B\frac{d}{dt}A+A\frac{d}{dt}B\\
        &=B\left(L\frac{dx}{dt}+x\frac{dL}{dt}\right)\\
        &=BL\frac{dx}{dt}\\
        \shortintertext{We know that the speed is not changing so we can come to the conclusion that:}\\
        \alignedbox{|E_{ind}|}{=B_{ext}Lv}\\
        \shortintertext{From Ohm's law:}\\
        E_{ind}&=I_{ind}R\\
        I_{ind}&=\frac{B_{ext}Lv}{R}\\
        P&=I_{ind}E_{ind}\\
        \alignedbox{P}{=\frac{B_{ext}^2L^2v^2}{R}}\\
    \end{align*}


    \subsection{Induced Electric Field}
    Farade's Law: $E_{ind}=-\frac{d}{dt}\Phi_B$ (Figure 6.13). This means that $E_{ind}$ points the same direction as $I_{ind}$. This gives us that $\oint\vec{E_{ind}}\cdot d\vec{l}=-\frac{d}{dt}\Phi_B$. These charged particles must be in a loop in order to come to this conclusion. Farade's law helps us further relate the previous equation.
    \begin{align*}
        \oint\vec{E_{ind}}\cdot d\vec{l}&=-\frac{d}{dt}\int\vec{B}\cdot d\vec{a}\\
    \end{align*}
    A time varying magnetic flux induces an electric field. Electric and magnetic fields are frame dependent. They depend on what inertial frame they are in. An inertial frame is one in which there is no acceleration.
    \subsubsection{Induction Applied to Circuits}
    Let's consider a solenoid. If we apply a current then there will be a magnetic field throughout the inside of the solenoid. The best way to solve for the magnetic field within a solenoid is by using Ampere's law, $B=\mu_0nI$. The flux through the solenoid is:
    \begin{align*}
        \Phi_B&=\int\vec{B}\cdot d\vec{a}\\
        \Phi_B&=Ba\\
        \shortintertext{We can say that because the flux is proportional to the magnetic field, the flux is proportional to the current as well. We now introduce inductance L.}\\
        \Phi_B&=LI\\
        \shortintertext{L depends on the solenoid's geometry.}\\
        N\Phi_B&=LI\\
        \shortintertext{Where N is the number of turns in the solenoid}\\
        \alignedbox{L}{=N\frac{\Phi_B}{I}}\\
        \shortintertext{Let's find the $E_{ind}$ of the solenoid}\\
        E_{ind}=-\frac{d}{dt}\Phi_B&=-\frac{d}{dt}\left(\frac{LI}{N}\right)\\
        E_{ind}&=-\frac{L}{N}\frac{d}{dt}I
    \end{align*}
    For a self induced emf:
    \begin{align*}
        E_{ind}&=-L\frac{dI}{dt}
    \end{align*}
    \subsubsection{Example 1}
    Let's find the amount of energy stored in the magnetic field of a solenoid. First let's find the power.
    \begin{align*}
        |P|&=IV=|-LI\frac{dI}{dt}|\\
        \shortintertext{We know that power is the change of energy over the change in time, this is why we are able to jump to the next step.}\\
        U&=\int Pdt\\
        &=\int LI\frac{dI}{dt}dt\\
        &=\int LIdI\\
        \alignedbox{U_B}{=\frac{1}{2}LI}\\
        \shortintertext{Recall from electricity, $U_E=\frac{1}{2}CV^2$. In this case the charges are moving. With the magnetic energy equation, we are changing the energy for already moving charges.}\\
    \end{align*}


    \subsection{List of Maxwell's Equations:}
    \begin{align*}
        \int\vec{E}\cdot d\vec{a}=\frac{Q_{enc}}{\epsilon_0}\\
        \oint\vec{B}\cdot d\vec{l}=\mu_0I_{th}\\
        \text{also }\oint\vec{B}\cdot d\vec{a}=0\\
        \shortintertext{There are no magnetic charges/monopoles. Magnetic fields never terminate.}\\
        \shortintertext{We have recently used Farade's law}\\
        E_{ind}=-\frac{d}{dt}\Phi_B\\
        \oint\vec{E}\cdot d\vec{l}=-\frac{d}{dt}\int\vec{B}\cdot d\vec{a}\\
        \shortintertext{We will update Ampere's law to be:}\\
        \oint\vec{B}\cdot d\vec{l}=\mu_0I_{th}+\mu_0I_{disp}\\
        \text{or }\oint\vec{B}\cdot d\vec{l}=\mu_0\int\vec{J}\cdot d\vec{a}+\mu_0\epsilon_0\frac{d}{dt}\Phi_E\\
        \shortintertext{Where $I_{disp}$ and $\epsilon_0\frac{d}{dt}\Phi_E$ are the displacements of the currents}
    \end{align*}
    Example with the new Ampere's Law:
    Current changing with a capacitor (figure 6.13). We are allowing a bubble to encapsulate the leftmost plate where there is no loop. There must be a changing electric flux to resolve this issue.
    \begin{align*}
        \text{Gauss's Law:}\\
        \oint\vec{E}\cdot d\vec{a}=\frac{Q_{enc}}{\epsilon_0}\\
        \oint\vec{B}\cdot d\vec{a}=0\\
        \text{Farade's Law:}\\
        \oint\vec{E}\cdot d\vec{l}=-\frac{d}{dt}\int\vec{B}\cdot d\vec{a}\\
        \text{Ampere-Maxwell:}\\
        \oint\vec{B}\cdot d\vec{l}=\mu_0\int\vec{J}\cdot d\vec{a}+\mu_0\epsilon_0\frac{d}{dt}\int\vec{E}\cdot d\vec{a}\\
        \text{Lorenty Force Law:}\\
        \vec{F}=q\vec{E}+q\vec{v}\times\vec{B}
    \end{align*}
    Here's the differential form of all of these equations:
    \begin{align*}
        \vec{\nabla}&=\frac{\partial}{\partial x}\hat{i}+\frac{\partial}{\partial y}\hat{j}+\frac{\partial}{\partial z}\hat{k}\\
        \vec{\nabla}\cdot\vec{E}&=\frac{\rho}{\epsilon_0}\\
        \vec{\nabla}\cdot\vec{B}&=0\\
        \vec{\nabla}\times\vec{E}&=-\frac{\partial}{\partial t}\vec{B}\\
        \vec{\nabla}\times\vec{B}&=\mu_0\vec{J}+\mu_0\epsilon_0\frac{\partial}{\partial t}\vec{E}\\
    \end{align*}
\newpage
\section{Optics}
    \subsection{Traveling Waves}
    Let's first just consider a sine wave. We see that $y=sin(x)$. Let's say that the y-axis is the vertical displacement vertically, and x-axis is the horizontal displacement. The amplitude is $y_0$ and $-y_0$. Our original equation needs a few fixes. 
    \begin{align*}
        y&=y_0sin(x)\\
        \shortintertext{Because you cannot take the sin of something in meters, we need to change this to be an angle. Pme full oscilation coresponds to $2\pi rad$ or $\lambda$ distance}\\
        y&=y_0sin\left(2\pi\frac{x}{\lambda}\right)\\
        \shortintertext{Sometimes we use the parameter called the wave number:}\\
        k&=\frac{2\pi}{\lambda}\\
        y&=y_0sin(kx)
    \end{align*}
    What if also the wave moves to the right at speed $v$ (figure 6.16)? Let's take advantage of inertial frames. Let the prime axis ($x'$ and $y'$) travel \underline{with} the wave. In the prime frame, $v'=0$. We can write $y'=y_0'sin(kx')$. Now we want to get $y$ based on $y'$. We can see that $y_0=y'_0$. What about $x$ and $x'$?
    \begin{align*}
        \Delta x&=v\Delta t=v(t-0)\\
        x-x'&=vt\\
        \shortintertext{Both sides of this equation provide a positive displacement. And now to get $y$ from $y'$, we will sub out x' for x.}
        y'&=y_0sin(kx')\\
        &=y_0sin(kx')\\
        y&=y_0sin(k(x-vt))\\
        y&=y_0sin(kx-kvt)\\
        \shortintertext{Let's look at $kv$}\\
        \to kv&=\frac{2\pi}{\lambda}v\\
        &=\frac{2\pi}{\lambda}\frac{\lambda}{T}\\
        &=\frac{2\pi}{T}\\
        y&=y_0sin\left(\frac{2\pi}{\lambda}x-\frac{2\pi}{T}t\right)\\
        \shortintertext{Notice that $\frac{2\pi}{T}=2\pi f=\omega$ Where $\omega$ is the angular frequency.}\\
        y&=y_0sin(kx-\omega t)\\
        \shortintertext{Recall from 1D kinematics: $x=x_0+v_0t+\frac{1}{2}a_xt^2$}
        \shortintertext{We will add a phase constant $\Phi_0$}\\
        \to y(x,t)&=y_0sin(kx-wt+\phi_0)
        \shortintertext{The total phase is:}\\
        \phi&=kx-\omega t+\phi_0\\
    \end{align*}
    \subsubsection{Example 1}
    The following $\vec{E}$ and $\vec{B}$ fields satisfy this set of four equations (figure 7.1)
    \begin{align*}
        \vec{E}&=E_0sin(kz-\omega t)\hat{i}\\
        \vec{B}1&=B_0sin(kz-\omega t)\hat{j}\\
        \shortintertext{The pointing vector tells us the direction and magnitude of this combined electromagnetic wave. We must cross $\vec{B}$ and $\vec{E}$ in order to get the correct direction. In this case we are going to to $\vec{E}\times\vec{B}$}\\
        \vec{S}&\approx\vec{E}\times\vec{B}\\
        \shortintertext{In order to make this an equality we must add $\frac{1}{\mu_0}$.}\\
        \alignedbox{\vec{S}}{=\frac{1}{\mu_0}\vec{E}\times\vec{B}}\\
        \shortintertext{This means that electric fields do not need a median in order to move. Now let's figure out how fast this electromagnetic wave moves. We recently saw in class that $kv=\omega$. For this case we do find that $v=\frac{\omega}{k}=\frac{1}{\sqrt{\mu_0\epsilon_0}}$. It was immediately noticed that light traveled at exactly this value. This is denoted as:}\\
        c&=\frac{1}{\sqrt{\mu_0\epsilon_0}}\approx3\times10^8\frac{m}{s}\\
        \shortintertext{It turns out that light is an electromagnetic wave! Optics is the applied study of electromagnetic waves.}
    \end{align*}


    \subsection{Electromagnetic Spectrum}
    For light,
    \begin{align*}
        c=\frac{\omega}{k}=\frac{2\pi f}{\frac{2\pi}{\lambda}}\\
        c=\lambda f\\
        \shortintertext{Notice that we can change the wavelength or the frequency but the product must result in c. This gives us a spectrum or wavelengths we can choose.}
    \end{align*}
    In terms of wavelength, humans can see from violet $(\sim400nm)$ up to red $(\sim600nm)$.  In terms of frequency, red is about $1.21\times10^34$ and violet is around $7.86\times10^31$


    \subsection{Creating EM waves}
    Figure 7.2. I flip the switch up and change the capacitor to $Q_0=cv_0$. Then I flip the switch to exclude the battery. The new circuit just excludes the battery. The charges are going to flow cw through the inductor, which rejects the current. The charges are going to equilibriate and the current is going to weaken. This is an oscilating current.
    \begin{align*}
        \omega&=\frac{1}{\sqrt{lc}}\\
        f&=\frac{1}{2\pi}\frac{1}{\sqrt{lc}}\\
    \end{align*}

    
    \subsection{Pointing Vectors}
    Recall that was saw that we can make an EM wave with an LC oscillator circuit. The energy stored in an electric field in a capacitor is:
    \begin{equation*}   
        U_E=\frac{1}{2}CV^2
    \end{equation*}
    The energy density can be written more generally as, where $u_E$ is the energy density and $U_B$ is the potential energy:
    \begin{equation*}
        u_E=\frac{U_E}{volume}=\frac{1}{2}\epsilon_0E^2
    \end{equation*}
    The energy stored in a magnetic field in an inductor is:
    \begin{equation*}
        U_B=\frac{1}{2}LI^2
    \end{equation*}
    The energy density can be writted more generally as:
    \begin{equation*}
        u_B=\frac{U_B}{volume}=\frac{1}{2}\frac{1}{\mu_0}B^2
    \end{equation*}
    Consider figure 7.1 again. On the EM wave, the energy density is:
    \begin{align*}
        u&=u_E+u_B\\
        u&=\frac{1}{2}\epsilon_0E^2+\frac{1}{2}\frac{1}{\mu_0}B^2\\
        u&=\frac{1}{2}\epsilon_0(cB)^2+\frac{1}{2}\frac{1}{\mu_0}B^2\\
        \shortintertext{Recall that $c=\frac{1}{\sqrt{\epsilon_0\mu_0}}$}\\
        u&=\frac{1}{2}\epsilon_0\frac{1}{\epsilon_0}B^2+\frac{1}{2}\frac{1}{\mu_0}B^2\\
        u&=\frac{1}{\mu_0}B^2\\
        \shortintertext{Equivalently, because $B=\frac{E}{c}$:}\\
        \to u&=\frac{1}{\mu_0}\frac{E^2}{c^2}\\
        &=\frac{1}{\mu_0}\epsilon_0\mu_0E^2\\
        \alignedbox{u}{=\epsilon_0E^2}\\
        &\text{Energy density in EM wave}
    \end{align*}
    An electromagnetic wave transports energy in its EM fields. This energy transport is called the pointing vector $\vec{S}$.
    \begin{align*}
        S&=\frac{energyfluxthroughareaintime\Delta t}{A\Delta t}\\
        &=\frac{1}{\mu_0}EB\\
        \alignedbox{\text{In Vector Form, }\vec{s}}{=\frac{1}{\mu_0}\vec{E}\times\vec{B}}
    \end{align*}
    $u(z,t)$ is the wave's energy density, so if i want the energy transport in time, we must multiply it by the volume to get energy.
    \begin{align*}
        u*volume&=u*A\\
        S&=\frac{uAc\Delta t}{A\Delta t}=uc\\
        &=\epsilon_0E^2c=\epsilon_0(cE)(E)
    \end{align*}
    When we see light, our eye is detecting the EM waves in the pointing vector. Our brain arranges out the fast oscillations. So essentially, the intensity of light we see is the time-average pointing vector. Intensity is script i:
    \begin{align*}
        i&=S_{avg}=\left(\frac{1}{\mu_0}EB\right)_{avg}\\
        &=\left(\frac{1}{\mu_0}E\frac{E}{c}\right)_{avg}\\
        &=\frac{1}{\mu_0c}(E^2)_{avg}\\
        &\neq\frac{1}{\mu_0}(E_{avg})^2\\
        E&=E_0sin(kz-\omega t)\\
        E_{avg}&=0\\
        (E^2)_{avg}&=E_0^2(sin^2(kz-\omega t))_{avg}=\frac{1}{2}E_0^2\\
        i&=\frac{1}{\mu_0c}\frac{E_0^2}{2}\\
        \shortintertext{We can also write:}\\
        \sqrt{(E^2)}_{avg}&=E_{rms}\\
        E_{rms}&=\frac{E_0}{\sqrt{2}}
    \end{align*}
    More on intensity. The general definition of intensity is:
    \begin{equation*}
        \mathscr{I}=\frac{P}{A}
    \end{equation*}
    Light has no mass, and there is no mass in electromagnetic fields. Although it doesn't have math, it had momentum and pressure. Radiation pressure $(P_{rad}$):
    % First PRAD below is power radiation, after \frac{F_{rad}c}{A} P_rad changes to rad pressure
    \begin{align*}
        P_{rad}&=\frac{dW}{d}=\frac{\vec{F_rad}\cdot d\vec{z}}{dt}=\frac{F_{rad}dz}{dt}\\
        P_{rad}&=F_{rad}c\\
        \mathscr{I}&=\frac{P}{A}\\
        &=\frac{F_{rad}c}{A}\\
        &=\mathscr{P}_{rad}c\\
        \to P_{rad}&=\frac{\mathscr{I}}{c}\\
    \end{align*}
    Note: $F=ma$ does not apply here, but Newton's second law, $\vec{F}_{net}=\frac{d}{dt}\vec{p}$. For a massive object,
    \begin{align*}
        \vec{p}&=m\vec{v}\\
        \vec{F}_{net}&=\frac{d}{dt}(m\vec{v})\\
        &=m\frac{d}{dt}\vec{v}\\
        &=m\vec{a}\\
    \end{align*}
    Here, we do not have mass in the EM wave. We cannot use $\vec{F}=m\vec{a}$ in any calculations.\newline\newline
    Let's consider an EM wave that is incident on a surface. How much force will be exerted on the surface when the wave hits it? This depends on whether the wave is absorbed or reflected.
    \begin{align*}
        \vec{F}&=\frac{\Delta \vec{p}}{\Delta t}\\
        \shortintertext{If the wave is reflected, how must $\Delta\vec{p}$ look? Let's call the initial momentum $\vec{p}_0$.}\\
        \Delta\vec{p}&=\vec{p}_+-\vec{p}_0\\
        &=-2\vec{p}_0\\
        \shortintertext{By contrast, during an absorbtion,}
        \Delta\vec{p}&=\vec{p}_f-\vec{p}_i\\
        \Delta\vec{p}&=-\vec{p}_0\\
    \end{align*}
    With this, consider:
    \begin{align*}
        \mathscr{P}_{rad}&=\frac{\mathscr{I}}{c}\\
        \frac{F_{rad}}{A}&=\frac{\mathscr{I}}{c}\\
        \alignedbox{F_{rad}}{=\frac{\mathscr{I}A}{c}=\frac{S_{avg}A}{c}\text{ Absorbtion}}\\
        \shortintertext{For reflection we can just update the derivation:}\\
        \alignedbox{F_{rad}}{=2\frac{\mathscr{I}A}{c}=2\frac{S_{avg}A}{c}\text{Reflection}}\\
    \end{align*}
    If you would like to use EM waves to exert a force on something, it is twice as efficient to have the wave reflect instead of absorb. For example, a solar sail. Consider a satalite in space that has a solar sail that is being pushed by the electromagnetic waves of the sun. The sunlight will exert a force with reflection. Keep in mind that $A$ is the sail's cross sectional area.
    \begin{align*}
        F_{rad}&=2\frac{\mathscr{I}A}{c}\\
        \shortintertext{This is exerted on the sail}\\
        2\frac{\mathscr{I}A}{c}&=ma\\
        a&=\frac{2\mathscr{I}A}{mc}\\
    \end{align*}


    \subsection{Polarization of Light}
    With a normal electromagnetic wave, $\vec{E}=E\hat{i}=E_0sin(kz-\omega t+\mathscr{P}_0)\vec{i}$ and $\vec{B}=B\hat{j}=B_0sin(kz-\omega t+\mathscr{P}_0)\hat{j}$. When such a wave $\left(\vec{S}=\frac{1}{\mu_0}\vec{E}\times\vec{B}\right)$ reaches a surface, that surface can polarize the light. Light from the sun and lightbulds are unpolarized. A \underline{polaroid} is a thin material that polarizes light. It does so by building long parallel chains of molecules. When sunlight reflects of the road, it is largely polarized in plane with the road. So windshields are polarized vertically to block thin horizontal glare.
    \subsubsection{Example 1}
    This is an intensity example. A $18W$ light bulb is 1m away from a tennis ball (diameter of 12cm). How much energy has the tennis ball absorbed. Assume that the ball absorbs 70 percent of incident energy (figure 7.3). Let's assume the light bulb is isotropic (the same in all directions). Let's assume theres no reflection off the table for simplicity. How do we determine how much light goes towards the ball? First lets determine which variables cover which units. r is going to be the distance from the ball, and $r_{ball}$ is the radius of the ball. $P_{light} = 18W$, $r=1m$, $r_{ball}=6cm=0.06m$, $t=1hr=3600s$. More generally, let's say that $P_{light}=P_{src}=18W$. First we need to determine the amount of power incident on the ball ("absorbed") is $P_{inc}=P_{src}\frac{a_{ball}}{a_{shell}}$. This is where $a_{ball}$ is the cross sectional area and $a_{shell}$ is the surface area of the sphere.
    \begin{align*}
        P_{inc}&=P_{src}\frac{a_{ball}}{a_{shell}}\\
        &=P_{src}\frac{\pi r_{ball}^2}{4\pi r^2}\\
        \alignedbox{P_{inc}}{=\frac{1}{4}P_{src}\frac{r_{ball}^2}{r^2}}\\
        \shortintertext{This is the power incident on the tennis ball, but how intense is the light at the tennis ball?}\\
        \mathscr{I}_{attheball}=\frac{P_{src}}{A_{shellattheball}}=\frac{P_{src}}{4\pi r^2}\\
        \shortintertext{Notice that $P_{inc}=\mathscr{I}_{src}a_{ball}$. The energy absorbed in one hour is $\mathscr{E}=P_{inc}t$}\\
        \mathscr{E}&=\frac{0.7}{4}P_{src}\frac{r_{ball}^2}{r^2}t\\
        \mathscr{E}&=\frac{0.7}{4}(18W)\frac{(0.6m)^2}{(1m)^2}(3600s)\\
        \alignedbox{\mathscr{E}}{=40.8J}\\
    \end{align*}


    \subsection{Geometric or Ray Optics}
    Reflection: consider a bullet. If we shoot a gun such that it hits the floor, it will bounce back (reflect off the surface) at a 90 degree angle from the angle it came in at. This is due to Newton's second law, every action must have an equal and opposite reaction (figure 7.4).
    \begin{align*}
        \vec{F}&=m\vec{a}\\
        \vec{F}&=\frac{\Delta\vec{p}}{\Delta t}\\
    \end{align*}
    The law of reflection states that $\theta_1=\theta_1'$. Let's now consider light incident on a still surface of water (figure 7.5). The law of refraction is known as snell's law:
    \begin{align*}
        n_1sin\theta_1&=n_2sin\theta_2\\
        \shortintertext{n is the index of refraction. The value of n depends on the medium}\\
        n&=\frac{c}{v}\\
        \shortintertext{For gasses a good approximation for n is 1. This also applies in a vaccum}\\
        n_{water}&=1.333\\
        n_{diamonds}&=2.42\\
    \end{align*}
    The incident angle at which light will totally internally reflect is called the critical angle $(\theta_c)$. Now we are going to figure out what $\theta_c$ is.
    \begin{align*}
        n_1sin\theta_c&=n_2sin90^o\\
        \theta_c&=arcsin\frac{n_2}{n_1}\\
        \shortintertext{For diamond to air,}\\
        \theta_c&=arcsin\frac{1}{2.42}\\
        &=24.4\si{\degree}\\
    \end{align*}
    A great application of this idea is fiber optic cables. For wavelets in Refraction we will wait some time $(\Delta t)$ for the wavelength to travel into the water. Keep in mind that light travels faster in air than in water.


    \subsection{Thin Lens Refraction}
    Look at figure 7.7. Here are the rules for ray diagrams (thin lenses):
    \begin{itemize}
        \item A raythrough the center of the lenz from the object goes straight through the lens.
        \item Parallel rays go through the focal point.
    \end{itemize}
    The crossing point from any two rays gives the image location. Real images are inverted. The thin lens equation is the following:
    \begin{equation*}
        \frac{1}{f}=\frac{1}{d_0}+\frac{1}{d_i}
    \end{equation*}
    For this problem, let's find $d_i$.
    \begin{align*}
        \frac{1}{f}-\frac{1}{d_0}&=\frac{1}{d_i}\\
        d_i&=\left(\frac{1}{f}-\frac{1}{d_0}\right)^{-1}\\
        &=\left(\frac{1}{f}+\frac{1}{-d_0}\right)^{-1}\\
        &=\frac{f(-d_0)}{f+(-d_0)}\\
        d_i&=\frac{fd_0}{d_0-f}\\
        &=\frac{(15cm)(46cm)}{46cm-15cm}=22.3cm\\
    \end{align*}
    With my ray diagram, I get 21.5cm, which is pretty close to 22.3cm
    \subsubsection{Example 1}
    Here $f=15cm$ and $d_0=10cm$. This is figure 7.8. This gives is a virtual image. Virtual images are upright and cannot be seen/projected on a screen. Again, $\frac{1}{f}=\frac{1}{d_0}+\frac{1}{d_i}$ gives:
    \begin{align*}
        d_i&=\frac{fd_0}{d_0-f}\\
        d_i&=\frac{(15cm)(10)}{(10cm)-(15cm)}=-30cm
    \end{align*}
    By hand on the board, we got -33cm. With a pen on gridpaper, we should be within 0.5cm.
    \newline\newline
    The ratio of image height to object (with a minus sign) is called magnification.
    \begin{equation*}
        m=\frac{h_i}{h_0}=\frac{-d_i}{d_0}
    \end{equation*}
    Here we have $m=\frac{h_i}{h_0}=\frac{37cm}{11.5cm}=3.2$ also, if we calculate using $d_i$ and $d_0$, $m=-\frac{-33cm}{10cm}=3.3$
    \newline\newline
    Diverging lens ray diagram (figure 7.9). Must treat f as negative. 
    \begin{align*}
        d_i=\frac{fd_0}{d_0-f}=\frac{(-15cm)(47cm)}{47cm-(-15cm)}
    \end{align*}
    I measure -11.4cm. This matches exactly. Now to calculate the magnification:
    \begin{align*}
        m&=\frac{h_i}{h_0}=\frac{-d_i}{d_0}\\
        &=\frac{2.5cm}{10cm}\text{ or }-\frac{-11.4cm}{47cm}\\
        &= 0.25
    \end{align*}
    
    \subsection{Wave Interference}
    Let's recall from last class that we treated light as particle-like. For example, ray tracing. However at times, light acts like a wave. For context, let's consider water waves. Both waterwaves and light waves have interference. Let's think about an interferometer (figure 7.10). The interference equation is the following:
    \begin{equation*}
        differenceinpathlength=integernumberof\lambda s
    \end{equation*}
    For our laser interferometer, this equation will give constructive interference.
    \begin{equation*}
        2d_1-2d_2=m\lambda
    \end{equation*}
    Where $m$ is the integer from above. Now for destructive interference:
    \begin{align*}
        2d_1-2d_2=m\lambda+\frac{1}{2}\lambda\\
        2d_1-2d_2=\left(m+\frac{1}{2}\right)\lambda
    \end{align*}
    let's discuss more of what happens at an interface. Consider again light refracting from air to water. We know that $n_{air}=1$ and $n_{water}=1.33$. $n=\frac{c}{v}$. In waves, $v=\lambda f$. If in air or a vacuum, $c=\lambda f$, but in water $v=\lambda f$. Either $\lambda$ or $f$ needs to change because of the change in medium.
    \begin{align*}
        \frac{c}{n}&=\frac{\lambda f}{n}\\
        \shortintertext{Do we want $\frac{c}{n}=\frac{\lambda}{n}f$ or $\frac{c}{n}=\lambda\frac{f}{n}$?}\\
        \shortintertext{We cannot have the second option because frequency \underline{must} be constant across the boundary. There is no way for the frequency to change as it goes through an interface. If this were the case then the frequency would increase when it comes back in contact with air.}
    \end{align*}
    \subsubsection{Example 2}
    This involes thin film interference. On a soap bubble there are many different colors that reflect off of the bubble. We are going to assume that the soapy water has $n= 1.4$. Some of the light will reflect and some of the light will refract and then reflect off the inner surface. Recall that the interference equation is $difinpathlength=integernumberof\lambda s$.
    \begin{align*}
        2t&=m\lambda\\
        \shortintertext{This is on the right track but an important detail is missing.}
    \end{align*}
    In order to determine what this important detail thats missing is, we must consider first a wave phase shift. With a heavy rope knotted with a light rope, we can see that the wave from the rope is refracted, not reflected.\newline\newline
    The following is true for light:
    \begin{itemize}
        \item From high n to low n, there is no phase change upon reflection.
        \item From low n to high n, there is a $180\si{\degree}$ or $\pi rad$ or flip in phase, or $\frac{1}{2}\lambda$
        \item For refraction/transmission, there is never a phase change.
    \end{itemize}
    Earlier, we found $2t=m\lambda$. But we must still account for phase flips for reflections.
    \begin{align*}
        difinpathlength&=integernumberof\lambda s\\
        2t+\frac{1}{2}\lambda&=m\lambda\\
        \shortintertext{Because the changeing of the color of the light occurs within the soapy bubble, we are going to use the wavelength of light in soapy water on both sides of our equation.}
        \to 2t+\frac{1}{2}\frac{\lambda_0}{n_{s.w.}}&=m\frac{\lambda_0}{n_{s.w.}}\\
        2t&=\left(m-\frac{1}{2}\right)\frac{\lambda_0}{n_{s.w.}}\text{ constructive}\\
        2t&=\left(m-\frac{1}{2}\right)\frac{\lambda_0}{n_{s.w.}}+\frac{1}{2}\frac{\lambda_0}{n_{s.w.}}\\
        2t&=m\frac{\lambda_0}{n_{s.w.}}\text{ destructive}
        \shortintertext{Both the constructive and destructive are for $m=1,2,3,4,$.}
    \end{align*}

    \subsection{Slit Interference}
    For a single slit diffraction, $\frac{a}{2}sin\theta=m\lambda$ for construction, $m=0,1,2$. For destructino, it is $asin\theta=m\lambda$, $m=0,1,2$. For double slit interference, the contructive equation is $dsin\theta=m\lambda$ for $m=0,1,2$. Destructive is $dsin\theta=\left(m+\frac{1}{2}\right)\lambda$. Notice that $sin\theta=\frac{y_m}{\sqrt{y_m^2+D^2}}$.

    \subsection{Crystallography}
    Also called Xr-ray diffraction or bragg diffration. The following is a basic crystal (figure 7.11). 
    \begin{align*}
        differenceinpathlength&=intnum\lambda\\
        \alignedbox{2dsin\theta}{=m\lambda}
    \end{align*}
    \subsection{Electron Diffraction}
    Consider a single-slit setup. When shooting electrons at the slit, we also see an interference pattern. Electrons, then must also have wave characteristics. deBroglie:
    \begin{equation*}
        \lambda_c=\frac{h}{p}
    \end{equation*}
    Where $h$ is the Planchs constant.

    \subsection{Special Relativity}
    \begin{itemize}
        \item 1860s Maxwell unifies electricity and magnetism.
        \item Light is an electromagnetic wave
        \item Waves require a medium
        \item In 1877, Nichelsen and mosley try to measure the luminous ether, the medium in which light must travel. They measured no change.
    \end{itemize}
    Einstein came up with two postulates to try to understand everythign that doesn't make sense (1905). The Principle of Relativity: The laws of physice are the same in all inertial (non-accelerating) frames. Invariance of c. Signals don't arrive instantaneously but propogate. Thus, there must be a maximum universal speed.
    \subsubsection{Example}
    Let's look at a light clock (figure 7.12). The left side of the figure is when the light clock is at rest. We can derive that $t_0=\frac{2L_0}{c}$. The right side of the figure is when the light clock is at speed. We know that the light is moving at $\sqrt{c^2 + u^2}$. Now let's put this into terms of light:
    \begin{align*}
        t&=\frac{2\sqrt{L_0^2+\left(\frac{ut}{2}\right)^2}}{\sqrt{c^2+u^2}}\\
        &=\frac{2\sqrt{\left(\frac{ct}{2}\right)^2+\left(\frac{ut}{2}\right)^2}}{\sqrt{c^2+u^2}}\\
        &=\frac{2\sqrt{\left(\frac{1}{2}\right)^2\left[(ct)^2+(ut)^2\right]}}{\sqrt{c^2+u^2}}\\
        t&=\frac{t\sqrt{c^2+u^2}}{\sqrt{c^2+u^2}}\\
        t&=t\\
        \shortintertext{However, this is \underline{NOT} reality. From a special relativity (SR) approach things are different. Light always travels at c in \underline{all} inertial reference frames. We can use the same figure as before but we much do a different analysis. Becuase the speed of light is always c, our analysis triangles from before does not work. The way the velocity vectors combine is differnet.}\\
        t&=\frac{2\sqrt{L_0^2+\left(\frac{ut}{2}\right)^2}}{c}\\
        t&=\frac{2\left(\frac{ct_0}{2}\right)^2\left(\frac{ut}{2}\right)^2}{c}\\
        (ct)^2&=(ct_0)^2+(ut)^2\\
        (ct)^2-(ut)^2&=(ct_0)^2\\
        t^2(c^2-u^2)&=(ct_0)^2\\
        t^2&=\frac{(ct_0)^2}{c^2-u^2}\\
        t&=\frac{ct_0}{\sqrt{c^2-u^2}}\\
        \alignedbox{t}{=t_0\frac{1}{\sqrt{1-\frac{u^2}{c^2}}}}
        \shortintertext{This is called Time Dilation. Time panes move slowly for moving objects.}
    \end{align*}
    Now let's say the clokc ticks at rest with a time of $t_0=1sec$. In order to make the clock tick one percent slower, that is, $t=1.01sec$, the clock must move at a speed of $42,000,000\frac{m}{s}$!
\newpage

\end{document}