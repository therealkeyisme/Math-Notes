\chapter{}

\section{}

\section{}

\section{}

	\begin{problem}
		From Page 320 number 6. If we take the 95\% upper, $n=60$, $s^2=12.5$, and $\overline{x}=18.6$ the formula we are going to be using is $\overline{x}+Z_\alpha\cdot \frac{s}{\sqrt{n} }$.

		\begin{equation}
			18.6+1.645 \frac{\sqrt{12.5}}{\sqrt{60}}\approx19.55
		\end{equation}

		Because our result was 19.55, our interval is $(-\infty, 19.55)$.
	\end{problem}

	If we were to take $\overline{x_1}-\overline{x_2}$, we would need to do 

	\begin{equation}
		\overline{x_1}-\overline{x_2}+Z_\alpha\cdot \sqrt{\frac{s_1^2}{n_1}+\frac{s_2^2}{n_2}}.
	\end{equation}

	So if we were to have the $\hat{p_1}-\hat{p_2}$ lower,

	\begin{equation}
		\hat{p_1}-\hat{p_2}-Z_\alpha\sqrt{\frac{\hat{p_1}\hat{q_2}}{n_1}+\frac{\hat{p_2}\hat{q_2}}{n_2}}.
	\end{equation}

	\begin{problem}
		Let's take a look at a problem, where 55\% of 2000 American adults surveyed said they have watched digitally streamed TV programming on some type of device. $\hat{W}$ sample size would be required for the width of a 99\% CI to be at most 0.5 irrespective of the value of $\hat{p}$. We know that the value of $\hat{p}$ is 55\%. and we need to find the $[a,b]$ interval.

		$$
		\begin{aligned}
			\left(\hat{p}+2.576\sqrt{\frac{\hat{p}\hat{q}}{n}}\right)-\left(\hat{p}-2.576\sqrt{\frac{\hat{p}\hat{q}}{n}}\right)\\
			&=2\cdot 2.576\sqrt{\frac{\hat{p}\hat{q}}{n}}\\
			&= \text{size of the CI}\\
			0.5 &> 2\cdot 2.576\sqrt{\frac{\hat{p}\hat{q}}{n}}
		\end{aligned}
		$$

		Consider the worst scenario:

		$$
		\begin{aligned}
			2\cdot 2.576\sqrt{\frac{\frac{1}{2}\frac{1}{2}}{n}}&<0.05\\
			\sqrt{\frac{\frac{1}{4}}{\sqrt{n}}}&<\frac{0.5}{2\cdot2.576}\\
			\frac{\frac{1}{2}}{\sqrt{n}}&<\frac{.05}{2*2.576}\\
			\frac{2.576}{0.5}&<\sqrt{n}\\
			n&>\left(\frac{2.576}{.05}\right)^2\\
			n&=2655
		\end{aligned}
		$$
		\end{problem}

