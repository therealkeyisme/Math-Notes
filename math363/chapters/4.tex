\chapter{Orthogonality}

\section{}

  Recall that 

  \begin{equation}
    \vec{v}\cdot\vec{w}=v_1\cdot w_1+\ldots+v_n\cdot w_n = | | \vec{v} | | | | \vec{w} | | \cos(\theta)
  ,\end{equation}

  where $\theta$ is the angle between $\vec{v}$ and $\vec{w}$. Because $\cos\left(\frac{\pi}{2}\right)=0$, we know that $\vec{v}\cdot\vec{w}=0$ if and only if $\vec{v}$ is orthogonal to $\vec{w}$. In general, given $\vec{v},\vec{w}\in\R^n$, $\vec{v}$ is orthogonal to $\vec{w}$ if and only if the angle between them is $\frac{\pi}{2}$.

  \begin{definition}
    Let $U$ be any inner product space. A basis $\vec{u},\ldots,\vec{u_h}\in V$ is orthogonal if $\vec{u_j}\cdot \vec{u_i} = 0$ whenever $i\neq j$.\newline
    In addition, if $| |u_i| |=1$ for all $i$'s, the basis is orthonormal.
  \end{definition}
  
  Here are a few examples of orthonormal and orthogonal basis's:

  \begin{enumerate}
    \item $\SmallMatrix{1\\0}\SmallMatrix{0\\1}$ is an orthonormal basis.
    \item $\SmallMatrix{2\\0}\SmallMatrix{0\\1}$ is an orthogonal basis.
    \item $\SmallMatrix{1\\0\\0}\SmallMatrix{0\\1\\0}\SmallMatrix{0\\0\\1}$ is an orthonormal basis.
  \end{enumerate}

  If $\vec{v_1},\ldots,\vec{v_n}$ is an orthogonal basis, then 

  \begin{equation}
    \frac{\vec{v_1}}{| |\vec{v_1}| |},\frac{\vec{v_2}}{| | \vec{v_2} | |},\ldots,\frac{\vec{v_n}}{| | v_n | |}
  \end{equation}
  is an orthonormal basis.

  \begin{problem}
    Consider $\vec{u_1}=\SmallMatrix{1\\2}$ and $\vec{u_2}=\SmallMatrix{2\\-1}$. We know that these two matrices are not linearly dependent because $\vec{u_1}$ is not a multiple of $\vec{u_2}$. We can see that this is an orthogonal basis.
    \begin{equation}
      | | \vec{u_1} | | = \sqrt{1^2+2^2} =\sqrt{5} 
    \end{equation}

    so we know

    \begin{equation}
      v_1=\begin{pmatrix} \frac{1}{\sqrt{5} }\\\frac{1}{\sqrt{5} } \end{pmatrix} ,v_2=\begin{pmatrix} \frac{1}{\sqrt{5} }\\\frac{1}{\sqrt{5} } \end{pmatrix} 
    \end{equation}
  \end{problem}

  \begin{proposition}
    Assume $\vec{v_1},\ldots,\vec{v_n}\in V$ with $\vec{v_i}\neq \vec{0}$ for all $i$. Assume that $<v_i,v_j\ge 0$ whenever $i\neq j$, then $\{v_1,\ldots,v_n\}$ is linearly independent.
    \begin{proof}
      Assume that $c_1\vec{v_1}+\ldots+c_n\vec{n}=0$. Let $i\in\{1,\ldots,n\}$
      \begin{align*}
        <c_1\vec{v_1}+\ldots+c_n \vec{v_n}, v_i> = <0,\vec{v_i}>\\
        c_1<v_1,v_i> + c_2<v_2,v_i>+\ldots+ c_i<v_i,v_i>+\ldots+c_n<v_i,v_n>\\
        c_i<v_i,v_i> = 0\\
        c_i = 0
      .\end{align*}
      Now we know that $v_1,\ldots,v_n$ are linearly independent.
    \end{proof}
  \end{proposition}

  \begin{corollary}
    If $dim(v)=n$ and $\vec{v_1},\ldots,\vec{v_n}$ are $n$ vectors such that $<v_i,v_j> = 0$ wherever $i\neq j$, then $\{\vec{v_1},\ldots,\vec{v_n}\}$ is a basis.
  \end{corollary}

  \begin{theorem}
    Let $\vec{u_1},\ldots,\vec{u_n}$ be an orthonormal basis for $v$. Let $\vec{v}\in V$. Then we know $\vec{v}=c_1\vec{u_1}+\ldots+c_n \vec{u_n}$. In fact, $c_i=<\vec{v},\vec{u_i}>$ and 
    \begin{equation}
      | |\vec{v} | | = \sqrt{<\vec{v},\vec{u_1},>^2+<\vec{v},\vec{u_2}^2 + <\vec{v},\vec{u_n}>^2}
    \end{equation}
  \end{theorem}

  \begin{problem}
    $\mathbb{P}^2$ polynomials of degree $\le 2$ on [0,1]. Use the $L^2$ norm.

    \begin{equation}
      <p,q> = \int_0^{1}p(x)q(x)dx
    \end{equation}

    Let $p_1=1,p_2=x-\frac{1}{2},p_3=x^2-x+\frac{1}{6}$.

    \begin{align}
      <p_1,p_2> &= \int_0^{1}x-\frac{1}{2}dx=0\\
      <p_1,p_3>&=\int_0^{1}x^2-x+\frac{1}{6}dx=0.
    \end{align}

    We have an orthogonal basis because of this. In order to check to see if it is orthonormal we must also do $<p_1,p_1>,<p_2,p_2>,<p_3,p_3>$
  \end{problem}
  INSERT NOTES FROM 03.08 HERE

  So if we have the basis $\{\SmallMatrix{1\\2},\SmallMatrix{2\\1}\}, v_1=\SmallMatrix{1\\2}$. $v_2$ is equal to 

  \begin{align*}
    \begin{pmatrix} 2\\1 \end{pmatrix} -\frac{\Big<\begin{pmatrix} 2\\1 \end{pmatrix} , \begin{pmatrix} 1\\2 \end{pmatrix} \Big>}{| | \begin{pmatrix} 1\\2 \end{pmatrix} }\cdot \begin{pmatrix} 1\\2 \end{pmatrix} \\
    \begin{pmatrix} 2\\1 \end{pmatrix} -\frac{4}{\left(\sqrt{1^2+2^2} \right)^2 }\\
    \begin{pmatrix} 2\\1 \end{pmatrix} -\frac{4}{5}\begin{pmatrix} 1\\2 \end{pmatrix}\\ 
    \begin{pmatrix} 2\\1 \end{pmatrix} -\begin{pmatrix} \frac{4}{5}\\ \frac{8}{5} \end{pmatrix} = \begin{pmatrix} \frac{10}{5}\\\frac{5}{5} \end{pmatrix} -\begin{pmatrix} \frac{4}{5}\\\frac{8}{5} \end{pmatrix} \\
    =\begin{pmatrix} \frac{6}{3} \\ -\frac{3}{5}\end{pmatrix} 
    \SmallMatrix{2\\1}-\frac{<\SmallMatrix{2\\1}, \SmallMatrix{1\\2}>}{| | \SmallMatrix{1\\2}| | ^2}\cdot \SmallMatrix{1\\2}
    \SmallMatrix{2\\1}-\frac{4}{(\sqrt{1^2+2^2} )^2}\\
    \SmallMatrix{2\\1} - \frac{4}{5}\SmallMatrix{1\\2}\\
    \SmallMatrix{2\\1}-\SmallMatrix{\frac{4}{5}\\\frac{8}{5}}=\SmallMatrix{\frac{10}{5}\\\frac{5}{5}}-\SmallMatrix{\frac{4}{5}\\\frac{8}{5}}\\
    \SmallMatrix{\frac{6}{3}\\-\frac{3}{5}}\\
  .\end{align*}

  We can conclude that our basis is $\{\SmallMatrix{1\\2},\SmallMatrix{\frac{6}{5}\\-\frac{3}{5}}\}$. Let's now use our basis and rewrite it as
  \[
  \begin{pmatrix} 2\\3 \end{pmatrix} =c_1\begin{pmatrix} 1\\2 \end{pmatrix} +c_2\begin{pmatrix} \frac{6}{5}\\-\frac{3}{5} \end{pmatrix} 
  .\] 
  We can simplify this and solve for $c_1,c_2$

  \begin{align*}
    c_1&=<\begin{pmatrix} 2\\3 \end{pmatrix} ,\begin{pmatrix} 1\\2 \end{pmatrix}> = 2+6=8\\
    c_2&=<\begin{pmatrix} 2\\3 \end{pmatrix} ,\begin{pmatrix} \frac{6}{5}\\-\frac{3}{5} \end{pmatrix}> = \frac{12}{5}-\frac{9}{5}=\frac{3}{5}\\
  .\end{align*}
  Using Theorem 4.9 from the book, we do the following with an orthogonal basis to get its norm:

  \begin{align*}
    a_1=\frac{8}{| | v_1| |^2}=\frac{8}{(\sqrt{1^2+2^2} )^2}=\frac{8}{5}\\
    a_2=\frac{\frac{3}{5}}{| | v_2 | |^2}=\frac{\frac{3}{5}}{\left(\frac{6}{5}\right)^2+\left(-\frac{3}{5}\right)^2}\\
    =\frac{\frac{3}{5}}{\frac{36}{25}+\frac{9}{25}}=\frac{15}{25}\ldots\\
    \begin{pmatrix} 2\\3 \end{pmatrix} =a_1\begin{pmatrix} 1\\2 \end{pmatrix} +a_2\begin{pmatrix} \frac{6}{5}\\-\frac{3}{5} \end{pmatrix} \\
    \frac{8}{5}\begin{pmatrix} 1\\2 \end{pmatrix} + \frac{1}{3}\begin{pmatrix} \frac{6}{5}\\-\frac{3}{5} \end{pmatrix} \\
    =\begin{pmatrix} \frac{8}{5}\\\frac{16}{5} \end{pmatrix} +\begin{pmatrix} \frac{2}{5}\\-\frac{1}{5} \end{pmatrix} =\begin{pmatrix} 2\\3 \end{pmatrix} 
  .\end{align*}

  \begin{problem}
    Example of an orthogonal basis. Let $\mathbb{T}^{n}$ be the vector space of trigonometric polynomials.
    \[
      \mathbb{T}^{n}=\sum_{0\le j+k\le n}a_{jk}\sin^{j}(x)\cos^k(x)\\
    .\] 
    Using the $L^2$ norm:
    \[
      <f,g> = \int_{-\pi}^{\pi}f\cdot g
    .\] 
    An orthogonal basis is $\{1,\cos(x),\sin(x),\cos(2x),\sin(2x),\cos(3x),\sin(3x),\ldots\}$. If we were going to do the $L^2$ norm for any of these equations we would need to do the following
    \[
      \int_{-\pi}^{\pi}\sin(2x)\cos(4x)dx
    .\] 
    This equation is the Fourier series.
  \end{problem}



























