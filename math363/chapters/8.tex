\section{Eigenvalues and Singular Values}

\begin{definition}
  Let $A$ be an $nxn$ matrix. A scalar $\lambda$ is an eigenvalue of $A$ if 
  \[
  A\vec{v}=\lambda\vec{v}
  .\] 
  For some nonzero vector $\vec{v}\cdot\vec{v}$ is called the eigen vector corresponding to lambda.
\end{definition}
\begin{eg}
  Consider $A=\SmallMatrix{2&1\\1&2}$. Let's find the eigenvalues of this matrix.
    \begin{align*}
    0 &= det\left( \begin{pmatrix}2&1\\1&2\end{pmatrix}-\begin{pmatrix}\lambda&0\\0&\lambda\end{pmatrix}\right) \\
      &=det\begin{pmatrix} 2-\lambda&1\\1&2-\lambda \end{pmatrix} \\
      &=(2-\lambda)^2-1\\
    0&=(3-\lambda)(1-\lambda).
    \end{align*}
  Our eigenvalues are 3 and 1. Remember that $(A-\lambda I)\vec{v}=\vec{0}$. Consider $\lambda = 3$
  \begin{align*}
    \begin{pmatrix} -1&1\\1&-1 \end{pmatrix} \begin{pmatrix} x\\y \end{pmatrix} =\begin{pmatrix} 0\\0 \end{pmatrix} \\
    -x+y=0\\
    x=y\\
    \begin{pmatrix} x\\x \end{pmatrix} =x\begin{pmatrix} 1\\1 \end{pmatrix} 
  .\end{align*}
  Now consider $\lambda = 1$.
  \begin{align*}
    A(\lambda I)\vec{v}=\vec{0}\\
    (A-\lambda I)\vec{v}\\
    \begin{pmatrix} 1&1\\1&1 \end{pmatrix} \begin{pmatrix} x\\y \end{pmatrix} =\begin{pmatrix} 0\\0 \end{pmatrix} \\
    x+y=0\\
    y=-x\\
    \begin{pmatrix} x\\-x \end{pmatrix} =x\begin{pmatrix} 1\\-1 \end{pmatrix} 
  .\end{align*}
\end{eg}

\begin{eg}
  Consider $A=\SmallMatrix{1&1&2\\0&1&2\\0&0&3}\begin{pmatrix}1&1&2\\0&1&2\\0&0&3\end{pmatrix}$. For eigenvalues, $0=det(A-\lambda I)$.
  \begin{align*}
    0 = det \begin{pmatrix} 1-\lambda&1&2\\0&1-\lambda&2\\0&0&3-\lambda \end{pmatrix} \\
    0=(1-\lambda)(1-\lambda)(3-\lambda)
  .\end{align*}
  This makes our eigenvalues 1 and 3. For $\lambda=3$,
  \begin{align*}
    \begin{pmatrix} -2&1&2\\0&-2&2\\0&0&0 \end{pmatrix} \begin{pmatrix} x\\y\\z \end{pmatrix} &=\begin{pmatrix} 0\\0\\0 \end{pmatrix} \\
    -2y+2z&=0\\
    y&=0\\
    -2x+z+2z&=0\\
    -2x+3z&=0\\
    3z&=2x\\
    y&=z=\frac{2}{3}
  .\end{align*}
  This would make our eigenvector 
  \[
  \begin{pmatrix} x\\\frac{2}{3}x\\\frac{2}{3}x \end{pmatrix} =x\begin{pmatrix} 1\\\frac{2}{3}\\\frac{2}{3} \end{pmatrix} 
  .\] 
\end{eg}
\begin{definition}
  Given an eigenvalue $\lambda$ of $A$, the corresponding eigenvectors form a subspace denoted $v_\lambda$. Note that $v_\lambda=ker(A-\lambda I)$
\end{definition}
\begin{note}
  $\lambda=0$ is an eigenvalue of $A$ if and only if the $ker(A-\lambda I)=ker(A)=v_0\neq \{0\}$. This is true if and only if $A$ is singular ($det(A)=0$).
\end{note}
\begin{eg}
  Consider $\begin{pmatrix} 1&1\\1&1 \end{pmatrix} $. The determinant of 
  \[
    \begin{pmatrix} 1-\lambda&1\\1&1-\lambda \end{pmatrix} 
  \] 
  is equal to zero. We can see that when $\lambda =0$, the determinant is zero.
\end{eg}
\begin{prop}
  If $A$ is a real matrix and $\lambda+i\mu$ is an eigenvalue of $A$ with eigenvector $\vec{v}=\vec{x}+i\vec{y}$, then $\lambda-i\mu$ is an eigenvalue of $A$ with eigenvector $\vec{x}-i\vec{y}$.
\end{prop}
