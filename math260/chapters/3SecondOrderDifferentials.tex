\chapter{Second-order Differential Equations}

\section{General Theory of Homogeneous Linear Equations}
  The standard form for a general second order ordinary differential equation is

  \begin{equation}
    x''(t)+p(t)x'(t)+q(t)x(t)=f(t).
  \end{equation}
  
  For us, $p(t)$ and $q(t)$ are typically going to be constants. Also we know that if $f(t)=0$, the equation is homogenous.

  \begin{lemma}
    GGiven an equation in the form 

    \begin{equation}
      x''(t)+p(t)x'(t)+q(t)x(t)=0,
    \end{equation}

    if $x_1(t)$ and $x_2(t)$ are solutions to equation (3.2), then 

    \begin{equation}
      x(t)=C_1x_1(t)+C_2x_2(t)
    \end{equation}

    is also a solution to equation (3.1) for all $C_1,C_2$'s. Equation (3.3) is the general solution to a homogeneous ODE
  \end{lemma}

  Let's take a look at an application of lemma (3.1.1).

  \begin{problem}
    Consider the equation

    \begin{equation}
    t^2x'''+2tx''-6x'=0.
    \end{equation}

    This equation is a third order, linear and homogeneous, but not autonomous ordinary differential equation. Let's put this differential equation into standard form:

    \begin{equation}
    x''+\frac{2}{t}x''+\frac{6}{t^2}x'=0
    \end{equation}

    As long as $t_0\neq0$, we have a unique solution if given 3 initial conditions. The domain of this ordinary differential equation is $t>0,t<0$. Earlier, in section 3.1, we showed that $1,t^3,\frac{1}{t^2}$ were all solutions. Because this ordinary differential equation is linear and homogeneous, we know that 

    \begin{equation}
    x(t)=C_11+C_2t^3+C_3\frac{1}{t^2},
    \end{equation}

    is also a solution for all $C_1,C_2,C_3$. Just note, equation (3.6) is a linear combination of $1,t^3,\frac{1}{t^2}$. So, this is a \underline{general solution} as long as the functions, $1,t^3,\frac{1}{t^2}$ are all “different” from each other.
  \end{problem}

  What is "different"? Different means no one function in a set can be written as a linear combination of other functions in the set. Different means that all of the functions are linearly independent. More generally, if 

  \begin{equation}
    C_1f_1(t)+C_2f_2(t)+\dots+C_nf_n(t)=0
  \end{equation}

  \begin{theorem}
    WWe can instead test for linear independence by using the wronskian,

    \begin{equation}
      w(f_1,f_2)(t)=det\begin{vmatrix}
        f_1(t)&&f_2(t)\\f'_1(t)&&f'_2(t)\\
      \end{vmatrix}
    \end{equation}

    \begin{itemize}
      \item If $w(t)=0$ for all $t$'s, then $\{f_1,f_2\}$ are linearly dependent.
      \item If $w(t)=0$ for all $t$'s, then $\{f_1,f_2\}$ are linearly independent.
    \end{itemize}
  \end{theorem}

  Now, if we try to use the wronskian on problem 16, we do it like so:

  \begin{align}
    w(t)=
    \begin{vmatrix}
      1&&t^3&&\frac{1}{t^2}\\
      0&&3t^2&&-\frac{2}{t^2}\\
      0&&6t&&\frac{6}{t^4}\\
    \end{vmatrix}\\
    \to 1\times
    \begin{vmatrix}
      3t^2&&-\frac{2}{t^2}\\
      6t&&\frac{6}{t^4}\\
    \end{vmatrix}\\
    =\frac{18}{t^2}+\frac{12}{t^2}=\frac{30}{t^2}\neq0
  \end{align}

  So we know that $\{1,t^3,\frac{1}{t^3}\}$ are linearly independent. 

  Let's consider another problem where we need to show that the solutions are linearly dependent.

  \begin{problem}
    Let's consider the equation $y''(x)=-9y$, which has the solutions $\{\cos(3x),\sin(3x)\}$ and let's show that the two solutions are linearly independent:

  \begin{equation}
    w(x)=\begin{bmatrix} \cos(3t)&\sin(3t)\\-3\sin(3t)&3\cos(3t) \end{bmatrix}
    =3\cos^2(3x)+3\sin^2(3x)=3\neq0
  \end{equation}
  
  Which means that both solutions are valid for this ordinary differential equation. We can turn our original equation into $y''(x)+9y=0$ which is a homogeneous ordinary differential equation. Our $p(x)=0$, and because we know that $w(x)=Ce^{-\int p(x)dx}$, we know that the solution to our equation is 0.
  \end{problem}

  \begin{theorem}
    What to we do when solving linear non-homogeneous ordinary differential equations?
    \begin{enumerate}
      \item Solve the associated homogeneous ODE $\to X_h(t)$, which will have many unknown constants
      \item Find one solution to nonhomogeneous ODE $\to X_p(t)$
    \end{enumerate}

    The general solution to the original nonhomogeneous ODE is 

    \begin{equation}
      x(t)=x_h(t)+x_p(t)
    \end{equation}
  \end{theorem}

  Let's try out this new theorem on an example problem.

  \begin{problem}
    Consider the differential equation $y''-5y'-6y=4x$. We claim that 

    \begin{equation}
      y(x)=c_1e^{-x}+c_2e^{6x}-\frac{2}{3}+\frac{5}{9}
    \end{equation}

    Are $\{e^{-x},e^{6x}\}$ linearly independent? Verify both solutions to equation (3.14).

    \begin{align}
      w(x)&=\begin{bmatrix} e^{-x}&e^{6x}\\-e^{-x}&6e^{6x} \end{bmatrix} \\
          &=6e^{5x}+e^{5x}=7e^{5x}
    \end{align}

    Now we need to verify that this is a solution:

    \begin{align}
      y'(x)&=-C_1e^{-x}+6C_2e^{6x}-\frac{2}{3}\\
      y''(x)&=C_1e^{-x}+36C_2e^{6x}
    \end{align}

     Let's see if the left hand side is equal to $4x$ like it was above.

     \begin{align*}
       y''-5y'-6y=\left[C_1e^{-x}+36C_2e^{6x}\right]&-5\left[-C_1e^{-x}+6C_2e^{6x}-\frac{2}{3}\right]\\
                                                    &-6\left[C_1e^{-x}+C_2e^{6x}-\frac{2}{3}+\frac{5}{9}\right]
     \end{align*}
     \begin{align*}
       &=C_1[e^{-x}+te^{-x}-6e^{-x}]+C_2[36e^{6x}-30e^{6x}-6e^{6x}]+\left[\frac{10}{3}+4x-\frac{10}{3}\right]\\
       &=0+0+4x
     .\end{align*}

  \end{problem}

\section{Homogeneous Linear Equations with Constant Coefficients}

  Homogeneous ordinary differential equations are ones that come in the form 
  \begin{equation}
    ay''+by'+cy=0
  \end{equation}

  The first method of solving homogeneous linear equations with constant coefficients is the method of lucky guess. In this method, we let $y(x)=e^{rx}$. If we take the multiple derivatives of $e^{rx}$, we can find what $y'(x)$ and $y''(x)$ equal. Now we can plug in the $e^{rx}$ into our equation to get 

  \begin{align*}
    ar^2e^{rx}+bre^{rx}+ce^{rx}=0\\
    e^{rx}[ar^2+br+c]=0
  .\end{align*}
  \begin{equation}
    \boxed{ar^2+br+c=0}
  \end{equation}
  
  There are 3 distinct cases for the roots of this equation.
  \begin{enumerate}
    \item 2 real distinct roots.
    \item 1 repeated root.
    \item 2 complex conjugate roots ($A+Bi$).
  \end{enumerate}

  Let's take a look at an example of this method in action.

  \begin{problem}
    Consider the equation 
    \[
    y''-5y'-6y=0
    .\] 
    We are going to let $y(x)=e^{rx}$. We can skip adding $e^{rx}$ to the beginning of each term because we can just divide that over in our minds. After going through that we get 
    \begin{align*}
      r^2-5r-6&=0\\
      (r-6)(r+1)&=0\\
      r&=6,-1
    .\end{align*}
    
    From this, we know that our two solutions are $y_1(x)=e^{6x},y_2(x)=e^{-x}$. Our general solution is 
    \begin{equation}
      \boxed{y(x)=C_1 e^{6x}+C_2e^{-x}}
    \end{equation}
    Now all we need to do is check the wronskian $W(x)$ to make sure that $\{e^{r_1x},e^{r_2x}\}$ are linearly independent.
  \end{problem}

  Let's explore a case of a real repeated root. Let's look at the equation
  \[
  y''-8y'+16=0 
  .\] 
  We can let $y(t)=e^{rt}$, to get
  \[
  r^2-8r+16=0
  .\] 
  Our roots for $r$ are 4 and 4. One of our solutions should be $y_1(t)=e^{4t}$, but because we have a repeated root, I claim that $y_2(t)=te^{4t}$. We can verify that this is a repeated root by putting it in a wronskian:
    \begin{align*}
      w(t)&=\left| 
      \begin{matrix}
        e^{4t}&te^{4t}\\
        4e^{4t}&e^{4t}+4te^{4t}
      \end{matrix}
      \right|\\
          &=e^{8t}+4te^{8t}-4te^{8t}\\
          &=e^{8t}\neq 0
    .\end{align*}
    Because the wronskian result was not equal to zero, we know that these are two linearly independent solutions. Our general solution in this case would be
    \[
      y(t)=C_1e^{4t}+C_2te^{4t}
    .\] 
    If we were to solve a 4th order ordinary differential equation, our general solution would look something like this:
    \[
      y(t)=C_1e^{2t}+C_2e^{-3t}+C_3te^{-3t}+C_4t^2e^{-3t}
    .\] 
    As you can see, the main difference between the second and third order ordinary differential equations is the amount of roots you need to take. Let's take a look at the third case, having 2 complex conjugate values for $r=\alpha\pm i\beta$.\newline\newline
    Consider the equation
    \[
    y''-2y'+5y=0
    .\] 
    Let $y(t)=e^{rt}$. Our previous equation will turn into $r^2-2r+5=0$, and we will need to use the quadratic formula for this case.

    \begin{align*}
      r&=\frac{2\pm\sqrt{4-20} }{2}\\
       &=\frac{2\pm\sqrt{-16} }{2}\\
       &=1\pm 2i
    .\end{align*}

    Because of Euler's identity,
    \[
      e^{i\theta}=\cos(\theta)+i\sin(\theta)
    ,\]
    we can determine that the general solution of this ordinary differential equation is
    \[
    y(t)=C_1e^{1t}\cos(2t)+C_2e^{1t}\sin(2t)
    .\] 
    \begin{problem}
      Solve the homogeneous linear ordinary differential equation,
      \[
      x'''+3x''-4x'-12x=0
      .\] 
      Let $x(t)=e^{rt}$, this means that 
      \begin{align*}
        x'&=re^{rt}\\
        x''&=r^2e^{rt}\\
        x'''=r^3e^{rt}
      .\end{align*}
    Our expanded out equation is 
    \[
      e^{rt}[r^3+3r^2-4r-12]=0
    .\] 
    From this, we can see that our roots are $r=2,-2,-3$, which means that our general solution is
    \[
      x(t)=C_1e^{2t}+C_2e^{-2t}+C_3e^{-3t}
    .\] 
    \end{problem}

\section{The Spring-Mass Equation}

  The spring-mass equation is a variation of Hooke's law, which states
  \[
    F_{spring}=-kx(k>0)
  .\] 
  This tries to bring mass back to equilibrium where $x(t)$ is the distance of the mass from its equilibrium position at time $t$. The differential version of the equation is
  \[
    mx''+bx'+kx=F(t)
  .\] 
  Where $mx''$ is the acceleration, $bx'$ is the resistance friction, $kx$ is the spring force and $F(t)$ is any extra force acting on the system. Where is gravity within this equation? It drops out because the stretching of the string due to gravity puts the block attached to the string into equilibrium.\newline\newline
  When you first attach the mass and the mass stops moving,
  \[
    m(0)=-k(s)-b(0)-mg\to mg=ks\to k
  .\] 
  \begin{problem}
    A mass weighing 10 pounds stretches a string $\frac{1}{4}ft$. Assuming no dampening, find the amplitude period of oscillations of the mass if it is released from a point $\frac{1}{10}ft$below it's equilibrium position with an initial upward velocity of $\frac{1}{20}\frac{ft}{s}$.\newline
    Because the mass weighs 10 pounds, $mg=10$, which means that $m=\frac{10}{32}=\frac{5}{16}\frac{lbs^2}{ft}$. The first thing we need to do is find K. This can be done by setting $mg=ks$ and solving for $k$.
    \begin{align*}
      10&=\frac{1}{4}k\\
      k=\frac{10}{\frac{1}{4}}=40\\
      [k]=\frac{10lb}{\frac{1}{ft}}=\left[\frac{lb}{ft}\right]
    .\end{align*}
    Our ordinary differential equation is going to be 
    \[
    \frac{5}{16}x''+40x=0
    .\] 
  \end{problem}
\section{Non homogeneous Linear Equations}

\section{The Forced Spring-Mass System}
