\newpage
\section{Systems of Ordinary Differential Equations}
\subsection{}
Systems of ordinary differential equations have 1 dependent variable and 1 independent variable. Let's look at an example.

\begin{eg}
  Here is a system of 3 first order ordinary differential equations of dimension 3 
  \begin{align*}
    x'&=4x-7y+z^2\\
    y'&=2t-e^{t}y-z\\
    z'&=z^2+y^2-yx
  .\end{align*}
\end{eg}

We can also turn higher order ODE's into systems of first-order ODEs.
\begin{eg}
  Consider the equation $x''+7x''+7tx'+6x=t^2$. If we let $y=x',z=x''=y',x'''=z'$, then using the ODE,
  \[
  z'=t^2-6x+7ty-7z
  ,\]
  we can make a system of nonlinear, nonautonomous, nonhomogeneous ordinary differential equations.\footnote{After this system we went to chapter 2.6}
  \begin{align*}
    x'&=y\\
    y'&=z\\
    z'&=t^2-6x+ty-7z
  .\end{align*}
\subsection{Matrix Algebra}
A matrix is a rectangular array of numbers. 
\begin{eg}
  Consider $A=\begin{bmatrix} 2&3&1\\4&5&0 \end{bmatrix} $. $A$ has 2 rows and 3 columns, so its size is $2\times 3$. The $a_{ij}$ is an entry in $A$'s $i^{th}$ row and $j^{th}$ column.
\end{eg}
\begin{eg}
  Consider $B=\begin{bmatrix} 5&0&0\\0&-1&0\\0&0&3 \end{bmatrix} $, which is a $3\times 3$ matrix. This is also a square matrix, due to the size. $B$ is also a diagonal matrix, because it's only nonzero entries are on the main diagonals.
\end{eg}
An upper triangular matrix looks like 
\[
  \begin{bmatrix} 5&2&0\\0&-1&3\\0&0&3 \end{bmatrix} 
.\] 
A lower triangular matrix looks like
\[
  \begin{bmatrix} 5&0&0\\ \pi&-1&0\\-1&-\frac{1}{2}&3 \end{bmatrix} 
.\] 
\end{eg}
We can add and subtract matrices element-wise, but only if the matrices have the same size. We can also multiply a matrix by a scalar
\[
  3\begin{bmatrix} 4&5\\6&7 \end{bmatrix} =\begin{bmatrix} 12&15\\18&21 \end{bmatrix} 
.\] 
A row vector is $\begin{bmatrix} 2&3&1 \end{bmatrix} $, in this case the size would be $1\times 3$. A column vector looks like $\begin{bmatrix} 2\\3\\1 \end{bmatrix} $ with the size $3\times 1$. The zero vector is of size $n\times n$, and has $0$ in every index in the matrix. The transpose of a matrix reverses all of the columns and rows. Consider
\[
  \begin{bmatrix} 2&3&1\\4&5&6 \end{bmatrix} ^{T}=\begin{bmatrix} 2&4\\3&5\\1&6 \end{bmatrix} 
.\] 
Here are the properties of a transpose:
\begin{itemize}
  \item $\left( A^{T} \right)^{T}=A $
  \item $\left( A+B \right) ^{T}=A^{T}+B^{T}$
  \item $(kA)^{T}=kA^{T}$
  \item $\left( AB \right) ^{T}=B^{T}A^{T}$
\end{itemize}
Now we are going to start talking about matrix multiplication. Matrix multiplication is like the dot product. The $(i,j)^{th}$ entry of $AB$ is the dot product of the $i^{th}$ row of $A$ w/ $j^{th}$ column of $B$. Consider the following
\[
  \begin{bmatrix} 1&2&3 \end{bmatrix} \begin{bmatrix} 4\\5\\6 \end{bmatrix} =(1\times 4)+(2\times 5)+(3\times 6)= 32
.\] 
In order to do matrix multiplication, rows of $A$ must match the columns of $B$ and vice versa.
