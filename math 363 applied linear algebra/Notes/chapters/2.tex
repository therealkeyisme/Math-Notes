\chapter{Vector Spaces and Bases}

\section{Real Vector Spaces}

\section{Subspaces}

\section{Span and Linear Independence}

  \subsection{Linear Independence and Dependence}
  
\section{Basis and Dimension}
  \begin{definition}
    AA basis of a vector space $v$ is a collection of vectors $\vec{v_1},\dots,\vec{v_n}$ that 1. span $v$ and 2. are linearly dependent.
  \end{definition}

  \begin{problem}
    If we are looking at $\mathbb{R}^2$, with $e_1=\left(\begin{smallmatrix}1\\0\end{smallmatrix}\right)$, and $e_2=\left(\begin{smallmatrix}0\\1\end{smallmatrix}\right)$
    We can tell that this is a basis of $\mathbb{R}^2$. We can tell this because the span of $v$ is linearly independent. We can also see this because:

    \begin{equation}
      a\times e_1+b\times e_2=\begin{pmatrix}a\\b\end{pmatrix}
    \end{equation}
  \end{problem}

  \begin{problem}
    Now we are going to look at an example in $\mathbb{R}^3$, with $e_1=\left(\begin{smallmatrix}1\\0\\0\end{smallmatrix}\right)$, $e_2=\left(\begin{smallmatrix}0\\1\\0\end{smallmatrix}\right)$, and $e_3=\left(\begin{smallmatrix}0\\0\\1\end{smallmatrix}\right)$. We can figure out that this is a basis by doing the same technique as we did before:

    \begin{align}
      c_1\vec{e_1}+c_2\vec{e_2}+c_3\vec{e_3}=0\\
      c_1=c_2=c_3=0
    \end{align}

    Because $c_1$, $c_2$, and $c_3$ are all equal to zero, $\vec{e_1}$, $\vec{e_2}$, and $\vec{e_3}$ form a basis.
  \end{problem}

  \newpage

  \begin{theorem}
    IIf a vector space $v$ has a basis with $n$ elements, then every basis of $v$ has $n$ elements. We say $v$ has dimension $n$. We write down $v=n$
  \end{theorem}

  \begin{theorem}
    IIf the dimension of $v$ is $n$, then any collection of $n+1$ or more vectors must be linearly dependent.
  \end{theorem}

  \begin{theorem}
    SSuppose $v=n$
    
    \begin{enumerate}
      \item Every collection of more than $n$ vectors is linearly dependent.
      \item No set of fewer than $n$ vectors spans $v$.
      \item A set of $n$ vectors is a basis if and only if it spans $v$.
      \item A set of $n$ vectors is a basis if and only if it is linearly dependent.
    \end{enumerate}
  \end{theorem}

  \begin{problem}
    Assume $1,x,x^2$ is a basis for $\mathbb{P}^2$. We are going to multiply $1\times5$, $x\times6$, and $x^2\times2$.

    \begin{align}
      5+6x+2x^2&\\
      c_1\times1+c_2\times x+c_3\times x^2&=0\\
      dim(\mathbb{P}^2)&=3\\
    \end{align}
  \end{problem}

  \begin{theorem}
   h$\vec{v_1},\dots\vec{v_2}$ form a basis of $v$ if and only if for all $\vec{v}\in v$, there exist unique $c_{1},\dots,c_{n}$ such that $\vec{v}=c_1\vec{v_1}+\dots c_n\vec{v_n}$
  \end{theorem}

  \begin{problem}
    Let $v=\mathbb{R}^2$. Let $\vec{v}=\left(\begin{smallmatrix}4\\3\end{smallmatrix}\right)$. We know from previous problems that $\left(\begin{smallmatrix}1\\0\end{smallmatrix}\right)\left(\begin{smallmatrix}0\\1\end{smallmatrix}\right)$ is a basis of $\mathbb{R}^2$. We can also figure out what our basis is by trying to figure out what our $c_1$ and $c_2$ values should be. Because the matrices we know are a basis consist of 1's and 0's, we can see that 

    \begin{equation}
      4
      \begin{pmatrix}
        1\\0
      \end{pmatrix}
      +
      \begin{pmatrix}
        0\\1
      \end{pmatrix}
      =
      \begin{pmatrix}
        4\\3
      \end{pmatrix}.
    \end{equation}
    The coordinates of $\vec{v}$ with respect to this basis, are $(4,3)$. Let's consider a different basis. We are now going to look at the basis,

    \begin{equation}
      \left\{
        \begin{pmatrix}
          1\\-3
        \end{pmatrix},
        \begin{pmatrix}
          2\\-1
        \end{pmatrix}
      \right\}
    \end{equation}

    Now we need to figure out the $a$ and $b$ values for this basis respectively.

    \begin{align}
      a
      \begin{pmatrix}
        1\\-3 
      \end{pmatrix}
      + b
      \begin{pmatrix}
        2\\-1
      \end{pmatrix}
      =
      \begin{pmatrix}
        4\\3
      \end{pmatrix}
      \\
    \end{align}
    The coordinates of $\vec{v}$ with respect to this basis are $(4,3)$. Let's consider the same problem but with the following basis:

    \begin{equation}
      \left\{
        \begin{pmatrix}
          1\\-3
        \end{pmatrix}
        ,
        \begin{pmatrix}
          2\\-1
        \end{pmatrix}
      \right\}
    \end{equation}
    Now, we are going to do the same thing as before, where we solve for $a$ and $b$ in the following equation

    \begin{equation}
      a
      \begin{pmatrix}
        1\\-3\
      \end{pmatrix}
      +b
      \begin{pmatrix}
        2\\-1
      \end{pmatrix}
      =
      \begin{pmatrix}
        4\\3
      \end{pmatrix}
    \end{equation}

    We can setup this to be a system of equations that we can turn into a matrix

    \begin{equation}
      \begin{cases}
        1a+2b=4\\
        -3a+(-1)b=3\\
      \end{cases}
      \to
      \begin{bmatrix}
        1&2&\vdots&4\\
        -3&-1&\vdots&3\\
      \end{bmatrix}
    \end{equation}
    And now we can use the basic row operation $R_2=R_2+3R_1$ in order to solve for $a$ and $b$:

    \begin{equation}
      \begin{bmatrix}
        1&2&\vdots&4\\
        0&5&\vdots&5
      \end{bmatrix}
    \end{equation}

    \begin{align}
      5b=15 && a+2b=4\\
      b=3 && 1+2*3=4\\
      && a=-2
    \end{align}
  \end{problem}
  
\section{The fundamental Matrix Subspaces (Kernel and Image)}

  \begin{definition}
    The image of an $m\times n$ matrix $A$ is the subspace spanned by the columns of A.
  \end{definition}