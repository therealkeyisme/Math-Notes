\chapter{Vector Spaces and Bases}

\section{Real Vector Spaces}
  A vector space is the abstract reformulation of the quintessential properties of $n$-dimensional Euclidean Space $\R^n$, which is defined as the set of all real (column) vectors with $n$ entries. The basic laws of vector addition and scalar multiplication in $\R$

  \begin{definition}
    AA vector space is a set of $V$ equipped with two operations:
    \begin{itemize}
      \item Addition: adding any pair of vectors v, w $\in V$ produces another vector v + w $\in V$;
      \item Scalar Multiplication: multiplying a vector v $\in V$ by a scalar $c\in\R$ produces a vector $c$v$\in V$
    \end{itemize}

    These are subject to the following axioms, valid for all u, v, w $\in V$ and all scalars $c,d\in\R$:
    \begin{itemize}
      \item Commutativity of Addition: v + w = w + v.
      \item Associativity of Addition: u + (v + w) = (u + v) + w.
      \item Additive Identity: There is a zero element $0\in V$ satisyfing v + 0 = v = 0 + v.
      \item Additive Inverse: For each v $\in V$ there is an element -v $\in V$ such that v+(-v)=0=(-v)+v.
      \item Distributivity: $(c+d)$v=($c$v) + ($d$v), and $c$(v+w)=($c$v)+($c$w).
      \item Assosiativity of Scalar Multiplication: $c$($d$v) = ($cd$)v.
      \item Unit for Scalar Multiplication: the scalar 1$\in\R$ satisfies 1v=v.
    \end{itemize}
  \end{definition}

  \begin{theorem}
    LLet $V$ be a vector space.
    \begin{itemize}
      \item $0\times\vec{V}=\vec{0}$
      \item $-1\vec{V}=-\vec{V}$
      \item $c\times\vec{0}=\vec{0}$
      \item If $c\times\vec{V}=\vec{0}$, then $c=0$ or $\vec{V}=\vec{0}$
    \end{itemize}
  \end{theorem}
  
  Here are some examples of vector spaces:

  \begin{itemize}
    \item $\R^n=\left\{\left.\left(\begin{smallmatrix}r_1\\r_2\\\vdots\\r_n\end{smallmatrix}\right)\right|r_1,r_2,r_n\in\R\right\}$
    \item $M_{m\times n}=$ The $m$ by $n$ matrices over $\R$.
    \item $\P^n=$ the polynomials of degree $\le n$.
  \end{itemize}

  \begin{definition}
    LLet $V$ be a vector space over $F$. $W\le V$ is a subspace of $V$ if $W$ is a vector space over $F$ under the same operation as $V$.
  \end{definition}

  An example of definition (2.1.2). Let $V=\R^3=\left\{\left.\left(\begin{smallmatrix}a\\b\\c\end{smallmatrix}\right)\right|a,b,c\in\R\right\}$. $V$ is a vector space. If we let $W=\left\{\left.\left(\begin{smallmatrix}a\\b\\0\\\end{smallmatrix}\right)\right|a,b\in\R\right\}$, then $W$ is a subspace of $V$.

  \begin{theorem}
    Let $V$ be a vector space. Let $W\le V$. $W$ is a subspace of $V$ if 
    \begin{itemize}
      \item $w\neq0$
      \item $\forall w_1w_2\in W;w_1+w_2\in W$
      \item $\forall c\in F;\vec{W}\in W;c\cdot\vec{W}\in W$
    \end{itemize}
  \end{theorem}

  If we were to let $V=\R^3$ for $\left(\begin{smallmatrix}a\\b\\c\\\end{smallmatrix}\right)$. We can determine the following:

  \begin{itemize}
    \item $\{\vec{0}\}$ is a subspace of $V$.
    \item $\left\{\left.\left(\begin{smallmatrix}a\\0\\c\end{smallmatrix}\right)\right|a,c\in\R\right\}$ is a subspace of $V$.
    \item Consider the equation $\left\{\left.\left(\begin{smallmatrix}x\\x\\0\end{smallmatrix}\right)\right|x\in\R\right\}=W$ Show that $W$ is a subspace of $V$.
    \begin{itemize}
      \item $\left(\begin{smallmatrix}0\\0\\0\end{smallmatrix}\right)\in W$ so $W\neq0$.
      \item $\left(\begin{smallmatrix}x\\x\\0\end{smallmatrix}\right),\left(\begin{smallmatrix}y\\y\\0\end{smallmatrix}\right)\in W$. Then $\left(\begin{smallmatrix}x\\x\\0\end{smallmatrix}\right)+\left(\begin{smallmatrix}y\\y\\0\end{smallmatrix}\right)=\left(\begin{smallmatrix}x+y\\x+y\\0\end{smallmatrix}\right)$
      \item $\left(\begin{smallmatrix}x\\x\\0\end{smallmatrix}\right)\in W$, then $c\times\left(\begin{smallmatrix}x\\x\\0\end{smallmatrix}\right)=\left(\begin{smallmatrix}cx\\cx\\0\end{smallmatrix}\right)\in W$.
      \item Therefore we know that $W$ is a subspace of $V$ with respect to scalar multiplication and addition.
    \end{itemize}
    \item $W=\left\{\left.\left(\begin{smallmatrix}x\\y\\2x+3y\end{smallmatrix}\right)\right|x,y\in\R\right\}$. $W$ is a subspace of $V$.
  \end{itemize}

  $\R^3$ only has 4 kinds of subspaces. $\R^3$, $\{\vec{0}\}$, planes passing through the origin and lines that are passing through the origin.

  
\section{Subspaces}

  \begin{definition}
    LLet $I$ be an interval in $\R$. Let $\mathbb{F}(I)$ be the vector space of functions $\mathbb{F}=I\to\R$.

    \begin{itemize}
      \item $\mathbb{C}^0(I)=$ the continuous functions from $I\to\R$ is a subspace.
      \item $\P^n(I)=$ polynomials of degree $\le n$ restricted to $\mathbb{F}(I)$. This is a subspace of $C^0(I)$.
      \item $\P^\infty(I)=$ all polynomials on $I$. This is a subspace of $\mathbb{F}(I)$.
      \item $\mathbb{C}^n (I)=$ the set of functions $f:I\to\R$ such that $f',f''...f^{(n)}$ all exist and are continuous.
      \item $\mathbb{C}^\infty(I)=$ functions from $I\to\R$ such that $f',f'',f'''$ all exist and are smooth functions.
      \item $A(I)=$ the functions in $\mathbb{C}^\infty(I)$ such that all $A\in I$, the power series $f(a)+\frac{f'(a)(x-a)}{1!}+\frac{f''(a)(x-a)^2}{2!}+\dots$ converges for all $x\in I$ sufficiently close to a.
    \end{itemize}
  \end{definition}

  \begin{problem}
    Show that $v=\left(\begin{smallmatrix}x\\y\\2x+y\end{smallmatrix}\right)$ is a subspace of $\R^3$.

    \begin{itemize}
      \item Because $\left(\begin{smallmatrix}0\\0\\2(0)+0\end{smallmatrix}\right)=\left(\begin{smallmatrix}0\\0\\0\end{smallmatrix}\right)$ is in $v$, $v$ is not empty.
      \item Let $\vec{v_1}=\left(\begin{smallmatrix}x_1\\y_1\\2x_1+y_1\end{smallmatrix}\right),\vec{v_2}=\left(\begin{smallmatrix}x_2\\y_2\\2x_2+y_1\end{smallmatrix}\right)\in V$
        \begin{equation}
          v_1+v_2=\begin{pmatrix}
            x_1+x_2\\
            y_1+y_2\\
            2x_1+y_1+2x_2+y_2
          \end{pmatrix}
          = \begin{pmatrix}
            x_1+x_2\\
            y_1+y_2\\
            2(x_1+x_2)+(y_1+y_2)
          \end{pmatrix}
          \in V
        \end{equation}
        so $v$ is closed with respect to addition.
      \item Let $r\in\R$ and $\left(\begin{smallmatrix}x\\y\\2x+y\end{smallmatrix}\right)=\vec{v}\in v$
    \end{itemize}
  \end{problem}

\section{Span and Linear Independence}

  If we let $V$ be a vector space over $\R$ and let $\vec{v_1},\dots,\vec{v_n}\in V$, then we can determine that the 
  \begin{equation}
    span(\{\vec{v_1},\dots,\vec{v_n}\})=\{c_1\vec{v_1}+\dots+c_n\vec{v_n}|c_1,\dots,c_n\in\R\}
  \end{equation}

  \begin{proposition}
    TThe span of $\{\vec{v_1},\vec{v_2}\}$ is a subspace of $V$.
    \begin{proof}
      \begin{align}
        c_1\vec{v_1}+\dots+c_n\vec{v_n}\\
        k_1\vec{v_1}+\dots+k_n\vec{v_n}\in span\\
      \end{align}

      If we add together both of the equations above we get

      \begin{align}
        (c_1+k_1)\vec{v_1}+\dots+(c_n+k_n)\vec{v_n}\in span.\\
        r(c_1\vec{v_1}+\dots+c_n\vec{v_n})\\
        =rc_1\vec{v_1}+\dots+rc_n\vec{v_n}\in span
      \end{align}
    \end{proof}
  \end{proposition}

  \begin{problem}
    Let $V\in\R^3$. Also we are going to let

    \begin{equation}
      \vec{v_1}=\begin{pmatrix}
        1\\2\\0
      \end{pmatrix}
      ,span(\vec{v_1})=\left\{c\left.\begin{pmatrix}1\\2\\0\end{pmatrix}\right|c\in\R\right\}
    \end{equation}
    $\left(\begin{smallmatrix}1\\2\\0\end{smallmatrix}\right)$ is a vector in 3-space. $c\times\left(\begin{smallmatrix}1\\2\\0\end{smallmatrix}\right)$ expands, contracts, changes direction. This is a line which goes through $\left(\begin{smallmatrix}0\\0\\0\end{smallmatrix}\right)$.
    \begin{equation}
      c\begin{pmatrix}
        1\\2\\0 
      \end{pmatrix}
    \end{equation}
    is in the xy-plane, let's solve for $y$ to find the equation of the line that is drawn by the vector:

    \begin{align}
      \begin{pmatrix}
        c\\2c\\0
      \end{pmatrix}\\
      x&=c\\
      y&=2c\\
      \frac{1}{2}y&=c\\
      \to x=\frac{1}{2}y\\
      \to y=2x\\
    \end{align}
    Now we are going to let $\vec{v_2}=\left(\begin{smallmatrix}1\\1\\0\end{smallmatrix}\right)$. Consider the span of $(\{\vec{v_1},\vec{v_2}\})$. The span of $(\{\vec{v_1}\vec{v_2}\})$ is a plane.
  \end{problem}

  In $\R^3$, if $\vec{0}\neq\vec{v}\in\R$, then the span$\vec{v}$ is a line.

  \begin{problem}
    Let $v=\P^2$. $v$ is the set of polynomials of degree $\le2\in\R$.

    \begin{itemize}
      \item $span(1,x,x^2)=\P^2$
      \item $span(4,2x)=\P^1$, which means all polynomials of degree $\le1$
    \end{itemize}

  \end{problem}
  \begin{definition}
    LLet $v$ be a vector space. $\vec{v_1},\dots,\vec{v_n}$ are linearly dependant if there exists $c_1,\dots,c_n$ are not all zero, such that $c_1\vec{v_1}+\dots+c_n\vec{v_n}=\vec{0}$, otherwise, $\vec{v_1},\dots,\vec{v_n}$ are linearly independent.
  \end{definition}
  
  If we let $\vec{v_1}=\SmallMatrix{1\\2},\vec{v_2}=\SmallMatrix{2\\4}$, we can do a simple test to see if they are linearly independent. We know that $2\vec{v_1}=\vec{v_2}$, which means that $2\vec{v_1}+-1\vec{v_2}=0$. Because we can make $\vec{v_1}+\vec{v_2}$ by using a simple scalar value, these functions are linearly dependent.

  \begin{problem}
    Consider the following three matrices
    \begin{equation}
      \vec{v_1}=\begin{pmatrix}
        1\\2\\1
      \end{pmatrix}
      ,\vec{v_2}=\begin{pmatrix}
        2\\1\\1\\
      \end{pmatrix}
      ,\vec{v_3}=\begin{pmatrix}
        8\\1\\11
      \end{pmatrix}
    \end{equation}
    Are these matrices linearly dependent or independent from each other? The following equation will let us set up a matrix to determine the results:

    \begin{equation}
      c_1\vec{v_1}+c_2\vec{v_2}+c_3\vec{v_3}=\vec{0}
    \end{equation}

    We must try to see if there are any $c_n$ values that are not zero to make this true.

    \begin{equation}
      \begin{pmatrix}
        1&2&8\\2&-1&1\\1&3&11
      \end{pmatrix}
      \begin{pmatrix}
        c_1\\c_2\\c_3\\
      \end{pmatrix}
      =\begin{pmatrix}
        0\\0\\0
      \end{pmatrix}
    \end{equation}
    \begin{equation}
      = \begin{pmatrix}
        1&2&8&\vdots&0\\
        2&-1&1&\vdots&0\\
        1&3&0&\vdots&0\\
      \end{pmatrix}
    \end{equation}
    By doing some elementary row operations, we can find that in row echelon form, the matrix from equation (2.21) can be written as

    \begin{equation}
      \begin{pmatrix}
        1&2&8&\vdots&0\\
        9&1&3&\vdots&0\\
        0&0&0&\vdots&0\\
      \end{pmatrix}
    \end{equation}
    And from here we can solve for the different $c_n$ values.

    \begin{align}
      1c_1+2c_2+8c_3&=0\\
      c_1+3c_3&=0\\\\
      c_3=-3c_3\\
      c_1=-2c_3\\
      c_3=c_3  
    \end{align}

    Because we have this relationship where $c_1,c_2,c_3$ all depend on each other, we can tell that this is linearly independent.
  \end{problem}

  \subsection{Linear Independence and Dependence}
  
\section{Basis and Dimension}
  \begin{definition}
    AA basis of a vector space $v$ is a collection of vectors $\vec{v_1},\dots,\vec{v_n}$ that 1. span $v$ and 2. are linearly dependent.
  \end{definition}

  \begin{problem}
    If we are looking at $\mathbb{R}^2$, with $e_1=\left(\begin{smallmatrix}1\\0\end{smallmatrix}\right)$, and $e_2=\left(\begin{smallmatrix}0\\1\end{smallmatrix}\right)$
    We can tell that this is a basis of $\mathbb{R}^2$. We can tell this because the span of $v$ is linearly independent. We can also see this because:

    \begin{equation}
      a\times e_1+b\times e_2=\begin{pmatrix}a\\b\end{pmatrix}
    \end{equation}
  \end{problem}

  \begin{problem}
    Now we are going to look at an example in $\mathbb{R}^3$, with $e_1=\left(\begin{smallmatrix}1\\0\\0\end{smallmatrix}\right)$, $e_2=\left(\begin{smallmatrix}0\\1\\0\end{smallmatrix}\right)$, and $e_3=\left(\begin{smallmatrix}0\\0\\1\end{smallmatrix}\right)$. We can figure out that this is a basis by doing the same technique as we did before:

    \begin{align}
      c_1\vec{e_1}+c_2\vec{e_2}+c_3\vec{e_3}=0\\
      c_1=c_2=c_3=0
    \end{align}

    Because $c_1$, $c_2$, and $c_3$ are all equal to zero, $\vec{e_1}$, $\vec{e_2}$, and $\vec{e_3}$ form a basis.
  \end{problem}

  \newpage

  \begin{theorem}
    IIf a vector space $v$ has a basis with $n$ elements, then every basis of $v$ has $n$ elements. We say $v$ has dimension $n$. We write down $v=n$
  \end{theorem}

  \begin{theorem}
    IIf the dimension of $v$ is $n$, then any collection of $n+1$ or more vectors must be linearly dependent.
  \end{theorem}

  \begin{theorem}
    SSuppose $v=n$
    
    \begin{enumerate}
      \item Every collection of more than $n$ vectors is linearly dependent.
      \item No set of fewer than $n$ vectors spans $v$.
      \item A set of $n$ vectors is a basis if and only if it spans $v$.
      \item A set of $n$ vectors is a basis if and only if it is linearly dependent.
    \end{enumerate}
  \end{theorem}

  \begin{problem}
    Assume $1,x,x^2$ is a basis for $\mathbb{P}^2$. We are going to multiply $1\times5$, $x\times6$, and $x^2\times2$.

    \begin{align}
      5+6x+2x^2&\\
      c_1\times1+c_2\times x+c_3\times x^2&=0\\
      dim(\mathbb{P}^2)&=3\\
    \end{align}
  \end{problem}

  \begin{theorem}
   h$\vec{v_1},\dots\vec{v_2}$ form a basis of $v$ if and only if for all $\vec{v}\in v$, there exist unique $c_{1},\dots,c_{n}$ such that $\vec{v}=c_1\vec{v_1}+\dots c_n\vec{v_n}$
  \end{theorem}

  \begin{problem}
    Let $v=\mathbb{R}^2$. Let $\vec{v}=\left(\begin{smallmatrix}4\\3\end{smallmatrix}\right)$. We know from previous problems that $\left(\begin{smallmatrix}1\\0\end{smallmatrix}\right)\left(\begin{smallmatrix}0\\1\end{smallmatrix}\right)$ is a basis of $\mathbb{R}^2$. We can also figure out what our basis is by trying to figure out what our $c_1$ and $c_2$ values should be. Because the matrices we know are a basis consist of 1's and 0's, we can see that 

    \begin{equation}
      4
      \begin{pmatrix}
        1\\0
      \end{pmatrix}
      +
      \begin{pmatrix}
        0\\1
      \end{pmatrix}
      =
      \begin{pmatrix}
        4\\3
      \end{pmatrix}.
    \end{equation}
    The coordinates of $\vec{v}$ with respect to this basis, are $(4,3)$. Let's consider a different basis. We are now going to look at the basis,

    \begin{equation}
      \left\{
        \begin{pmatrix}
          1\\-3
        \end{pmatrix},
        \begin{pmatrix}
          2\\-1
        \end{pmatrix}
      \right\}
    \end{equation}

    Now we need to figure out the $a$ and $b$ values for this basis respectively.

    \begin{align}
      a
      \begin{pmatrix}
        1\\-3 
      \end{pmatrix}
      + b
      \begin{pmatrix}
        2\\-1
      \end{pmatrix}
      =
      \begin{pmatrix}
        4\\3
      \end{pmatrix}
      \\
    \end{align}
    The coordinates of $\vec{v}$ with respect to this basis are $(4,3)$. Let's consider the same problem but with the following basis:

    \begin{equation}
      \left\{
        \begin{pmatrix}
          1\\-3
        \end{pmatrix}
        ,
        \begin{pmatrix}
          2\\-1
        \end{pmatrix}
      \right\}
    \end{equation}
    Now, we are going to do the same thing as before, where we solve for $a$ and $b$ in the following equation

    \begin{equation}
      a
      \begin{pmatrix}
        1\\-3\
      \end{pmatrix}
      +b
      \begin{pmatrix}
        2\\-1
      \end{pmatrix}
      =
      \begin{pmatrix}
        4\\3
      \end{pmatrix}
    \end{equation}

    We can setup this to be a system of equations that we can turn into a matrix

    \begin{equation}
      \begin{cases}
        1a+2b=4\\
        -3a+(-1)b=3\\
      \end{cases}
      \to
      \begin{bmatrix}
        1&2&\vdots&4\\
        -3&-1&\vdots&3\\
      \end{bmatrix}
    \end{equation}
    And now we can use the basic row operation $R_2=R_2+3R_1$ in order to solve for $a$ and $b$:

    \begin{equation}
      \begin{bmatrix}
        1&2&\vdots&4\\
        0&5&\vdots&5
      \end{bmatrix}
    \end{equation}

    \begin{align}
      5b=15 && a+2b=4\\
      b=3 && 1+2*3=4\\
      && a=-2
    \end{align}
  \end{problem}
  
\section{The fundamental Matrix Subspaces (Kernel and Image)}

  \begin{definition}
    TThe image of an $m\times n$ matrix $A$ is the subspace spanned by the columns of A.
  \end{definition}

  \begin{problem}

    Let's consider the following equation. 

    \begin{equation}
      \begin{bmatrix}
        1&2&3\\4&5&6\\
      \end{bmatrix}
      \begin{bmatrix}
        r_1\\r_2\\r_3
      \end{bmatrix}
      =
      \begin{bmatrix}
        x\\y
      \end{bmatrix}
    \end{equation}

    When we multiply our the matrix, we see that the span of the columns give us all the possible $\left[\begin{smallmatrix}x\\y\end{smallmatrix}\right]$ values. The span of the matrix 

    \begin{equation}
      \begin{bmatrix}
        1\times r_1&2\times r_2&3\times r_3\\
        4\times r_1&5\times r_2&6\times r_3\\
      \end{bmatrix}
    \end{equation}
    would be the values $\left[\begin{smallmatrix}1\\4\end{smallmatrix}\right]$, $\left[\begin{smallmatrix}2\\5\end{smallmatrix}\right]$, and $\left[\begin{smallmatrix}3\\6\end{smallmatrix}\right]$
  \end{problem}

  \begin{definition}
    AA space, $A$, is an $m\times m$ matrix, The kernel of $A$ is 
    \begin{align}
      A&=Ker(A)\\
      &=\{\vec{x}|A\vec{x}=\vec{0}\}
    \end{align}
  \end{definition}

  Using definition (2.5.2), if $A=\left[\begin{smallmatrix}1&0\\5&0\\\end{smallmatrix}\right]$, then 

  \begin{equation}
    \vec{x_1}=
    \begin{bmatrix}
      0\\0
    \end{bmatrix}
    , 
    \vec{x_2}=
    \begin{bmatrix}
      0\\5
    \end{bmatrix}.
  \end{equation}

  Something to keep in mind: If $\vec{x_1},\vec{x_2}\in Ker(A)$, then $r_1\vec{x_1}+r_2\vec{x_2}\in Ker(A)$. So the kernel of $A$ is a subspace of the domain of the function.

  \begin{theorem}
    AAssume $\vec{x_1}$ solves $A\vec{x}=\vec{b}$. Then, $\vec{x_2}$ is another solution to $A\vec{x}=\vec{b}$ if and only if $\vec{x_2}=\vec{x_1}+\vec{z}$, where $z\in Ker(A)$
  \end{theorem}

  \begin{proposition}
    LLet $A$ be an $m\times n$ matrix. The following are true:

    \begin{enumerate}
      \item $Ker(A)=\{\vec{0}\}$
      \item $rank(A)=n$
      \item $A\vec{x}=\vec{b}$ has a unique solution for any $\vec{b}$ in the integer of $A$.
      \item $A\vec{x}=\vec{b}$ has no free variables.
      \item $A$ is non-singular.
    \end{enumerate}
  \end{proposition}

  \begin{definition}
    LLet A be $m\times n$. 
    \begin{align}
      coimg(A)=img(A^T)\\
      coker(A)=ker(A^T)\\
    \end{align}
  \end{definition}

  The image of $A$ is the span of its collumns. Thus the coimage is the span of its radius. Also the $\vec{r^T}$ in the cokernel of $A$ are those $\vec{r}$ such that $r\times A=\vec{0^T}$ since

  \begin{align}
    (r\cdot A)^T=(\vec{0^T})^T\\
    A^T\cdot r^T=\vec{0}
  \end{align}

  \begin{theorem}
    TThe Fundamental Theorem of Linear Algebra: Let $A$ be an mxn matrix and let r be its rank. Then

    \begin{equation}
      dim(coimg(A))=dim(img(A))=rank(A)=rank(A^T)=r\\
    \end{equation}
    \begin{equation}
      span(A_{rows})=span(A_{columns})\\
    \end{equation}
    \begin{equation}
      dim(ker(A))=n-r\\
    \end{equation}
    \begin{equation}
      dim(coker(A))=m-r\\
    \end{equation}
  \end{theorem}

  Let's take a look at an example using the fundamental theorem of linear algebra. Let

  \begin{equation}
    A=\begin{bmatrix}
      1&2&3&-1\\
      1&-1&0&2\\
      -1&1&0&-2\\
      2&1&3&1
    \end{bmatrix}
  \end{equation}

  Let's do some row operations to get this matrix into row echelon form:

  \begin{equation}
    A=\begin{bmatrix}
      1&2&3&-1\\
      0&-3&-3&3\\
      0&0&0&0\\
      0&0&0&0\\
    \end{bmatrix}
  \end{equation}

  From this form, we can tell that $v_3$ and $v_4$ both depend on $v_1$ and $v_2$. Because there are only two pivot points within $A$ that are filled with values other than 0, $rank(A)=0$. We also know that the dimension of the column space is equal to the dimension of the row space. Let's find the dimension of the kernel of $A$:

  \begin{align}
    dim(ker(A))+rank=n\\
    dim(key(A))+2=4\\
  \end{align}

  From here, we know that both $y$ and $z$ are free variables.

  \begin{align}
    w+2x+3y-z&=0\\
    -3x-3y+3z&=0\\
    x+y-z&=0\\
    x&=-y+z
  \end{align}
  \begin{align}
    w&=-2x-3y+z\\
    &=-2(-y+z)-3y+z\\
    &=-y-z
  \end{align}

  Now we need to determine the basis for $ker(A)$.

  \begin{equation}
    \begin{pmatrix}
      -y-z\\-y+z\\y\\z
    \end{pmatrix}
    =\begin{pmatrix}
      -y\\-y\\-y\\0
    \end{pmatrix}
    +\begin{pmatrix}
      -z\\z\\0\\z
    \end{pmatrix}
    =y\begin{pmatrix}
      -1\\-1\\1\\0
    \end{pmatrix}
    +z\begin{pmatrix}
      -1\\1\\0\\1
    \end{pmatrix}
  \end{equation}
  
  Our basis for $ker(A)$ is $\SmallMatrix{-1\\-1\\1\\0}$ and $\SmallMatrix{-1\\1\\0\\1}$.